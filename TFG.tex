\documentclass[12pt]{report} %fuente a 12pt

% MÁRGENES: 2,5 cm sup. e inf.; 3 cm izdo. y dcho.
\usepackage[
a4paper,
vmargin=2.5cm,
hmargin=3cm
]{geometry}

\usepackage{fontspec} 
\defaultfontfeatures{Ligatures=TeX}
\setmainfont{Times New Roman}

\usepackage[spanish]{babel} 
\usepackage[spanish]{translator}

\usepackage{csquotes}
\setquotestyle[american]{english}

% DEFINICIÓN DE COLORES para portada y listados de código
\usepackage[table]{xcolor}
\definecolor{azulUC3M}{RGB}{0,0,102}
\definecolor{gray97}{gray}{.97}
\definecolor{gray75}{gray}{.75}
\definecolor{gray45}{gray}{.45}
\definecolor{citecolor}{RGB}{5, 99, 193}
\definecolor{colorPrimary}{HTML}{1976d2}
\definecolor{colorPrimaryVariant}{HTML}{004ba0}
\definecolor{colorOnPrimary}{HTML}{FFFFFF}
\definecolor{colorSecondary}{HTML}{03a9f4}
\definecolor{colorSecondaryVariant}{HTML}{007ac1}
\definecolor{colorOnSecondary}{HTML}{000000}
\definecolor{codebackground}{RGB}{240, 240, 235}


% Colores de los diagramas de Gantt
\definecolor{bar1}{HTML}{CD7B80}
\definecolor{bar2}{HTML}{303378}

% INTERLINEADO: Estrecho (6 ptos./interlineado 1,15) o Moderado (6 ptos./interlineado 1,5)
\renewcommand{\baselinestretch}{1.15}
\parskip=6pt

% ENLACES
\usepackage[hyphens]{url}
\usepackage{hyperref}
\hypersetup{
	colorlinks = true,
	citecolor = citecolor,
	linkcolor = black, % enlaces a partes del documento (p.e. índice) en color negro
	urlcolor = blue} % enlaces a recursos fuera del documento en azul

% EXPRESIONES MATEMATICAS
\usepackage{amsmath,amssymb,amsfonts,amsthm}

% diseño de PIE DE PÁGINA
\usepackage{fancyhdr}
\pagestyle{fancy}
\fancyhf{}

\renewcommand{\headrulewidth}{0.4pt}% Default \headrulewidth is 0.4pt
\renewcommand{\footrulewidth}{0.4pt}% Default \footrulewidth is 0pt
\rfoot{\thepage}
\fancypagestyle{plain}{\pagestyle{fancy}}

% PAQUETES PROPIOS
\usepackage{makecell}
\usepackage{pgfgantt}
\usepackage{rotating}
\usepackage{pdflscape}
\usepackage{pgfplots} 
\usepackage{tikz}
\usepackage{pgf-pie}  
\usepackage{caption}
\usepackage{subcaption}
\usepackage{comment}
\usepackage{tabularx}
\newcolumntype{Y}{>{\centering\arraybackslash}X}
\usepackage{pgf-umlsd}


\newcommand{\newthreadShift}[4][gray!30]{
	\newinst[#4]{#2}{#3}
	\stepcounter{threadnum}
	\node[below of=inst\theinstnum,node distance=0.8cm] (thread\thethreadnum) {};
	\tikzstyle{threadcolor\thethreadnum}=[fill=#1]
	\tikzstyle{instcolor#2}=[fill=#1]
}


\usepackage{minted}
\usepackage{mdframed}
\usepackage{appendix}

\makeatletter
\tikzset{
	dot diameter/.store in=\dot@diameter,
	dot diameter=3pt,
	dot spacing/.store in=\dot@spacing,
	dot spacing=3pt,
	dots/.style={
		line width=\dot@diameter,
		line cap=round,
		dash pattern=on 0pt off \dot@spacing
	}
}
\makeatother

% DISEÑO DE LOS TÍTULOS de las partes del trabajo (capítulos y epígrafes o subcapítulos)
\usepackage{titlesec}
\usepackage{titletoc}
\titleformat{\chapter}[block]
{\large\bfseries\filcenter}
{\thechapter.}
{5pt}
{\MakeUppercase}
{}
\titlespacing{\chapter}{0pt}{0pt}{*3}
\titlecontents{chapter}
[0pt]                                               
{}
{\contentsmargin{0pt}\bfseries\thecontentslabel.\enspace\uppercase}
{\contentsmargin{0pt}\bfseries\uppercase}                        
{\titlerule*[.7pc]{.}\bfseries\contentspage}                 

\titleformat{\section}
{\fontsize{14pt}{16.8pt}\selectfont\bfseries}
{\thesection.}
{5pt}
{}
\titlecontents{section}
[5pt]                                               
{}
{\contentsmargin{0pt}\thecontentslabel.\enspace}
{\contentsmargin{0pt}}
{\titlerule*[.7pc]{.}\contentspage}

\titleformat{\subsection}
{\normalsize\bfseries}
{\thesubsection.}
{5pt}
{}
\titlecontents{subsection}
[10pt]                                               
{}
{\contentsmargin{0pt}                          
	\thecontentslabel.\enspace}
{\contentsmargin{0pt}}                        
{\titlerule*[.7pc]{.}\contentspage}  

% DISEÑO DE TABLAS. Puedes elegir entre el estilo para ingeniería o para ciencias sociales y humanidades. Por defecto, está activado el estilo de ingeniería. Si deseas utilizar el otro, comenta las líneas del diseño de ingeniería y descomenta las del diseño de ciencias sociales y humanidades
\usepackage{multirow} % permite combinar celdas 
\usepackage{caption} % para personalizar el título de tablas y figuras
%\usepackage{floatrow} % utilizamos este paquete y sus macros \ttabbox y \ffigbox para alinear los nombres de tablas y figuras de acuerdo con el estilo definido. Para su uso ver archivo de ejemplo 
\usepackage{array} % con este paquete podemos definir en la siguiente línea un nuevo tipo de columna para tablas: ancho personalizado y contenido centrado
\newcolumntype{P}[1]{>{\centering\arraybackslash}p{#1}}
\DeclareCaptionFormat{upper}{#1#2\uppercase{#3}\par}

% Diseño de tabla para ingeniería
\captionsetup[table]{
	format=hang,
	name=Tabla,
	justification=centering,
	labelsep=period,
	width=.75\linewidth,
	labelfont=small,
	font=small,
}

% DISEÑO DE FIGURAS. Puedes elegir entre el estilo para ingeniería o para ciencias sociales y humanidades. Por defecto, está activado el estilo de ingeniería. Si deseas utilizar el otro, comenta las líneas del diseño de ingeniería y descomenta las del diseño de ciencias sociales y humanidades
\usepackage{graphicx}
\graphicspath{{imagenes/}} %ruta a la carpeta de imágenes

% Diseño de figuras para ingeniería
\captionsetup[figure]{
	format=hang,
	name=Figura,
	singlelinecheck=off,
	labelsep=period,
	labelfont=small,
	font=small		
}

% NOTAS A PIE DE PÁGINA
\usepackage{chngcntr} %para numeración contínua de las notas al pie
\counterwithout{footnote}{chapter}

%BIBLIOGRAFIA

\usepackage[backend = biber, style=ieee]{biblatex}

\addbibresource{bibliografia.bib}

% Añadimos las siguientes indicaciones para mejorar la adaptación de los estilos en español

\DefineBibliographyStrings{spanish}{
	url = {\adddot\space[En línea]\adddot\space Disponible en:}
}

\DefineBibliographyStrings{spanish}{
	urlseen = {acceso:}
}

\DefineBibliographyStrings{spanish}{%
	andothers = {et\addabbrvspace al\adddot}
}

\usepackage[titles]{tocloft}
\renewcommand\cftchapfont{\fontsize{14pt}{16.8pt}\selectfont\bfseries}
\renewcommand{\cftchapleader}{\cftdotfill{\cftdotsep}}

%-------------
%	DOCUMENTO
%-------------

\begin{document}
	\pagenumbering{roman} % Se utilizan cifras romanas en la numeración de las páginas previas al cuerpo del trabajo
	
	%----------
	%	PORTADA
	%----------	
	\begin{titlepage}
		\color{azulUC3M}
		\fontsize{14pt}{16.8}\selectfont
		\begin{center}
			\begin{figure}[H] %incluimos el logotipo de la Universidad
				\makebox[\textwidth][c]{\includegraphics[width=16cm]{Logo-UC3M-nuevo.png}}
			\end{figure}
			\vspace{0.3cm}
			\begin{Large}
				Grado Universitario en Ingeniería en Tecnologías de Telecomunicación\\			
				Curso 2020-2021\\
				\vspace{2cm}		
				\textsl{Trabajo Fin de Grado}
			\end{Large}
			\begin{center}
				\Huge Desarrollo de una aplicación móvil para la comunicación entre estudiantes en entornos de aprendizaje híbridos
			\end{center}
			\vspace*{0.5cm}
			\rule{10.5cm}{0.1mm}\\
			\vspace*{0.9cm}
			{\LARGE Martín Mateos Sánchez}\\ 
			\vspace*{1cm}
			\begin{Large}
				Tutor\\
				Carlos Alario Hoyos\\
				Leganés, 20 de septiembre de 2021\\
			\end{Large}
		\end{center}
		\vfill
		\color{black}
		% si nuestro trabajo se va a publicar con una licencia Creative Commons, incluir estas líneas. Es la opción recomendada.
		\includegraphics[width=4.2cm]{imagenes/creativecommons.png}\\ %incluimos el logotipo de creativecommons
		Esta obra se encuentra sujeta a la licencia Creative Commons\\ \textbf{Reconocimiento - No Comercial - Sin Obra Derivada}
	\end{titlepage}
	
	\newpage %página en blanco o de cortesía
	\thispagestyle{empty}
	\mbox{}
	
	%----------
	%	RESUMEN Y PALABRAS CLAVE
	%----------	
	\renewcommand\abstractname{\large\bfseries\filcenter\uppercase{Resumen}}
	\begin{abstract}
		\thispagestyle{plain}
		\setcounter{page}{3}
		
		\vspace{0.5cm}
		
		Este Trabajo de Fin de Grado surge en el contexto de la pandemia mundial causada por el SARS-CoV-2, o nuevo coronavirus, también denominado COVID-19. El proyecto tiene como objetivo crear un canal de comunicación entre estudiantes y docentes para su uso en los centros educativos de España añadiendo diferentes tipos de herramientas para potenciarla. Para dar forma a esta propuesta se ha desarrollado una aplicación para dispositivos Android llamada CoordinApp, la cual se clasifica como un VLE o EVA (\textit{Virtual Learning Environment} - Entorno Virtual de Aprendizaje).\par
		
		Para organizar a estudiantes y docentes se divide la estructura de trabajo en \textit{grupos} (asociaciones de estudiantes y las personas docentes de la asignatura, con chat integrado), \textit{asignaturas} (conjunto de grupos) y \textit{cursos} (conjunto de asignaturas).
		
		Para que los y las docentes puedan obtener realimentación de los y las estudiantes tendrán la opción de obtener la misma mediante diferentes tipos de actividades integradas en la app, pudiendo así evaluar el rendimiento general de los grupos y visualizar el peso de la contribución de cada estudiante (número de mensajes enviados por chat, resultados de las actividades integradas, etc), y poder valorar estos factores en la nota de evaluación o como se considere necesario.\par
		
		Se definen tres tipos de roles: administrador (tiene control total sobre la base de datos), docente (tiene control para ver estadísticas de sus estudiantes y gestionar los grupos de trabajo, además de evaluarles) y estudiante (tiene control de crear grupos con permiso de el o la docente).\par
		
		Para obtener una valoración inicial del proyecto se analizan los resultados de un estudio con una muestra de población a la que se le ha dado la oportunidad de probar la aplicación.
		
		\vspace{1cm}
		
		\noindent \textbf{Palabras clave:}
		Entorno Virtual de Aprendizaje, Aprendizaje colaborativo, Android, app, coordinación, cooperación, educación, COVID-19. 
		
		\vfill
		
		\end{abstract}
	
	\newpage %página en blanco o de cortesía
	\thispagestyle{empty}
	\mbox{}
	
	\renewcommand\abstractname{\large\bfseries\filcenter\uppercase{Abstract}}
	\begin{abstract}
		\thispagestyle{plain}
		\setcounter{page}{4}
		\vspace{0.5cm}
	    This work has been made in the context of the global pandemic caused by SARS-CoV-2 or new coronavirus, also known as COVID-19. The main purpose of this project is to create a communication channel between students and teachers for its use in educational facilities of Spain by adding different types of tools to enhance it. To do so, an app for mobile devices running Android has been developed named CoordinApp, which is a VLE (Virtual Learning Environment).\par
	    
	    To organize students and teachers, the work structure is divided into \textit{groups} (workplaces with students and the teacher of the subject, with chat integrated), \textit{subjects} (set of groups) and \textit{courses} (set of subjects).\par
	    
	    To get feedback from students teachers will be able to create different types of activities integrated within the app, and by doing so they will be able to evaluate the performance of the groups and visualize the weight of their students contribution in them (number of messages sent through the chat, results of the integrated activities, etc) and take into account these factors to evaluate them or whatever the teacher considers necessary.\par
	    
	    Three roles have been defined: administrator (has total control over the database), teacher (has the control to check the statistics of students and to manage the groups and evaluate them) and student (has the control to create groups with the supervision of the teacher).\par
	    
	    To get an initial overview of the project a study has been carried out with a group of people which has been able to test the app.
	    
	    \vspace{1cm}
	    
	    \noindent \textbf{Keywords:}
		Virtual Learning Environment, Collaborative learning, Android, app, coordination, cooperation, education, COVID-19. 
	    
	    \vfill
	\end{abstract}
	
	\newpage
	
	
	\newpage % página en blanco o de cortesía
	\thispagestyle{empty}
	\mbox{}
	
	
	%----------
	%	DEDICATORIA
	%----------	
	\chapter*{\bfseries Dedicatoria}
	
	\setcounter{page}{5}
	
	% ESCRIBIR LA DEDICATORIA AQUÍ	
	
	\noindent A mis padres, que lo han pasado peor que yo durante esta carrera y ya se merecen un respiro.\par
	
	\noindent A Elena, por estar siempre ahí en los momentos más duros.\par
	
	\noindent A todos los profesores y todas las profesoras, que con su forma de enseñar y su incansable búsqueda de lo mejor que hay en mí consiguieron inculcarme lo que hoy sé que es mi vocación. La educación nos hace libres e iguales, y una nueva forma de enseñar es posible.\par\vspace{0.5cm}
	
	\noindent A todos ellos, gracias.
	
	\vfill
	
	\newpage % página en blanco o de cortesía
	\thispagestyle{empty}
	\mbox{}
	
	
	%----------
	%	ÍNDICES
	%----------	
	
	%--
	% Índice general
	%-
	\renewcommand{\contentsname}{\bfseries Índice general}
	\tableofcontents
	\thispagestyle{fancy}
	
	\newpage % página en blanco o de cortesía
	\thispagestyle{empty}
	\mbox{}
	
	%--
	% Índice de figuras. Si no se incluyen, comenta las líneas siguientes
	%-
	\cftsetindents{figure}{0em}{3.5em}
	\renewcommand{\listfigurename}{\bfseries Índice de figuras}
	{%
    \let\oldnumberline\numberline%
    \renewcommand{\numberline}{\figurename~\oldnumberline}%
    \listoffigures%
    }
    
	\thispagestyle{fancy}
	\newpage % página en blanco o de cortesía
	\thispagestyle{empty}
	\mbox{}
	
	%--
	% Índice de tablas. Si no se incluyen, comenta las líneas siguientes
	%-
	\cftsetindents{table}{0em}{3.5em}
	\renewcommand{\listtablename}{\bfseries Índice de tablas}
	{%
    \let\oldnumberline\numberline%
    \renewcommand{\numberline}{Tabla~\oldnumberline}%
	\listoftables%
    }
	\thispagestyle{fancy}
	
	\newpage % página en blanco o de cortesía
	\thispagestyle{empty}
	\mbox{}
	
	\renewcommand*\footnoterule{}
	
	%----------
	%	TRABAJO
	%----------	
	\clearpage
	\pagenumbering{arabic} % numeración con múmeros arábigos para el resto de la publicación	
	
	% COMENZAR A ESCRIBIR EL TRABAJO
	\chapter{\bfseries Introducción}
	
	Este Trabajo de Fin de Grado centra su estudio en las consecuencias que la pandemia del COVID-19 ha supuesto para profesores, profesoras, alumnos y alumnas, y se intenta dar una solución a la falta de unanimidad en el uso de las herramientas de comunicación con estudiantes que los centros decidieron adoptar para mitigar el impacto de la pandemia en su formación. En este Trabajo de Fin de Grado se analiza el impacto de la pandemia en el caso particular de la educación española y qué consecuencias ha tenido el nuevo modelo de enseñanza tanto para los y las estudiantes y los profesores y las profesoras y por qué, y las ventajas y desventajas de cambiar a un modelo de aprendizaje híbrido pudiendo combinar aspectos de un sistema presencial y un sistema on-line para evitar que situaciones similares ocurran en un futuro.\par
	
	El intento de solución consiste en el desarrollo de una aplicación para dispositivos móviles con sistema operativo Android llamada CoordinApp. CoordinApp se clasifica como un VLE o EVA (\textit{Virtual Learning Environment} - Entorno Virtual de Aprendizaje) donde docentes y estudiantes pueden comunicarse mediante foros reducidos de chat para un seguimiento personalizado de los y las estudiantes por parte de los y las docentes, además de interactuar entre sí mediante diferentes tipos de actividades implementadas en la propia app.
	
	\section{Contexto}	
	
	La pandemia del COVID-19 ha supuesto un cambio de paradigma en la percepción de la educación por parte de la sociedad española. A pesar de los problemas que la situación de pandemia supuso para la población (de índole social, económica, etc) la preocupación por la educación en nuestro país subió de 5.2 puntos porcentuales en febrero de 2020 \cite{ref1} a 11.8 puntos porcentuales en marzo de 2020 \cite{ref2}, situándola como la sexta preocupación en nuestra sociedad para ese mes. \par
	
	En este contexto, se comenzaron a buscar herramientas de comunicación y manejo de grupos para continuar con las actividades docentes. Esto supuso un reto para las familias, ya que aproximadamente un cuarto de la población no disponía de un ordenador y un tercio de la misma carecía de un ordenador tipo tablet \cite{ref3}. También supuso un reto para el modelo educativo de nuestro país, ya que solo una décima parte del profesorado contaba con experiencia en entornos de aprendizaje híbridos \cite[p. 135]{ref4} (combinando presencialidad con modalidad on-line), calificando un 60\% del profesorado como \enquote{nula} o \enquote{baja} su formación con herramientas TIC (Tecnologías de la Información y la Comunicación, uso de dispositivos tecnológicos en el aula) \cite[p. 135]{ref4}, y es que el aprendizaje con el apoyo de las TIC, más concretamente el \textit{m-learning} (\textit{mobile learning} - aprendizaje mediante el uso de dispositivos móviles) todavía necesita integrarse en los sistemas de enseñanza en España \cite{ref5} en un mundo en el que las suscripciones de teléfonos móviles en las redes públicas de telefonía no han hecho más que crecer exponencialmente en los últimos años \cite{ref1000}.\par
	
	Por otra parte, las personas docentes de las etapas no universitarias recalcan la importancia de que exista una plataforma común que se pueda utilizar mediante móviles inteligentes, los conocidos como \textit{smartphones}. Argumentan que con ello se daría solución a la falta de medios de las familias \cite[p. 123]{ref6}, y es que los y las menores de edad en la etapa de educación secundaria cuentan en su mayoría de un dispositivo móvil y el 98.5\% de la población mayor de 16 años contaba con uno de estos dispositivos previo a la pandemia \cite{ref3}.
	
	\section{Motivación}
	
	La motivación para realizar este Trabajo de Fin de Grado surge durante la pandemia mundial causada por el COVID-19 entre los años 2019 y 2021, pudiendo no haber surgido en otro contexto. Es en este período en el cual la creciente preocupación por la educación en nuestra sociedad ha abierto la puerta al debate de nuevas metodologías educativas incorporando a gran escala las TIC a nuestro sistema educativo, reflejado en las demandas el cuerpo docente referentes a estas tecnologías, como obtener más formación con ellas \cite[p. 122]{ref6}.\par
	
	Estas tecnologías desarrollan la motivación del alumnado en las clases y aumentan su deseo a participar en ellas \cite{ref7}, siendo objeto de interés investigar cómo introducir estas herramientas beneficiosas en el sistema para que la etapa educativa sea una etapa de crecimiento y desarrollo personal.\par
	
	Otra de las motivaciones que ha llevado a la realización de este Trabajo de Fin de Grado es analizar la percepción del uso de la tecnología en las clases. Nuestra sociedad ve la tendencia de usar dispositivos electrónicos en el aula como algo generalmente negativo \cite{ref8}, contradiciendo los estudios que afirman que su uso es beneficioso para el aprendizaje. Se considera que la forma de conseguir que las TIC sean algo beneficioso para su uso educativo es contar con una formación adecuada con ellas fomentada en los centros educativos, más que prohibir su uso por parte de la Administración (principalmente en la etapa de secundaria) en un país en el que prácticamente la totalidad de los y las estudiantes de 15 años a fecha del año 2020 disponen de un dispositivo móvil \cite{ref9}.   
	
	\section{Objetivos}
	
	El principal objetivo de este Trabajo de Fin de Grado consiste en desarrollar una aplicación para dispositivos móviles con sistema operativo Android llamada CoordinApp, aprovechando la gran disponibilidad de estos dispositivos entre la población.\par 
	
	Se pretende que la aplicación sirva como caso de estudio práctico para la implantación de una plataforma educativa unificada en España atendiendo a las demandas de los y las docentes y liberar el código mediante una licencia de código abierto para que cualquier persona pueda contribuir a su desarrollo y mejora.  Dicha aplicación se cataloga como un Entorno Virtual de Aprendizaje. \par
	
	También se pretende que toda la comunicación entre estudiantes y docentes fuera del aula se realice mediante la aplicación propuesta, evitando herramientas de empresas privadas como correo electrónico o aplicaciones de mensajería, pudiendo así el profesor realizar una valoración de la participación y el desempeño de los y las estudiantes de forma más focalizada y directa.\par  
	
	Por otra parte, atendiendo a la L.O 3/2020 que dicta: \enquote{La ciudadanía reclama un sistema educativo moderno, más abierto, menos rígido, multilingüe y cosmopolita que desarrolle todo el potencial y talento de nuestra juventud}\footnote{Ley Orgánica 3/2020, de 29 de diciembre, por la que se modifica la Ley Orgánica 2/2006, de 3 de mayo, de Educación (BOE núm. 340, de 30 de diciembre de 2020)} la aplicación desarrollada busca adaptar las necesidades de cada estudiante. Es por esto que la persona docente de la asignatura puede crear grupos de trabajo reducidos atendiendo a criterios que considere, como por ejemplo crear un grupo de trabajo con estudiantes que tienen más probabilidades de ser más participativos en dicho grupo que en otro diferente, o crear grupos de trabajo de forma aleatoria para observar en qué grupo una persona estudiante podría contribuir de forma positiva, y se podrá enviar cuestionarios personalizados atendiendo a las habilidades de los y las participantes del grupo.\par
	
	El proyecto busca atender las demandas del R.D 126/2014, el cual regula el currículo básico de educación primaria y en el que se hace referencia al uso de las TIC en la asignatura de Lengua Castellana y Literatura\footnote{Real Decreto 126/2014, de 28 de febrero, por el que se establece el currículo básico de la Educación Primaria (BOE núm. 52, de 1 de marzo de 2014)}  y del R.D 1105/2014, el cual enfatiza que el uso de las TIC debe ser obligatorio en todos los niveles educativos, desde 1º de ESO hasta 2º de Bachillerato\footnote{Real Decreto 1105/2014, de 26 de diciembre, por el que se establece el currículo básico de la Educación Secundaria Obligatoria y del Bachillerato (BOE núm. 3, de 3 de enero de 2015)}. 
	
	\section{Marco regulador}
	
	Debido a que este Trabajo de Fin de Grado propone una solución práctica en forma de aplicación móvil al problema que se plantea hay que tener en cuenta los problemas legales que dicha solución puede suponer tanto para la persona desarrolladora de la aplicación como para las personas que van a hacer uso de ella, además de la legalidad de la propia aplicación. \par
	
	Se pretende distribuir la aplicación comenzando en un estado miembro de la Unión Europea como es España, para después pasar al resto de países miembros y a largo plazo distribuir el proyecto de CoordinApp a todos los países del mundo. Es por esto que se estudiará el marco regulador referente a España, que debe cumplir con la legalidad vigente de la Unión Europea.\par
	
	Como la aplicación va a ser distribuida a terceros (los centros educativos, que serán los que obtengan la licencia de utilizar CoordinApp) es conveniente registrar la aplicación como un objeto de propiedad intelectual. El R.D 1/1996 en su art. 10 señala: \enquote{Son objeto de propiedad intelectual [...] i) Los programas de ordenador}\footnote{Real Decreto Legislativo 1/1996, de 12 de abril, por el que se aprueba el texto refundido de la Ley de Propiedad Intelectual, regularizando, aclarando y armonizando las disposiciones legales vigentes sobre la materia (BOE núm. 97, de 22 de abril de 1996)}, y, debido a que un teléfono móvil es \textit{de facto} un ordenador, CoordinApp se puede considerar una obra en forma de programa informático. Si nos acogemos a dicha ley, por su art. 5 el autor sería la persona natural que crea la obra (aplicación móvil), y tiene todos los derechos sobre la misma, reflejado así en el art. 2. En este caso sería el único programador que ha desarrollado la aplicación. \par
	
	Sin embargo, será más práctico dotar al proyecto de CoordinApp de una GNU o LPG (\textit{General Public License} - Licencia Pública General) debido a la naturaleza que tiene el proyecto, pudiendo cualquier persona extender y adaptar o modificar la aplicación a las necesidades de cada centro educativo, protegiendo el proyecto de CoordinApp de intentos de apropiación mediante el uso que las GNUs hacen del \textit{Copyleft}.\par
	
	Para almacenar la información de los usuarios de CoordinApp se hace uso de la base de datos Google Firestore, propiedad de Google. Cualquier uso de datos tanto de particulares como de empresas en España debe de cumplir con la L.O 3/2018\footnote{Ley Orgánica 3/2018, de 5 de diciembre, de Protección de Datos Personales y garantía de los derechos digitales (BOE núm. 294, de 6 de diciembre de 2018)}, la cual cumple con el Reglamento General de Protección de Datos o RGPD de la Unión Europea. Los artículos de dicho reglamento más relevantes por la naturaleza del proyecto serían el art. 5, que indica que los datos deben de ser confidenciales, el art. 6, referente al tratamiento de los datos basado en el consentimiento del afectado (es decir, el usuario debe conocer que se están recopilando datos de su actividad en la aplicación) y el art. 7, que habla sobre el consentimiento de los menores de edad, el cual dicta que si los usuarios son mayores de 14 años son ellos los que pueden dar su consentimiento salvo excepciones, y si son menores de 14 años requerirán del consentimiento de los padres o de tutores legales. También hay que tener en cuenta los artículos del Título IV de la ley, referentes a las transferencias internacionales de datos, ya que la base de datos utilizada en el proyecto hace mención al Reglamento General de Protección de Datos de la UE y a la Ley de Privacidad del Consumidor de California. Si bien ambas leyes tienen inconsistencias entre ellas \cite{ref13} Google se compromete a \enquote{contar con un fundamento legal para las transferencias de datos de conformidad con las leyes de protección de datos aplicables}.\par
	
	La aplicación hace uso de ilustraciones en formato SVG de terceros obtenidas de la página web Flaticon, sitio web operado por Freepik Company S.L. Se permite a los usuarios descargar estas ilustraciones bajo las condiciones de uso del sitio web \cite{ref1101}. Al descargar una ilustración se recomienda atribuirla a su autor haciendo mención a la página de su porfolio.
	
	\section{Impacto socio-económico}
	
	Para estudiar el impacto que este Trabajo de Fin de Grado tiene en la sociedad se van a estudiar tres dimensiones: la ambiental, la económica y la social. La tabla \ref{tbl:table1} muestra las puntuaciones de dichas dimensiones, según la notación \textit{Puntuación mínima} : \textit{Puntuación máxima} que se le puede otorgar a cada categoría de las dimensiones (vida útil y riesgos).
	
	\begin{figure}[H]
	    \centering
		\begin{tabular}{|c|c|c|}
			\cline{2-3}
			\multicolumn{1}{c|}{}& \textbf{Vida útil} & \textbf{Riesgos} \\ \hline
			\textbf{Ambiental} & \makecell[c]{Huella ecológica\\ 0 : 20} & \makecell[c]{Riesgos ambientales\\ $-$20 : 0}
			\\ \hline
			\textbf{Económico} & \makecell[c]{Plan de viabilidad\\ 0 : 20} & \makecell[c]{Riesgos económicos\\ $-$20 : 0}
			\\ \hline
			\textbf{Social} & \makecell[c]{Impacto social\\ 0 : 20} & \makecell[c]{Riesgos sociales\\ $-$20 : 0}\\ \hline
			\textbf{Puntuación total} & \multicolumn{2}{c|}{$-$60 : 60} \\ \hline
		\end{tabular}
		\captionof{table}{Matriz de sostenibilidad de la aplicación del resultado del proyecto}
		\label{tbl:table1}
	\end{figure}
	
	Se realiza una estimación de las consecuencias \textit{a posteriori} que tendrá la aplicación práctica del proyecto. Se utilizará el baremo de la tabla \ref{tbl:table2} para su calificación. En caso de obtenerse una puntuación entre dos rangos de sostenibilidad se clasificará al proyecto como un proyecto sosteniblemente híbrido entre esos dos rangos.
	\begin{figure}[H]
		\begin{tabular}{|c|c|c|c|}
			\hline
			\textbf{Nada sostenible} &\textbf{Poco sostenible} & \textbf{Medianamente sostenible} & \textbf{Muy sostenible} \\ \hline
			$-$60 : $-$30 & $-$30 : 0 & 0 : 30 & 30 : 60 \\ \hline
		\end{tabular}
		\captionof{table}{Sostenibilidad del proyecto según su puntuación total}
		\label{tbl:table2}
	\end{figure}
	
	\subsection{Dimensión ambiental}
	
	\begin{itemize}
		\item \textbf{Huella ecológica}: Previo a la pandemia, un 89.6\% de las personas docentes en España tenía experiencia únicamente con el modelo presencial. Utilizando este porcentaje, y teniendo en cuenta los datos del número de docentes y estudiantes actual en España \cite{ref14}, además de tener en cuenta el número de toneladas de CO2 por persona debido al transporte en nuestro país \cite{ref15} se ha realizado una estimación de la huella de carbono causada por este modelo educativo en cuanto a desplazamientos:
		
		\begin{figure}[H]
		    \centering
			\begin{tabular}{|c|c|c|}
				\hline
				\textbf{Tipo de perfil} & \textbf{Número de personas} & \textbf{Total de toneladas de CO2} \\ \hline
				Docente & 882\,625 & \multirow{2}{*}{13\,687\,403}\\
				Estudiante & 9\,894\,858 & \\ \hline
			\end{tabular}
			\captionof{table}{Emisiones de CO2 por desplazamientos de docentes y estudiantes}\label{tbl:table3}
		\end{figure}
		
		Si se optase por un modelo educativo híbrido en el que las personas estudiantes tuviesen que ir como mucho tres días de la semana al instituto podríamos reducir las emisiones un 40\%, evitando los desplazamientos dos días a la semana, lo cual nos dejaría un total de 8\,212\,442 toneladas equivalentes de CO2.\par
		
		Sin embargo, hay que tener en cuenta que esta solución requeriría de un nuevo factor de consumo: el de los centros de datos. Suponiendo que Google instaurase un nuevo centro de datos en Europa para cubrir la nueva demanda, y sabiendo que el consumo de datos representa el 1\% del consumo de energía a nivel mundial \cite{ref16}, si extrapolamos los datos al caso Español por el CO2 emitido por la producción de energía \cite{ref17} tendríamos un total de 453\,840 toneladas extra de CO2 emitido en el peor de los casos, generando un total de 8\,666\,282 toneladas de CO2 equivalente con un modelo híbrido de educación, siendo menos que las toneladas emitidas con un modelo completamente presencial. Esto supone una reducción total en la huella de carbono del 29.33\% por la educación. Es por esto que para la huella ecológica, siendo una reducción del 100\% un máximo de 20 puntos, le damos una puntuación positiva de 5.87 puntos.
		
		\item \textbf{Riesgos ambientales}: Como hemos comentado, con un modelo educativo híbrido en el peor de los casos se emitirían 453\,840 toneladas de CO2, eliminando completamente la huella ecológica de los desplazamientos. Es por esto que las emisiones de un modelo completamente híbrido siempre serán menores que las emisiones de un modelo completamente presencial, y por ello el riesgo ambiental es de 0 puntos.
		
	\end{itemize}
	
	\subsection{Dimensión económica}
	\begin{itemize}
		\item \textbf{Plan de viabilidad}: España se gastó en educación en el año 2019 un total de 47\,448 millones de euros en gastos corrientes, de los cuales un 61.1\% corresponden a personal y un 8.7 \% corresponden a bienes y servicios \cite{ref18}. En este último porcentaje se encuentran los gastos energéticos para enfriar o calentar el centro educativo. Un centro educativo de media consume anualmente unos 55\,222 kWh \cite{ref19}, y debido a que hay un total de 28\,624 centros educativos \cite{ref20} tenemos un gasto energético de 1\,580\,674\,528 KWh anuales. Tomando como precio estándar 0.047 euros/KWh previo a la pandemia \cite{ref1102} el gasto correspondiente a estudiar en los centros es de 75\,413\,981 euros. Reduciendo la jornada 2 días a la semana con un modelo híbrido reduciríamos estos gastos un 40\%, con un gasto total de 45\,248\,389 euros.\par
		
		Suponiendo que un modelo híbrido usando dispositivos móviles empezaría a funcionar desde los 10 años (últimos cursos de primaria) y asumiendo que prácticamente la totalidad de los y las adolescentes de más de 15 años tienen un dispositivo móvil \cite{ref9} si se tiene en cuenta que hay 2\,523\,498 menores de entre 10 y 14 años en España \cite{ref1103}, y suponiendo que tenemos la misma cantidad de menores con la misma edad en ese rango, si se toma como referencia el porcentaje que no cuenta con un teléfono móvil tendríamos un total de 942\,776 menores que no disponen de uno de estos dispositivos. Si el estado otorgase un teléfono móvil con valor de 40 euros (como el LG RAY 5.5) a cada niño y niña para poder aplicar la metodología on-line implicaría un coste de 37\,711\,040 euros en dispositivos, suponiendo un aumento en el gasto de 7\,545\,448 euros con respecto al ahorro en electricidad. Sin embargo, el gobierno lanzó un programa para digitalizar los centros educativos españoles con un presupuesto de 260 millones de euros, de los cuales 190 son de ayudas de la Unión Europea y 70 millones de las Comunidades Autónomas \cite{ref1104}, por lo que utilizando la estrategia del ministerio se obtiene 3.13 veces el precio de utilizar la estrategia propuesta. Con estos datos, le damos al plan de viabilidad 20 puntos positivos.
		
		\item \textbf{Riesgos económicos}: Si bien el proyecto se va a realizar mediante una licencia de código libre, se necesita una financiación inicial para lanzar una primera versión al mercado. Uno de los riesgos económicos a los que nos podemos enfrentar es que el proyecto en cuestión hay que implantarlo, existiendo otras alternativas ya disponibles en los centros educativos como la plataforma Moodle, y podríamos no obtener financiación por parte de la administración. Sin embargo, no hay una plataforma educativa que tenga el monopolio, y muchos optan por otras herramientas que no sean plataformas, haciendo uso de medios más tradicionales como el correo electrónico o aplicaciones de mensajería \cite{ref4}. Si ofrecemos a este perfil de usuario una alternativa que incluya todo esto (el 59.3\% de la población) podríamos convencer a la administración de que la app puede resultar rentable para una financiación inicial. Por esto, el 40.7\% de la población no usaría nuestra aplicación. La probabilidad del riesgo es por tanto 0.47, que multiplicado por un impacto en caso de ocurrir de 0.7 puntos (muy alto) tenemos una exposición al riesgo de 0.329. Si una exposición de 1 son $-$20 puntos, tenemos una puntuación negativa de $-$6.68 puntos.
		
	\end{itemize}
		
	\subsection{Dimensión social}	
	\begin{itemize}
		\item \textbf{Impacto social}: Los principales colectivos afectados por la interrupción de la enseñanza presencial son las personas estudiantes y docentes. Esto ha supuesto un reto para las familias, principalmente en la obtención de recursos tecnológicos si se tiene en cuenta la falta de medios en cuanto a dispositivos no móviles. Esta falta de medios afecta a un 14\% del alumnado en nuestro país \cite{ref01}. Sin embargo, la disposición de un teléfono móvil es mucho mayor, por lo que realizando la aplicación para dispositivos móviles podemos reducir la brecha tecnológica y evitar la exclusión de estudiantes que no disponen de uno de estos dispositivos, garantizando que más estudiantes puedan acceder a la educación.
		Se puntúa esta ventaja con 8.6 puntos, siendo 10 puntos un 100\% de estudiantes con disposición de teléfono móvil.\newline
		Por otra parte, el modelo educativo actual no tiene en cuenta las horas de descanso que necesitan los menores de edad para tener una buena salud y poder rendir correctamente, ya que los menores tienen que atender a clases a horas muy tempranas de la mañana. Se ha demostrado que los y las estudiantes rinden mejor académicamente si se retrasa la hora de inicio de las clases, siendo un 93.5\% los y las estudiantes que no tienen suficiente descanso \cite{ref02}. Un modelo semipresencial permitiría flexibilizar este horario ya que los y las estudiantes no tendrían que desplazarse al centro de estudios. Los y las estudiantes no verían afectadas sus relaciones sociales puesto que se trata de un modelo que se combina con la presencialidad. Es por esto que esta ventaja se puntúa con 9.35 puntos positivos, siendo 10 puntos un 100\% de estudiantes que se beneficiasen del aumento del descanso.\newline
		Por otra parte, un modelo híbrido necesitaría el mismo número de docentes para cumplir con la demanda, por lo que los puestos de trabajo no se verían afectados.
		
		\item \textbf{Riesgos sociales}: Los colectivos afectados por este nuevo modelo de enseñanza podrían ser las compañías de transporte que llevan a los y las estudiantes al colegio o el personal de limpieza de los centros, ya que al estar menos concurrido el centro educativo se requeriría de menos limpieza, pudiendo reducir un 40\% del tiempo de trabajo de estos perfiles. La probabilidad de que una pandemia de estas características ocurra es del 38\% \cite{ref03}, por lo que la exposición al riesgo es de 0.38, y tiene un impacto que afecta al 40\% del tiempo de jornada de una persona con el trabajo descrito, por lo que su impacto es de 0.4, lo que nos da una exposición al riesgo de 0.152. Si una exposición de 1 son $-20$ puntos, tenemos una puntuación de $-$3.04 puntos.
		
	\end{itemize}

	La siguiente tabla resume las puntuaciones obtenidas según el análisis realizado:
	
	\begin{figure}[H]
	    \centering
		\begin{tabular}{|c|c|c|}
			\cline{2-3}
			\multicolumn{1}{c|}{}& \textbf{Vida útil} & \textbf{Riesgos} \\ \hline
			\textbf{Ambiental} & \makecell[c]{Huella ecológica\\ 5.87} & \makecell[c]{Riesgos ambientales\\ 0}
			\\ \hline
			\textbf{Económico} & \makecell[c]{Plan de viabilidad\\ 20} & \makecell[c]{Riesgos económicos\\ $-$6.68}
			\\ \hline
			\textbf{Social} & \makecell[c]{Impacto social\\ 17.95} & \makecell[c]{Riesgos sociales\\ $-$3.04}\\ \hline \hline
			\textbf{Suma} & 43.82 & $-$9.72 \\ \hline
			\textbf{Puntuación total} & \multicolumn{2}{c|}{34.1} \\ \hline
		\end{tabular}
		\captionof{table}{Puntuaciones finales del análisis socio-económico}\label{tbl:table4}
	\end{figure}
	
	Según el criterio seguido, el proyecto estaría catalogado como \textbf{muy sostenible}.

	\section{Estructura del documento}
	
	Este Trabajo de Fin de Grado cubre los siguientes capítulos de ahora en adelante:
	
	\begin{itemize}
		
		\item El segundo capítulo analiza en el Estado del Arte los conceptos del ámbito tecnológico y pedagógicos relacionados con este Trabajo de Fin de Grado, así como las tecnologías utilizadas para el desarrollo de CoordinApp.
		
		\item El tercer capítulo analiza el diseño de los requisitos funcionales y no funcionales que tiene que cumplir la aplicación para obtener la solución deseada.
		
		\item El cuarto capítulo expone cómo se han implementado los diseños descritos en el tercer capítulo, justificando el por qué se han tomado las soluciones propuestas.
		
		\item El quinto capítulo analiza los resultados de un estudio que se ha llevado a cabo con una muestra de la población del uso de la aplicación desarrollada previa a su lanzamiento.
		
		\item El sexto capítulo analiza la planificación que ha seguido el desarrollo de este Trabajo de Fin de Grado y se estudia el presupuesto estimado de su elaboración.
		
		\item El séptimo capítulo expone las conclusiones de la realización de este Trabajo de Fin de Grado y las líneas futuras que debería seguir el proyecto.  
		
	\end{itemize}
	
	\chapter{\bfseries Estado del arte}
	
	\section{Pandemia del COVID-19}
	
	COVID-19 (\textit{COronaVIrus Disease 19} o enfermedad por coronavirus 19) es el término que la Organización Mundial de la Salud (OMS) ha dado a la enfermedad causada por el SARS-CoV-2 (\textit{Sever Acute Respiratory Syndrome CoronaVirus 2}, o coronavirus de tipo 2 causante del síndrome respiratorio agudo severo), siendo este nuevo coronavirus el séptimo de los coronavirus que afectan al ser humano. Fue descubierto por primera vez en la ciudad de Wuhan, en la provincia de Hubei, China, y se estima que ya estaba en transmisión entre finales de noviembre y principios de diciembre del año 2019 \cite[p. 336]{ref24}. De los siete tipos de coronavirus existentes, los más potencialmente peligrosos son el SARS-CoV-1 surgido también en China en el año 2002, el MERS-CoV (\textit{Middle East Respiratory Syndrome CoronaVirus}, coronavirus del síndrome respiratorio de Oriente Medio) surgido en Arabia Saudita en 2012, y ahora el SARS-CoV-2, ya que son los que provocan neumonías severas con riesgo real de muerte \cite[p. 367]{ref24}. Los dos primeros coronavirus tienen respectivamente un ratio de mortalidad del 15\% y del 37\% \cite[p. 87]{ref25}, mientras que el SARS-CoV-2 tiene un ratio de mortalidad del 3.14 \%, pudiendo deberse esta mejora a los avances médicos en los últimos años \cite[p. 88]{ref25}. Sin embargo, el SARS-CoV-2 ha afectado a un número mucho mayor de personas, con alrededor de 8000 casos el SARS-CoV-1, 2494 casos el MERS-CoV, y más de 205 millones de casos de SARS-CoV-2, suponiendo un total de más de 4,4 millones de muertes confirmadas en todo el mundo a fecha de la redacción de este documento \cite{ref26}.\par 
	
	En España, el primer caso de SARS-CoV-2 ocurrió en 31 de enero de 2020 en la isla de La Gomera \cite{ref28}. La situación epidemiológica en nuestro país fue empeorando hasta que el 14 de marzo se publica el R.D 463/2020, que limita en su art. 7 la libre circulación de personas\footnote{Real Decreto 463/2020, de 14 de marzo, por el que se declara el estado de alarma para la gestión de la situación de crisis sanitaria ocasionada por el COVID-19 (BOE núm. 67, de 14 de marzo de 2020)} haciendo que la población quedase confinada en sus casas, cerrando los centros considerados de actividades no esenciales, incluidos centros educativos. En total se produjeron tres declaraciones de estado de alarma, cada una relajando las medidas de los confinamientos ahora en territorios en los que había rebrotes de contagios, hasta que el 9 de mayo de 2021 concluyó el último de los tres \cite{ref30}. Se comenzó a vacunar en España contra el SARS-CoV-2 en diciembre de 2020, encontrándose el estado de la vacunación en la etapa 3 a fecha de la redacción de este documento, la última etapa de las planificadas \cite{ref32}.
	
	\subsection{Impacto de la pandemia en la educación}
	
	Con la llegada de la pandemia, el 94\% de los estudiantes de todo el mundo se vieron directamente afectados por ella en forma de cierres tanto locales como parciales. Los países con las rentas más altas optaron por un cierre a nivel nacional de los centros educativos, mientras que los que tenían menos renta optaron por cierres localizados \cite[p. 6]{ref1106}. Para intentar solucionar esta situación, se comenzaron a implantar nuevas metodologías educativas que dependieron nuevamente de los ingresos de los países. Los que tenían rentas altas alcanzaban una enseñanza en línea de entre el 80\% y el 85\%, mientras que en los de rentas más bajas no superaba el 50\% debiéndose a la falta de medios disponibles. Entre las nuevas metodologías durante este período se encuentran el uso de la radio, la televisión y la educación en línea, con países adaptando su modelo a una semipresencialidad o modelo híbrido, con grandes diferencias del modelo implantado dependiendo del continente \cite[pp. 13-14]{ref1106}. La CEPAL (Comisión Económica para América Latina y el Caribe) presentó un informe detallado en el que se destacan las dificultades que los países de la región de América Latina y el Caribe han tenido que superar, entre ellas las desigualdades notables de los países de esta región, la falta de acceso a internet o la falta de dispositivos electrónicos por parte de las familias de los y las estudiantes \cite{ref1107}.\par
	
	Si observamos la situación en España, en el curso académico 2019-2020 los y las estudiantes de todos los niveles educativos sufrieron un confinamiento general que duró desde el 14 de marzo hasta la finalización del curso, si bien las pruebas de acceso a la universidad se realizaron de forma presencial, aunque salvo por esa excepción y ante la imposibilidad de acudir a los centros educativos el modelo que se adoptó fue completamente a distancia durante ese curso académico. Este cambio de modelo repentino vino con una serie de problemas tanto de planificación como derivados de la accesibilidad a internet y de la preparación de estudiantes y docentes para el uso de las tecnologías implantadas \cite[pp. 20-21]{ref1108}. Estos problemas fueron consecuencia del aumento del uso de dispositivos electrónicos y tecnologías que este tipo de modelo requería para su correcto funcionamiento y de las desigualdades socio-económicas existentes entre las familias españolas. Acabado el curso 2019-2020, se reflejó rápidamente que la estrategia del Ministerio de Educación era volver al sistema presencial, lanzando una serie de medidas para la vuelta a las aulas para el curso 2020-2021 \cite{ref31}. Para el comienzo del curso académico 2020-2021 el modelo híbrido predominaba en todos los niveles académicos salvo en las etapas de infantil y primaria \cite[p. 58]{ref1108}.
	
	\section{Tecnología y educación}
	
	Si bien las TIC promueven la participación en clase y facilitan el trabajo a los y las docentes, no es suficiente con incluirlas en los planes de estudio de las asignaturas o formar a los alumnos y profesores de su uso. En el caso de España, su utilización ha quedado relegado al ámbito de la organización escolar. Sin embargo, para que el uso de las TIC en el aula tenga éxito se tiene que tener en cuenta la dimensión didáctica y pedagógica de estas \cite[p. 7]{ref1109}. Es por esto que las TIC no pueden ser un fin en sí mismas, sino una herramienta que facilite el aprendizaje en el aula \cite[p. 9]{ref1109}, ya que las TIC no transforman por sí solas la interacción entre docente y estudiante, sino que su uso se centra en ayudar a la persona docente a aplicar la pedagogía en su clase \cite{ref1110}, siendo esta la razón de que deben de ir acompañadas de un plan pedagógico y de conocer cómo funcionan las interacciones entre individuos dentro de un grupo. Una de las ramas que estudia estos comportamientos es el aprendizaje colaborativo, que se ve potenciado por el correcto uso de las TIC en el aula.
	
	\subsection{Aprendizaje colaborativo}
	
	El aprendizaje colaborativo se define como el conjunto de estrategias de enseñanza y aprendizaje que promueven la colaboración de los estudiantes en pequeños grupos con el objetivo de optimizar su propio aprendizaje y el de los miembros del grupo. Para que esto ocurra se deben cumplir unos requisitos básicos dentro de los grupos de trabajo \cite{ref101}:
	
	\begin{enumerate}
		\item Interdependencia positiva\par 
		
		Los y las estudiantes tienen que tener la percepción de que no pueden tener éxito en la tarea si los demás no lo tienen.
		
		\item Responsabilidad individual\par 
		
		Los resultados de cada integrante del grupo cuentan tanto para el individuo como para el grupo. Estos resultados deben de fortalecer a las personas estudiantes, aprendiendo de sus errores por medio del apoyo y la corrección de los y las demás integrantes del grupo.
		
		\item Interacción entre individuos\par 
		
		Los y las estudiantes tienen que sentirse realizados por sus compañeros y compañeras. Esto se consigue cuando se ayudan, asisten, alientan y alaban entre ellos. Como seres sociales que somos, los beneficios de estos comportamientos se ven potenciados por las interacciones entre personas, que pueden ser de carácter verbal y no verbal, encontrando su mayor beneficio cuando los grupos son de tamaño reducido (de 2 a 4 miembros).
		
		\item Habilidades sociales\par 
		
		De nada sirve crear grupos con miembros sin habilidades sociales. Se debe enseñar a los y las integrantes del grupo liderazgo, capacidad de decisión, de crear confianza entre los miembros, de comunicación y de manejo de conflictos, igual de importantes que las habilidades académicas.
		
		\item Crítica a los resultados del grupo\par 
		
		Los y las integrantes del grupo deben debatir sobre sus decisiones, potenciando las decisiones beneficiosas y descartando las negativas. Además, tienen que tener las habilidades necesarias para identificar, definir y solucionar los conflictos que puedan surgir dentro del grupo.
		
	\end{enumerate} 
	
	Tanto docentes como estudiantes deben tener cuenta estos requisitos. De lo contrario, el aprendizaje colaborativo no se dará o será ineficiente \cite{ref102}.\par
	
	Con el auge de las nuevas tecnologías y los constantes planes de integración de las TICs en las aulas españolas por parte del Ministerio de Educación en estos últimos años \cite{ref103} el aprendizaje colaborativo se estudia desde la nueva área de las ciencias denominada CSCL (\textit{Computer-Supported Collaborative Learning} - Aprendizaje Colaborativo Apoyado por Computador) en donde las TIC son el principal eje del aprendizaje colaborativo \cite{ref104}. Podemos encontrar ejemplos de herramientas de aprendizaje colaborativo desde estudios con casos teórico-prácticos \cite{ref105} a ejemplos más comerciales como:
	
	\begin{itemize}
		
		\item \textbf{Classroomscreen}\par 
		
		Plataforma que permite a docentes coordinar grupos de trabajo de estudiantes. La persona docente tiene a su disposición un conjunto de herramientas para coordinar el trabajo que las personas estudiantes tienen que realizar entre ellos, como dibujar, hacer encuestas en tiempo real o dar instrucciones a las personas estudiantes de cómo tienen que resolver el problema planteado por la persona docente. Aunque pensado para trabajar de forma presencial en las clases, debido a la pandemia del COVID-19 la plataforma admite ahora un modo de aprendizaje a distancia \cite{ref106}.
		
		\item \textbf{Padlet}\par
		
		Plataforma utilizada para la creación de muros digitales colaborativos. La persona docente puede crear muros en los que publicar notas, fotos, vídeos, audios, enlaces, publicaciones de redes sociales, etc. Además, puede crear muros para grupos específicos de estudiantes y pedirles que trabajen sobre él para cualquier tipo de actividad, como conocerse entre ellos y ellas, debatir sobre un tema o realizar propuestas. Está disponible para dispositivos con sistema operativo iOS, Android y dispositivos Kindle \cite{ref107}.
		
		\item \textbf{Stormboard}\par 
		
		Plataforma para crear muros de tablones virtuales. A diferencia de Padlet, esta herramienta centra su uso en el \textit{brainstorming} (lluvia de ideas) en cuyos muros se pueden agrupar las propuestas en grupos e interconectarlas. Las ideas se representan como notas, pudiendo contener texto, imágenes o archivos, entre otros. Los tablones se pueden exportar a Word, Excel o PowerPoint para una evaluación mejor estructurada del plan propuesto \cite{ref108}.  
		
	\end{itemize}
	
	\subsection{M-learning}
	
	El \textit{m-learning} (\textit{mobile learning}-aprendizaje móvil) se puede definir como \enquote{cualquier oferta educativa en la que las tecnologías dominantes son dispositivos portables} \cite[p. 262]{ref109}. Esta definición incluye a una gran variedad de dispositivos, desde teléfonos móviles a consolas portátiles, siempre que estos dispositivos cuenten con medios que puedan ser catalogados como herramientas educativas que contribuyan en alguna manera a la formación de individuos. \par
	
	Una de las mayores ventajas del \textit{m-learning} es que los dispositivos pueden ser utilizados para el aprendizaje tanto dentro del aula como fuera de ella, a pesar de la tendencia de asociar dispositivos portables con educación a distancia. Por otra parte, posibilitan el acceso a la educación en zonas más aisladas geográficamente (siempre que se tenga conexión a internet) y propician el aprendizaje colaborativo \cite[p. 57]{ref110}. En cuanto a desventajas podemos mencionar que hay que conocer aspectos de la población objetivo como sus competencias tecnológicas o el reducido tamaño de las pantallas \cite[p. 58]{ref110}.\par
	
	Desarrollar herramientas eficientes de \textit{m-learning} se considera una solución a los problemas que la pandemia del COVID-19 ha supuesto para los hogares españoles debido a la gran disponibilidades de uno de los tipos de dispositivos que se pueden aplicar a \textit{m-learning} como son los dispositivos móviles inteligentes \cite{ref3}.  
	
	\subsection{Entorno Virtual de Aprendizaje}
	\label{subsec:Entorno Virtual de Aprendizaje}
	Se considera un Entorno Virtual de Aprendizaje (EVA) a un espacio \cite{ref34}: 
	
	\begin{enumerate}
		\item De información diseñado\par
		
		El espacio debe cumplir con ciertos requisitos funcionales, siendo algunos de los más importantes:
		\begin{itemize}
			\item La información debe de estar almacenada en bases de datos u otras estructuras de información
			\item La información almacenada debe de estar producida por múltiples autores
			\item Se debe informar de la autoría de los contenidos subidos
		\end{itemize}
		
		\item Social\par
		
		Tiene que haber interacción alrededor de la información que está subida en el EVA. Esta interacción puede ser síncrona o asíncrona, uno-a-uno, uno-a-muchos o muchos-a-muchos.
		
		\item Que está explícitamente representado\par
		
		La representación del entorno de trabajo tiene un impacto en el proceso de aprendizaje. Es por esto que el espacio de trabajo debe de estar bien definido.
		
		\item En el que no solo los y las estudiantes están activos, sino también los actores\par
		
		Además de los y las estudiantes otros pueden contribuir al ambiente constructivo (i.e, los y las docentes).
		
		\item Que no está restringido a la educación a distancia\par
		
		Los EVA están diseñados para apoyar el aprendizaje presencial mediante comunicación asíncrona. Además, un entorno de aprendizaje híbrido se ve reforzado gracias a la copresencialidad debido a las limitaciones que la tecnología tiene actualmente.
		
		\item Que integra tecnologías heterogéneas y múltiples enfoques pedagógicos\par
		
		Un EVA integra varias tecnologías, entre ellas funciones administrativas (administrar los usuarios de un grupo, corregir actividades didácticas, etc). Además, un EVA debe tener herramientas que faciliten el aprendizaje de las personas estudiantes. Por ejemplo, un chat personal con la persona docente para resolver dudas.
		
		\item Que se solapa con entornos físicos\par
		
		Los EVAs pueden requerir de interacciones físicas para completar una tarea. Por ejemplo, una discusión cara a cara o dibujar algo en un papel.
		
	\end{enumerate} 
	
	Aunque los Entornos Virtuales de Aprendizaje están diseñados para que el aprendizaje sea colaborativo las interacciones que dan lugar a este aprendizaje pueden no ocurrir como se espera que ocurran \cite[p. 13]{ref34}. Es por esto que los EVAs deben contar con herramientas que avisen a la persona docente de la participación de las personas integrantes de los grupos y poder identificar los roles que se crean en estos \cite{ref35}, pudiendo así adaptar los grupos y balancearlos.\par
	
	Aún sabiendo las características que tiene que cumplir un software para considerarse un EVA es conveniente buscar referentes que hayan tenido un cierto éxito para aprender las características que hacen que un EVA sea viable en el mundo real. La UNESCO publicó una lista de Entornos Virtuales de Aprendizaje que son considerados de interés por el impacto que están teniendo debido a la pandemia del COVID-19 \cite{ref36}. Algunos de los más populares y que complementan la propuesta de este trabajo son:
	
	\begin{itemize}
		\item \textbf{Edmodo}\par
		
		Edmodo es una plataforma educativa totalmente gratuita que ofrece a los y las docentes crear clases especificando la asignatura que se imparte en ella e invitando a los participantes (los y las estudiantes) que van a formar parte. Los padres y las madres también pueden acceder mediante códigos especiales para revisar las notas de sus hijos e hijas. La clase en cuestión aparecerá como un blog o muro, en donde la persona docente puede publicar avisos como si de una red social se tratase. Puede además crear tareas que deben entregarse en una fecha límite, compartir archivos de varias formas, como en un post o guardándolos en carpetas especiales y crear pruebas de evaluación cuyos resultados se pueden ver en forma de gráficos mediante un panel de estadísticas. Debido a la pandemia del COVID-19 han desarrollado un kit de herramientas para reforzar el aprendizaje a distancia \cite{ref37}. Además, la plataforma hace uso de la gamificación (técnica que traslada la mecánica de los juegos al ámbito educativo-profesional) mediante insignias y premios que la persona docente puede dar a los y las estudiantes dependiendo de su rendimiento en las actividades, con los beneficios añadidos que la propia gamificación tiene para el aprendizaje \cite{ref38}. Este rendimiento puede observarse mediante herramientas de monitorización que la plataforma tiene integradas.   	
		
		\item \textbf{ClassDojo}\par
		
		Diseñada para estudiantes en la etapa de la escuela primaria, Classdojo ofrece a los y las docentes un conjunto de herramientas para coordinar a los y las estudiantes tales como: indicar los pasos a seguir durante una actividad, reproducir música adaptada a la naturaleza de la actividad, crear grupos de trabajo de forma aleatoria para que trabajen durante la clase, medir el nivel de ruido de la clase, crear discusiones uno a uno, elegir un o una voluntario o voluntaria al azar, contador de cuenta atrás de una actividad y mostrar a los y las estudiantes un mensaje de bienvenida a la clase. Además, los estudiantes y las estudiantes pueden subir contenidos como fotos y vídeos siempre con la monitorización de los y las docentes \cite{ref39}. Mediante el uso intensivo de la gamificación, Classdojo puede ayudar mediante el refuerzo positivo a evitar las conductas disruptivas que suceden en el aula \cite{ref40}.    
		
		\item \textbf{Moodle}\par
		
		De las plataformas mencionadas anteriormente, Moodle es el Entorno Virtual de Aprendizaje que más se incentiva por parte del Ministerio de Educación de España (afirmación basada en la cantidad de material que se publica sobre esta plataforma \cite{ref41}). Al contrario que Edmodo y ClassDojo y aunque estas son gratuitas de utilizar, Moodle cuenta con una licencia GNU GPL \cite{ref42}, lo cual ofrece a las instituciones la posibilidad de instalar sus propios servidores y monitorizar directamente el tráfico y los datos que se suben a la plataforma y de modificar esta atendiendo a las necesidades propias de cada centro (funcionalidades, estética adecuada a la institución, etc). Así pues, Moodle divide sus herramientas en recursos y actividades. La persona docente puede subir recursos a la plataforma en forma de archivos divididos en carpetas para una mayor organización, y otras funcionalidades más avanzadas como la creación de páginas dentro de la plataforma con diferente tipo de contenido. Estos recursos no implican la intervención de los y las estudiantes. Por otra parte, las actividades implican la intervención directa de los y las estudiantes. Algunos ejemplos de estas actividades programables por parte de la persona docente son la creación de cuestionarios, encuestas, foros de colaboración y dudas, chats y agrupación de estudiantes en grupos de trabajo a quienes se puede enviar este tipo de actividades \cite{ref43}.
		
	\end{itemize}
	
	\section{Tecnologías utilizadas}
	
	Para el desarrollo de la aplicación para dispositivos móviles que propone este Trabajo de Fin de Grado se tuvieron en cuenta diferentes factores a la hora de la elección de las tecnologías que se iban a utilizar. Uno de los requisitos imprescindibles para la elección de estas herramientas es que todas estuviesen amparadas por los desarrolladores del sistema operativo de los dispositivos móviles para los que se iba a desarrollar la aplicación. Una vez elegido el sistema operativo objetivo, en primer lugar se buscó que dicho sistema operativo fuese el mayoritario en el mercado. En segundo lugar, se buscó que el lenguaje de programación utilizado para el desarrollo gozase de cierto estatus entre los lenguajes de programación (ampliamente usado por la comunidad, fiable, documentado). En tercer lugar, se buscó un IDE (\textit{Integrated Developement Environment} - Entorno de Desarrollo Integrado, herramienta que facilita el desarrollo software) con una alta variedad de herramientas para el desarrollo de la app. En cuarto lugar, se buscó una plataforma que ofreciese herramientas para el almacenamiento de datos de la aplicación que fuesen fácil de integrar usando el IDE elegido. Todos estos requisitos llevaron a la elección de las siguientes tecnologías:
	
	\begin{itemize}
		
		\item \textbf{Android}\par
		
		Android es el sistema operativo para dispositivos móviles desarrollado en sus inicios por la compañía Android Inc., que en el año 2005 sería adquirida por Google. El código de Android es abierto bajo una licencia Apache 2.0, permitiendo que cualquier desarrollador o desarrolladora pueda implementar sus ideas sin pasar por la burocracia correspondiente a proyectos con otro tipo de licencias más restrictivas \cite{ref111}. Por otra parte, los desarrolladores y las desarrolladoras disponen de documentación constantemente actualizada de la página oficial de desarrollo de Android. Los lenguajes de programación oficiales para el desarrollo de aplicaciones Android son Java y Kotlin, aunque desde el año 2019 se puede ver un mayor protagonismo de Kotlin por parte de Google, siendo Kotlin el lenguaje promocionado en la página oficial de desarrolladores de Android por el que un 60\% de los desarrolladores optan \cite{ref112}. Tanto Java como Kotlin son lenguajes de programación de tipo OOP (\textit{Object Oriented Programming} - Programación Orientada a Objetos). Entre las ventajas de este tipo de lenguajes de programación están la modularidad que ofrecen mediante el uso de herencia y polimorfismo. Así mismo, Java es uno de los lenguajes de programación más populares del mundo debido a su alto rango de aplicaciones \cite{ref113}. Android Studio es el IDE oficial de Android. Su gratuidad hace que cualquier persona pueda programar aplicaciones Android utilizando la variedad de herramientas que ofrece este software, entre las que se incluyen emuladores en los que probar las aplicaciones desarrolladas sin necesidad de disponer un dispositivo hardware \cite{ref114}. Por todas estas características y la gran variedad de dispositivos a precios competitivos en el mercado, Android tiene un 77\% de cuota de mercado en España a fecha de redacción de este documento \cite{ref115}. Así pues, eligiendo desarrollar aplicaciones para dispositivos con sistema operativo Android podemos llegar a la mayor parte del mercado, pudiendo ser distribuida de forma sencilla a través de la plataforma oficial de distribución de aplicaciones Android, Google Play.
		
		\item \textbf{Google Firebase}\par 
		
		Google Firebase es una plataforma que permite mediante un conjunto integrado de herramientas el desarrollo de aplicaciones móviles y web. Fue desarrollada en un principio por Firebase Inc., que más tarde fue adquirida por Google, en el año 2014. Entre algunas de sus herramientas más destacadas nos encontramos \cite{ref116}:
		
		\begin{itemize}
			\item Analíticas del uso de la aplicación
			\item Predicciones de mercado
			\item \textit{Machine Learning}
			\item Funciones en la nube
			\item Autenticación
			\item \textit{Hosting} de aplicaciones web
			\item Cloud Firestore
		\end{itemize} 
		
		Cloud Firestore es una de las bases de datos que ofrece Google Firebase. Una base de datos es una colección de información (datos) normalmente almacenada en un sistema de computación. Esta información puede estar estructurada, no estructurada o semiestructurada \cite{ref117}. Las bases de datos son necesarias para el almacenamiento de información, especialmente si esa información necesita ser escalable, como es el caso de aplicaciones informáticas que van a ser utilizadas por una gran cantidad de usuarios. Dentro de las bases de datos nos encontramos dos principales tipos: SQL y NoSQL, siendo Cloud Firestore una base de tipo NoSQL. Las bases SQL se caracterizan en que su información está estructurada en forma de tablas, mientras que las NoSQL contienen información no estructurada o semiestructurada en archivos tipo JSON con campos de clave-valor. Si bien ambos tipos de bases de datos tienen sus ventajas e inconvenientes, las bases de datos NoSQL están optimizadas para búsquedas de información rápidas y eficaces independientes del tamaño de los datos almacenados, mientras que la recopilación de la información en bases de datos SQL dependen del tamaño de los datos almacenados, llegando a ser operaciones costosas computacionalmente si las tablas de datos son lo suficientemente grandes \cite{ref118}. Si bien las bases de datos NoSQL se encuentran en auge, el tipo de la base de datos debe depender de las necesidades específicas del proyecto \cite{ref119}. Además de la eficiencia de las bases de datos NoSQL se ha decidido desarrollar este proyecto con una base de datos de este tipo por su simplicidad, su fácil integración con Android Studio y su extensa documentación tanto en forma de manuales como de vídeos \cite{ref120}.
		
	\end{itemize}
	
	\chapter{\bfseries Diseño}
	\label{chapter3}
	En este capítulo se exponen las decisiones de diseño que se han llevado a cabo con el objetivo de crear una aplicación para dispositivos Android funcional. Las decisiones de diseño se han tomado en referencia a plataformas existentes (todos los EVAs deben tener características de diseño comunes) pero también se han diseñado características nuevas sin tener en cuenta las características propias de los Entornos Virtuales de Aprendizaje ya conocidos con el objetivo de crear una aplicación que resulte diferente y atractiva para el público general.
	
	\section{Requisitos de diseño}
	
	La primera fase del diseño de una aplicación o de cualquier proyecto que incluya funcionalidades es la definición de los requisitos. En esta sección se incluye el diseño de requisitos tanto funcionales como no funcionales que debe cumplir la aplicación. Los primeros tienen el objetivo de declarar los servicios que ofrece el sistema, mientras que los últimos tienen el objetivo de declarar las propiedades del sistema.  El formato de los requisitos se muestra en la siguiente tabla:
	
    \begin{figure}[H]
        \begin{tabular}{|c|c|} \hline
            \multicolumn{2}{|c|}{\textbf{Identificador}} \\ \hline
            \textbf{Nombre} & \phantom{aaaaaaaaaaaaaaaaaaaaaaaaaaaaaaaaaaaaaaaaaaaaaaaaaaaaaaaaaaaaaaaa} \\ \hline
            \textbf{Prioridad} & \\ \hline
            \textbf{Descripción} & \\ \hline
            \textbf{Verificación} & \\ \hline
        \end{tabular}
        
        \captionof{table}{Formato de requisito}
        \label{tbl:table301}
    \end{figure}
	
    \begin{itemize}
        \item \textbf{Identificador:} Identifica de forma única al requisito.
        Formato RFX-NNN si se trata de un requisito funcional, donde $\text{X} = \text{D}$ si se corresponde a un requisito relacionado con el rol de docente, $\text{X} = \text{E}$ si se corresponde a un requisito relacionado con el rol de estudiante y $\text{X} = \text{C}$ en caso de ser un requisito funcional común para ambos roles. Formato RNF-NNN si se trata de un requisito no funcional del propio sistema. Para ambos tipos de requisitos, NNN corresponde con un número de tres dígitos que identifica al requisito. 
        
        \item \textbf{Nombre:} Nombre del requisito.
        
        \item \textbf{Prioridad:} Grado de importancia del requisito para su implementación. Puede tomar tres valores: alta, media o baja.
        
        \item \textbf{Descripción}: Definición del requisito.
        
        \item \textbf{Verificación:} Acciones que tienen relación con el requisito y comportamiento esperado de dichas acciones.
    \end{itemize}
	
    Para el caso de este proyecto los requisitos funcionales definen los servicios que se deben ofrecer tanto a docentes como a estudiantes.

	\subsection{Requisitos funcionales}
	
	\subsubsection{De docente}
    
	\begin{figure}[H]
        \begin{tabularx}{\linewidth}{|c|X|} \hline
            \multicolumn{2}{|c|}{\textbf{Identificador: RFD-001}} \tabularnewline \hline
            \textbf{Nombre} & Visualización de estadísticas de estudiantes\tabularnewline \hline
            \textbf{Prioridad} & Media\tabularnewline \hline
            \textbf{Descripción} & Se deben mostrar estadísticas individuales de las personas estudiantes\tabularnewline \hline
            \textbf{Verificación} & La persona docente selecciona a un o una estudiante de la lista de estudiantes. En caso de que la persona estudiante seleccionada esté en uno o más grupos se mostrarán las estadísticas de las personas estudiantes en esos grupos. En caso contrario, se mostrará un aviso indicando que la persona estudiante seleccionada no está en ningún grupo. La persona docente puede buscar a una persona estudiante concreta sobre la que mostrar las estadísticas\tabularnewline \hline
        \end{tabularx}
        \captionof{table}{Requisito funcional RFD-001}
        \label{tbl:table302}
    \end{figure}
    
	\begin{figure}[H]
        \begin{tabularx}{\linewidth}{|c|X|} \hline
            \multicolumn{2}{|c|}{\textbf{Identificador: RFD-002}} \tabularnewline \hline
            \textbf{Nombre} & Control total sobre las peticiones\tabularnewline \hline
            \textbf{Prioridad} & Media\tabularnewline \hline
            \textbf{Descripción} & La persona docente debe tener control sobre las peticiones de creación de grupo enviadas por los o las estudiantes\tabularnewline \hline
            \textbf{Verificación} & Al enviar una persona estudiante una petición de creación de grupo a la persona docente podrán realizar dos acciones: aceptar o eliminar la petición. En caso de aceptar la petición se creará el grupo con las personas participantes que la persona estudiante ha seleccionado. El sistema escogerá a una persona portavoz ente los participantes de forma aleatoria. En caso de eliminar la petición esta desaparecerá de las pantallas de todas las personas participantes a quienes les había sido enviada\tabularnewline \hline
        \end{tabularx}
        \captionof{table}{Requisito funcional RFD-002}
        \label{tbl:table302}
    \end{figure}
    
	\begin{figure}[H]
        \begin{tabularx}{\linewidth}{|c|X|} \hline
            \multicolumn{2}{|c|}{\textbf{Identificador: RFD-003}} \tabularnewline \hline
            \textbf{Nombre} &  Control sobre los eventos\tabularnewline \hline
            \textbf{Prioridad} &  Baja\tabularnewline \hline
            \textbf{Descripción} & La persona docente debe tener control sobre los eventos enviados\tabularnewline \hline
            \textbf{Verificación} & Los eventos enviados por la persona docente aparecerán en carpetas dependiendo del grupo al que se le hayan enviado. La persona docente puede eliminar los eventos que han sido creados\tabularnewline \hline
        \end{tabularx}
        \captionof{table}{Requisito funcional RFD-003}
        \label{tbl:table302}
    \end{figure}
    
	\begin{figure}[H]
        \begin{tabularx}{\linewidth}{|c|X|} \hline
            \multicolumn{2}{|c|}{\textbf{Identificador: RFD-004}} \tabularnewline \hline
            \textbf{Nombre} & Creación de un único grupo de forma no automática\tabularnewline \hline
            \textbf{Prioridad} & Alta\tabularnewline \hline
            \textbf{Descripción} & La persona docente puede crear un único grupo seleccionando las personas participantes que considere\tabularnewline \hline
            \textbf{Verificación} & Se pulsará la primera opción del menú desplegable de la pantalla Grupos. Si la persona docente selecciona a una persona estudiante y pulsa la acción Crear se creará un grupo privado con la persona estudiante que ha seleccionado. En caso de seleccionar más de una persona estudiante se le pedirá a la persona docente que seleccione una persona portavoz para el grupo. Si lo hace y pulsa Crear se creará el grupo con las personas estudiantes y la persona portavoz seleccionadas. En caso contrario se mostrará un mensaje de aviso indicando que se debe seleccionar una persona portavoz\tabularnewline \hline
        \end{tabularx}
        \captionof{table}{Requisito funcional RFD-004}
        \label{tbl:table302}
    \end{figure}
    
	\begin{figure}[H]
        \begin{tabularx}{\linewidth}{|c|X|} \hline
            \multicolumn{2}{|c|}{\textbf{Identificador: RFD-005}} \tabularnewline \hline
            \textbf{Nombre} &  Creación de grupos con un número específico de estudiantes\tabularnewline \hline
            \textbf{Prioridad} & Alta \tabularnewline \hline
            \textbf{Descripción} & La persona docente puede delegar al sistema la función de crear grupos con un número de estudiantes de su elección \tabularnewline \hline
            \textbf{Verificación} & Se pulsará la segunda opción del menú desplegable de la pantalla Grupos. Eligiendo la primera opción que aparecerá, la persona docente debe indicar el número de estudiantes por grupo que desea. Si pulsa la acción Crear grupos y el resto de la división es mayor o igual a dos se tendrá en cuenta la selección o no de la casilla que aparecerá en dicho menú. Si la casilla está seleccionada se introducirá a las personas estudiantes restantes en un grupo de mayor tamaño al especificado, y en caso contrario se introducirá a las personas estudiantes restantes en un grupo de un tamaño menor al especificado. En caso de que el resto de la división sea de uno se introducirá a la persona estudiante restante en un grupo de mayor tamaño al especificado independientemente de la selección o no de la casilla. Las personas estudiantes que conformen los grupos serán elegidas de forma aleatoria, así como sus portavoces\tabularnewline \hline
        \end{tabularx}
        \captionof{table}{Requisito funcional RFD-005}
        \label{tbl:table302}
    \end{figure}
    
	\begin{figure}[H]
        \begin{tabularx}{\linewidth}{|c|X|} \hline
            \multicolumn{2}{|c|}{\textbf{Identificador: RFD-006}} \tabularnewline \hline
            \textbf{Nombre} &  Creación de un número específico de grupos\tabularnewline \hline
            \textbf{Prioridad} & Alta \tabularnewline \hline
            \textbf{Descripción} & La persona docente puede delegar al sistema la función de crear un número de grupos específico\tabularnewline \hline
            \textbf{Verificación} & Se pulsará la segunda opción del menú desplegable de la pantalla Grupos. Eligiendo la segunda opción que aparecerá, la persona docente debe indicar el número de grupos que desea. Si pulsa la acción Crear grupos se creará el número de grupos especificado, con independencia del resto de la división\tabularnewline \hline
        \end{tabularx}
        \captionof{table}{Requisito funcional RFD-006}
        \label{tbl:table302}
    \end{figure}
    
	\begin{figure}[H]
        \begin{tabularx}{\linewidth}{|c|X|} \hline
            \multicolumn{2}{|c|}{\textbf{Identificador: RFD-007}} \tabularnewline \hline
            \textbf{Nombre} &  Modificación de grupos\tabularnewline \hline
            \textbf{Prioridad} & Media \tabularnewline \hline
            \textbf{Descripción} & La persona docente puede mover a los participantes de un grupo a otro\tabularnewline \hline
            \textbf{Verificación} & Se pulsará la tercera opción del menú desplegable de la pantalla Grupos. La persona docente seleccionará los dos grupos que quiere modificar, momento en el que aparecerán dos listas con las personas participantes de ambos grupos. De la primera lista la persona docente seleccionará las personas estudiantes del primer grupo que quiera mover al segundo y viceversa. Pulsando en la acción Intercambiar se mostrará cómo quedan las personas participantes finales de ambos grupos. Pulsar Modificar grupos modificará las personas participantes de los grupos. Si un grupo se queda sin portavoz en este proceso el sistema elegirá a una nueva persona portavoz de forma aleatoria\tabularnewline \hline
        \end{tabularx}
        \captionof{table}{Requisito funcional RFD-007}
        \label{tbl:table302}
    \end{figure}
    
	\begin{figure}[H]
        \begin{tabularx}{\linewidth}{|c|X|} \hline
            \multicolumn{2}{|c|}{\textbf{Identificador: RFD-008}} \tabularnewline \hline
            \textbf{Nombre} & Cambio de portavoz \tabularnewline \hline
            \textbf{Prioridad} & Media \tabularnewline \hline
            \textbf{Descripción} & La persona docente puede cambiar la persona portavoz de un grupo \tabularnewline \hline
            \textbf{Verificación} & Se deberá pulsar la acción Cambiar portavoz, momento en el que aparecerá una lista con las personas estudiantes que no son portavoces. Pulsando en uno y realizando la acción Pulsar se cambiará a la persona portavoz\tabularnewline \hline
        \end{tabularx}
        \captionof{table}{Requisito funcional RFD-008}
        \label{tbl:table302}
    \end{figure}
    
    \begin{figure}[H]
        \begin{tabularx}{\linewidth}{|c|X|} \hline
            \multicolumn{2}{|c|}{\textbf{Identificador: RFD-009}} \tabularnewline \hline
            \textbf{Nombre} & Visualización de mensajes entre estudiantes \tabularnewline \hline
            \textbf{Prioridad} & Alta \tabularnewline \hline
            \textbf{Descripción} & La persona docente puede ver el número de mensajes que envían las personas estudiantes en el grupo privado que tienen entre ellas \tabularnewline \hline
            \textbf{Verificación} & El número de mensajes enviados en la sala de chat privada de estudiantes del grupo aparecerá en la tarjeta de grupo de la persona docente\tabularnewline \hline
        \end{tabularx}
        \captionof{table}{Requisito funcional RFD-009}
        \label{tbl:table302}
    \end{figure}
    
	\begin{figure}[H]
        \begin{tabularx}{\linewidth}{|c|X|} \hline
            \multicolumn{2}{|c|}{\textbf{Identificador: RFD-010}} \tabularnewline \hline
            \textbf{Nombre} & Creación de actividad de tipo entrada de texto \tabularnewline \hline
            \textbf{Prioridad} & Alta \tabularnewline \hline
            \textbf{Descripción} & La persona docente puede crear una actividad de tipo entrada de texto\tabularnewline \hline
            \textbf{Verificación} & Se pulsará la última opción del menú desplegable de la pantalla Interactividad. Se deberá seleccionar los grupos a los que se quiere enviar la actividad, introducir un título y seleccionar la modalidad de la misma. Al pulsar en la acción Crear la actividad se enviará a los grupos designados\tabularnewline \hline
        \end{tabularx}
        \captionof{table}{Requisito funcional RFD-010}
        \label{tbl:table302}
    \end{figure}
    
	\begin{figure}[H]
        \begin{tabularx}{\linewidth}{|c|X|} \hline
            \multicolumn{2}{|c|}{\textbf{Identificador: RFD-011}} \tabularnewline \hline
            \textbf{Nombre} &  Comprobación de actividad de tipo entrada de texto\tabularnewline \hline
            \textbf{Prioridad} & Media \tabularnewline \hline
            \textbf{Descripción} & La persona docente comprobará las respuestas de los grupos o estudiantes e introducirá las calificaciones \tabularnewline \hline
            \textbf{Verificación} & La actividad de tipo entrada de texto creada aparecerá en la carpeta de actividades de los grupos a los que se haya enviado. La persona docente recibirá respuestas de las personas estudiantes, las cuales calificará de una en una en caso de que sea una actividad evaluable, mostrándose la nota media de las respuestas. La persona docente no debe tener conocimiento del nombre de la persona estudiante a la que está evaluando\tabularnewline \hline
        \end{tabularx}
        \captionof{table}{Requisito funcional RFD-011}
        \label{tbl:table302}
    \end{figure}
    
	\begin{figure}[H]
        \begin{tabularx}{\linewidth}{|c|X|} \hline
            \multicolumn{2}{|c|}{\textbf{Identificador: RFD-012}} \tabularnewline \hline
            \textbf{Nombre} &  Creación de tipo multirrespuesta\tabularnewline \hline
            \textbf{Prioridad} & Alta \tabularnewline \hline
            \textbf{Descripción} & La persona docente puede crear una actividad de tipo multirrespuesta \tabularnewline \hline
            \textbf{Verificación} & Se pulsará la segunda opción del menú desplegable de la pantalla Interactividad. Se deberá seleccionar los grupos a los que se quiere enviar la actividad, introducir el título e introducir las opciones deseadas. Si la actividad es evaluable se debe seleccionar la respuesta correcta. Al pulsar en la acción Crear se enviará la actividad a los grupos seleccionados\tabularnewline \hline
        \end{tabularx}
        \captionof{table}{Requisito funcional RFD-012}
        \label{tbl:table302}
    \end{figure}
    
	\begin{figure}[H]
        \begin{tabularx}{\linewidth}{|c|X|} \hline
            \multicolumn{2}{|c|}{\textbf{Identificador: RFD-013}} \tabularnewline \hline
            \textbf{Nombre} &  Comprobación de actividad de tipo multirrespuesta\tabularnewline \hline
            \textbf{Prioridad} & Media \tabularnewline \hline
            \textbf{Descripción} & La persona docente comprobará las opciones elegidas por las personas estudiantes de la actividad \tabularnewline \hline
            \textbf{Verificación} & En la actividad aparecerán las opciones elegidas de las personas estudiantes y al lado de las respuestas aparecerá el porcentaje de integrantes del grupo que ha contestado dicha opción. En caso de ser una actividad evaluable, se mostrará la respuesta correcta\tabularnewline \hline
        \end{tabularx}
        \captionof{table}{Requisito funcional RFD-013}
        \label{tbl:table302}
    \end{figure}    
    
    \begin{figure}[H]
        \begin{tabularx}{\linewidth}{|c|X|} \hline
            \multicolumn{2}{|c|}{\textbf{Identificador: RFD-014}} \tabularnewline \hline
            \textbf{Nombre} &  Comprobación de actividad de tipo multirrespuesta\tabularnewline \hline
            \textbf{Prioridad} & Media \tabularnewline \hline
            \textbf{Descripción} & La persona docente comprobará las opciones elegidas por las personas estudiantes de la actividad \tabularnewline \hline
            \textbf{Verificación} & En la actividad aparecerán las opciones elegidas de las personas estudiantes y al lado de las respuestas aparecerá el porcentaje de integrantes del grupo que ha contestado dicha opción. En caso de ser una actividad evaluable, se mostrará la respuesta correcta\tabularnewline \hline
        \end{tabularx}
        \captionof{table}{Requisito funcional RFD-014}
        \label{tbl:table302}
    \end{figure}    
    
    \begin{figure}[H]
        \begin{tabularx}{\linewidth}{|c|X|} \hline
            \multicolumn{2}{|c|}{\textbf{Identificador: RFD-015}} \tabularnewline \hline
            \textbf{Nombre} & Visualización de las estadísticas de un grupo\tabularnewline \hline
            \textbf{Prioridad} & Alta \tabularnewline \hline
            \textbf{Descripción} & La persona docente puede comprobar las estadísticas de las actividades evaluables de un grupo \tabularnewline \hline
            \textbf{Verificación} & La persona docente pulsará la acción correspondiente para visualizar las estadísticas de un grupo en la carpeta de las actividades de uno de ellos. Las estadísticas deben mostrar las calificaciones de las actividades evaluables, tanto grupales como individuales\tabularnewline \hline
        \end{tabularx}
        \captionof{table}{Requisito funcional RFD-015}
        \label{tbl:table302}
    \end{figure}    

    \begin{figure}[H]
        \begin{tabularx}{\linewidth}{|c|X|} \hline
            \multicolumn{2}{|c|}{\textbf{Identificador: RFD-016}} \tabularnewline \hline
            \textbf{Nombre} & Creación de evento\tabularnewline \hline
            \textbf{Prioridad} & Baja \tabularnewline \hline
            \textbf{Descripción} & La persona docente puede crear eventos de actividades para grupos \tabularnewline \hline
            \textbf{Verificación} & Se pulsará la tercera opción del menú desplegable de la pantalla Interactividad. Se deberá seleccionar los grupos a los que se quiere enviar el evento. Se debe introducir el título del evento, la descripción del evento, dónde será el evento y la fecha del evento. Al pulsar en la acción Crear se enviará el evento a los grupos seleccionados\tabularnewline \hline
        \end{tabularx}
        \captionof{table}{Requisito funcional RFD-016}
        \label{tbl:table302}
    \end{figure}  

    \begin{figure}[H]
        \begin{tabularx}{\linewidth}{|c|X|} \hline
            \multicolumn{2}{|c|}{\textbf{Identificador: RFD-017}} \tabularnewline \hline
            \textbf{Nombre} & Visualización de los archivos de grupos \tabularnewline \hline
            \textbf{Prioridad} & Baja \tabularnewline \hline
            \textbf{Descripción} & La persona docente debe ver los archivos enviados por las salas de chat en la que se encuentra en los grupos y de las salas de chat privadas con las personas estudiantes\tabularnewline \hline
            \textbf{Verificación} & La persona docente debe tener dos secciones en las que puede ver los archivos enviados por ambos tipos de grupos\tabularnewline \hline
        \end{tabularx}
        \captionof{table}{Requisito funcional RFD-017}
        \label{tbl:table302}
    \end{figure}    
    
    \subsubsection{De estudiante}
    
    \begin{figure}[H]
        \begin{tabularx}{\linewidth}{|c|X|} \hline
            \multicolumn{2}{|c|}{\textbf{Identificador: RFE-001}} \tabularnewline \hline
            \textbf{Nombre} & Visualización de las personas estudiantes de la asignatura\tabularnewline \hline
            \textbf{Prioridad} & Baja \tabularnewline \hline
            \textbf{Descripción} & La persona estudiante puede comprobar información básica de la persona docente y de las personas estudiantes. La persona estudiante no debe tener acceso a las estadísticas de otras personas estudiantes\tabularnewline \hline
            \textbf{Verificación} & La persona estudiante podrá comprobar información básica de las personas participantes de la clase en su pantalla Inicio\tabularnewline \hline
        \end{tabularx}
        \captionof{table}{Requisito funcional RFE-001}
        \label{tbl:table302}
    \end{figure}    
    
    \begin{figure}[H]
        \begin{tabularx}{\linewidth}{|c|X|} \hline
            \multicolumn{2}{|c|}{\textbf{Identificador: RFE-002}} \tabularnewline \hline
            \textbf{Nombre} & Control parcial sobre las peticiones\tabularnewline \hline
            \textbf{Prioridad} & Media \tabularnewline \hline
            \textbf{Descripción} & La persona estudiante tiene control total sobre sus peticiones y ninguna sobre peticiones de otras personas estudiantes\tabularnewline \hline
            \textbf{Verificación} & La persona estudiante puede eliminar sus peticiones de creación de grupo y aceptar o rechazar las peticiones de otras personas estudiantes, pero las acciones sobre estas últimas deben suponer únicamente un cambio estético en las mismas\tabularnewline \hline
        \end{tabularx}
        \captionof{table}{Requisito funcional RFE-002}
        \label{tbl:table302}
    \end{figure}    
    
    \begin{figure}[H]
        \begin{tabularx}{\linewidth}{|c|X|} \hline
            \multicolumn{2}{|c|}{\textbf{Identificador: RFE-003}} \tabularnewline \hline
            \textbf{Nombre} & Control sobre los eventos\tabularnewline \hline
            \textbf{Prioridad} & Baja \tabularnewline \hline
            \textbf{Descripción} & La persona estudiante debe tener control sobre los eventos que han sido creados por ella \tabularnewline \hline
            \textbf{Verificación} & La persona estudiante podrá comprobar los eventos que han sido enviados por la persona docente y por las personas portavoces de los grupos en los que se encuentra. Las personas estudiantes que sean portavoces pueden eliminar los eventos que hayan creado para sus grupos\tabularnewline \hline
        \end{tabularx}
        \captionof{table}{Requisito funcional RFE-003}
        \label{tbl:table302}
    \end{figure}    
    
    \begin{figure}[H]
        \begin{tabularx}{\linewidth}{|c|X|} \hline
            \multicolumn{2}{|c|}{\textbf{Identificador: RFE-004}} \tabularnewline \hline
            \textbf{Nombre} & Creación de petición de grupo\tabularnewline \hline
            \textbf{Prioridad} & Alta \tabularnewline \hline
            \textbf{Descripción} & La persona estudiante puede crear una petición de creación de grupo \tabularnewline \hline
            \textbf{Verificación} & Se pulsará en la acción de creación de petición en la pantalla Grupos. Se desplegará una lista con las personas estudiantes de la asignatura, sin incluir a la propia persona estudiante que va a crear la petición. La persona estudiante seleccionará las personas estudiantes con las que le gustaría estar en el grupo. Al pulsar en la acción Crear se enviará la petición a todos las personas participantes incluidas\tabularnewline \hline
        \end{tabularx}
        \captionof{table}{Requisito funcional RFE-004}
        \label{tbl:table302}
    \end{figure}    
    
    \begin{figure}[H]
        \begin{tabularx}{\linewidth}{|c|X|} \hline
            \multicolumn{2}{|c|}{\textbf{Identificador: RFE-005}} \tabularnewline \hline
            \textbf{Nombre} & Creación de grupo privado con la persona docente\tabularnewline \hline
            \textbf{Prioridad} & Alta\tabularnewline \hline
            \textbf{Descripción} & La persona estudiante debe poder crear directamente el grupo privado de la persona docente \tabularnewline \hline
            \textbf{Verificación} & Se pulsará en la acción de chat privado con la persona docente. En caso de que el grupo privado ya esté creado se redirigirá a la persona estudiante a la sala de chat. En caso contrario, la persona estudiante creará la sala indirectamente al pulsar sobre la acción. Si es la persona estudiante la que crea el grupo privado, esta no le aparecerá a la persona docente hasta que la persona estudiante no envíe al menos un mensaje\tabularnewline \hline
        \end{tabularx}
        \captionof{table}{Requisito funcional RFE-005}
        \label{tbl:table302}
    \end{figure}    
    
    \begin{figure}[H]
        \begin{tabularx}{\linewidth}{|c|X|} \hline
            \multicolumn{2}{|c|}{\textbf{Identificador: RFE-006}} \tabularnewline \hline
            \textbf{Nombre} & Creación de eventos\tabularnewline \hline
            \textbf{Prioridad} & Baja \tabularnewline \hline
            \textbf{Descripción} & La persona estudiante podrá crear eventos sí y solo sí es portavoz \tabularnewline \hline
            \textbf{Verificación} & Se pulsará en la opción de creación de eventos en la pantalla Grupos. La persona estudiante creará el evento de la misma forma que es creado por la persona docente, con la excepción de que la persona estudiante solo podrá enviarlo a los grupos de los que es portavoz\tabularnewline \hline
        \end{tabularx}
        \captionof{table}{Requisito funcional RFE-006}
        \label{tbl:table302}
    \end{figure}    
    
    \begin{figure}[H]
        \begin{tabularx}{\linewidth}{|c|X|} \hline
            \multicolumn{2}{|c|}{\textbf{Identificador: RFE-007}} \tabularnewline \hline
            \textbf{Nombre} & Muestra de ambas salas de chat en los grupos\tabularnewline \hline
            \textbf{Prioridad} & Alta \tabularnewline \hline
            \textbf{Descripción} & La persona estudiante podrá acceder a ambas salas de chat del grupo, tanto con la persona docente como sin ella \tabularnewline \hline
            \textbf{Verificación} & Ambas salas de chat se mostrarán en los grupos en los que se encuentra la persona estudiante. Podrá acceder a ellas pulsando en la correspondiente acción\tabularnewline \hline
        \end{tabularx}
        \captionof{table}{Requisito funcional RFE-007}
        \label{tbl:table302}
    \end{figure}    
    
    \begin{figure}[H]
        \begin{tabularx}{\linewidth}{|c|X|} \hline
            \multicolumn{2}{|c|}{\textbf{Identificador: RFE-008}} \tabularnewline \hline
            \textbf{Nombre} & Respuesta a las actividades de tipo entrada de texto\tabularnewline \hline
            \textbf{Prioridad} & Alta \tabularnewline \hline
            \textbf{Descripción} & La persona estudiante podrá responder las actividades de tipo entrada de texto \tabularnewline \hline
            \textbf{Verificación} & Al recibir una actividad de tipo entrada de texto, si la actividad es de tipo no grupal podrá ser contestada por todas las personas estudiantes del grupo. En caso de que sea grupal solo podrá ser contestada por la persona portavoz del grupo\tabularnewline \hline
        \end{tabularx}
        \captionof{table}{Requisito funcional RFE-008}
        \label{tbl:table302}
    \end{figure}    
    
    \begin{figure}[H]
        \begin{tabularx}{\linewidth}{|c|X|} \hline
            \multicolumn{2}{|c|}{\textbf{Identificador: RFE-009}} \tabularnewline \hline
            \textbf{Nombre} & Respuesta a las actividades de tipo multirrespuesta\tabularnewline \hline
            \textbf{Prioridad} & Alta \tabularnewline \hline
            \textbf{Descripción} & La persona estudiante podrá responder las actividades de tipo multirrespuesta  \tabularnewline \hline
            \textbf{Verificación} & Al recibir una actividad de tipo multirrespuesta, si la actividad es de tipo no grupal podrá ser contestada por todas las personas estudiantes del grupo. En caso de que sea grupal solo podrá ser contestada por la persona portavoz del grupo\tabularnewline \hline
        \end{tabularx}
        \captionof{table}{Requisito funcional RFE-009}
        \label{tbl:table302}
    \end{figure}    
    
    \subsubsection{Comunes}
    
    	\begin{figure}[H]
        \begin{tabularx}{\linewidth}{|c|X|} \hline
            \multicolumn{2}{|c|}{\textbf{Identificador: RFC-001}} \tabularnewline \hline
            \textbf{Nombre} & Iniciar sesión \tabularnewline \hline
            \textbf{Prioridad} & Alta \tabularnewline \hline
            \textbf{Descripción} & Inicio de sesión de la persona introduciendo sus datos personales (e-mail y contraseña) para usar la aplicación \tabularnewline \hline
            \textbf{Verificación} & La persona debe introducir sus datos en la pantalla de inicio. En caso de éxito se le llevará a su pantalla de inicio correspondiente. En caso de error se mostrará un mensaje pidiendo que se revisen los datos\tabularnewline \hline
        \end{tabularx}
        \captionof{table}{Requisito funcional RFC-001}
        \label{tbl:table302}
    \end{figure}
    
	\begin{figure}[H]
        \begin{tabularx}{\linewidth}{|c|X|} \hline
            \multicolumn{2}{|c|}{\textbf{Identificador: RFC-002}} \tabularnewline \hline
            \textbf{Nombre} & Selección de curso y asignatura \tabularnewline \hline
            \textbf{Prioridad} & Alta \tabularnewline \hline
            \textbf{Descripción} & Selección de asignatura del curso en las que la persona está registrada\tabularnewline \hline
            \textbf{Verificación} & La persona selecciona el curso y la respectiva asignatura de dicho curso sobre la que quiere mostrar información en un menú que aparecerá en la parte superior de la pantalla a la derecha. Se debe mostrar un desplegable sobre los cursos y asignaturas en los que está registrada. Una vez haya seleccionado una asignatura se le redirigirá a la pantalla de inicio. El diálogo mostrado no debe desvanecerse\tabularnewline \hline
        \end{tabularx}
        \captionof{table}{Requisito funcional RFC-002}
        \label{tbl:table302}
    \end{figure}
    
    \begin{figure}[H]
        \begin{tabularx}{\linewidth}{|c|X|} \hline
            \multicolumn{2}{|c|}{\textbf{Identificador: RFC-003}} \tabularnewline \hline
            \textbf{Nombre} & Envío de mensajes de texto por la sala chat\tabularnewline \hline
            \textbf{Prioridad} & Alta\tabularnewline \hline
            \textbf{Descripción} & Las personas pueden enviar mensajes de texto por las salas de chat \tabularnewline \hline
            \textbf{Verificación} & Las personas deben introducir el texto que quieran enviar en la barra de entrada de texto que aparece en la parte inferior de la pantalla y pulsar la acción diseñada para enviar el mensaje\tabularnewline \hline
        \end{tabularx}
        \captionof{table}{Requisito funcional RFC-003}
        \label{tbl:table302}
    \end{figure}    
    
    \begin{figure}[H]
        \begin{tabularx}{\linewidth}{|c|X|} \hline
            \multicolumn{2}{|c|}{\textbf{Identificador: RFC-004}} \tabularnewline \hline
            \textbf{Nombre} & Envío de enlaces por la sala de chat\tabularnewline \hline
            \textbf{Prioridad} & Media \tabularnewline \hline
            \textbf{Descripción} & Las personas pueden enviar enlaces a recursos externos por la sala de chat \tabularnewline \hline
            \textbf{Verificación} & La persona puede enviar un enlace a un recurso externo de la misma forma que puede enviar mensajes de texto. El enlace será cliqueable, y redirigirá a la persona al recurso al que apunta\tabularnewline \hline
        \end{tabularx}
        \captionof{table}{Requisito funcional RFC-004}
        \label{tbl:table302}
    \end{figure}    
    

    \begin{figure}[H]
        \begin{tabularx}{\linewidth}{|c|X|} \hline
            \multicolumn{2}{|c|}{\textbf{Identificador: RFC-005}} \tabularnewline \hline
            \textbf{Nombre} & Envío de archivos por la sala de chat\tabularnewline \hline
            \textbf{Prioridad} & Alta \tabularnewline \hline
            \textbf{Descripción} & La persona puede enviar archivos por la sala de chat \tabularnewline \hline
            \textbf{Verificación} & La persona pulsará la acción diseñada para adjuntar archivos. Podrá subir tres formatos de archivos diferentes: PDF, JPEG (JPG) y PNG. Los archivos aparecerán en la sala de chat con un formato diferente a los archivos de tipo texto \tabularnewline \hline
        \end{tabularx}
        \captionof{table}{Requisito funcional RFC-005}
        \label{tbl:table302}
    \end{figure}    
	
	\begin{figure}[H]
        \begin{tabularx}{\linewidth}{|c|X|} \hline
            \multicolumn{2}{|c|}{\textbf{Identificador: RFC-006}} \tabularnewline \hline
            \textbf{Nombre} & Descarga de archivos enviados por las salas de chat\tabularnewline \hline
            \textbf{Prioridad} & Alta \tabularnewline \hline
            \textbf{Descripción} & La persona puede descargar los archivos enviados por las salas de chat \tabularnewline \hline
            \textbf{Verificación} & La persona podrá descargar los archivos tanto desde la sala de chat como desde la pantalla de archivos \tabularnewline \hline
        \end{tabularx}
        \captionof{table}{Requisito funcional RFC-006}
        \label{tbl:table302}
    \end{figure}    
	
	\begin{figure}[H]
        \begin{tabularx}{\linewidth}{|c|X|} \hline
            \multicolumn{2}{|c|}{\textbf{Identificador: RFC-007}} \tabularnewline \hline
            \textbf{Nombre} & Comprobación de la pertenencia o no a un grupo\tabularnewline \hline
            \textbf{Prioridad} & Media \tabularnewline \hline
            \textbf{Descripción} & La persona podrá comprobar la pertenencia o no a un grupo de un estudiante \tabularnewline \hline
            \textbf{Verificación} & La persona introducirá el nombre de una persona estudiante completo en la barra de búsqueda en la pantalla Grupos. Una vez hecho esto, se mostrarán los grupos en los que se encuentra dicha persona estudiante \tabularnewline \hline
        \end{tabularx}
        \captionof{table}{Requisito funcional RFC-007}
        \label{tbl:table302}
    \end{figure}    
	
	\begin{figure}[H]
        \begin{tabularx}{\linewidth}{|c|X|} \hline
            \multicolumn{2}{|c|}{\textbf{Identificador: RFC-008}} \tabularnewline \hline
            \textbf{Nombre} & Mostrar participantes de un grupo\tabularnewline \hline
            \textbf{Prioridad} & Baja \tabularnewline \hline
            \textbf{Descripción} & La persona puede mostrar las personas participantes de un grupo \tabularnewline \hline
            \textbf{Verificación} & La persona pulsará la acción diseñada para ello. Se mostrarán las personas participantes del grupo \tabularnewline \hline
        \end{tabularx}
        \captionof{table}{Requisito funcional RFC-008}
        \label{tbl:table302}
    \end{figure}    
	
	\subsection{Requisitos no funcionales}
	
	\begin{figure}[H]
        \begin{tabularx}{\linewidth}{|c|X|} \hline
            \multicolumn{2}{|c|}{\textbf{Identificador: RNF-001}} \tabularnewline \hline
            \textbf{Nombre} & Coherencia en la sincronía\tabularnewline \hline
            \textbf{Prioridad} & Alta \tabularnewline \hline
            \textbf{Descripción} & Las personas pueden enviar datos a la vez \tabularnewline \hline
            \textbf{Verificación} & Las personas no deben encontrar comportamientos anómalos al recibir o enviar mensajes, interactividades, archivos, peticiones, eventos o cualquier tipo de datos simultáneamente\tabularnewline \hline
        \end{tabularx}
        \captionof{table}{Requisito no funcional RFN-001}
        \label{tbl:table302}
    \end{figure}   
    
    \begin{figure}[H]
        \begin{tabularx}{\linewidth}{|c|X|} \hline
            \multicolumn{2}{|c|}{\textbf{Identificador: RNF-002}} \tabularnewline \hline
            \textbf{Nombre} & Neutralidad de género \tabularnewline \hline
            \textbf{Prioridad} & Alta \tabularnewline \hline
            \textbf{Descripción} & Se debe hacer referencia a las personas de la aplicación con términos neutros \tabularnewline \hline
            \textbf{Verificación} & Se hará referencia a las personas con términos como \enquote{docente} o \enquote{estudiante}, evitando el uso de artículos determinados\tabularnewline \hline
        \end{tabularx}
        \captionof{table}{Requisito no funcional RFN-002}
        \label{tbl:table302}
    \end{figure}   

    \begin{figure}[H]
        \begin{tabularx}{\linewidth}{|c|X|} \hline
            \multicolumn{2}{|c|}{\textbf{Identificador: RNF-003}} \tabularnewline \hline
            \textbf{Nombre} & Simplicidad \tabularnewline \hline
            \textbf{Prioridad} & Media \tabularnewline \hline
            \textbf{Descripción} & Se mostrará todas la información de una pantalla en la propia pantalla\tabularnewline \hline
            \textbf{Verificación} & La persona no debe viajar a otra pantalla para encontrar información referente a dicha pantalla\tabularnewline \hline
        \end{tabularx}
        \captionof{table}{Requisito no funcional RFN-003}
        \label{tbl:table302}
    \end{figure}   

    \begin{figure}[H]
        \begin{tabularx}{\linewidth}{|c|X|} \hline
            \multicolumn{2}{|c|}{\textbf{Identificador: RNF-004}} \tabularnewline \hline
            \textbf{Nombre} & Compatibilidad \tabularnewline \hline
            \textbf{Prioridad} & Media \tabularnewline \hline
            \textbf{Descripción} & La aplicación debe ser soportada por la mayoría de dispositivos Android\tabularnewline \hline
            \textbf{Verificación} & Se requerirá de una versión de Android con una antigüedad no superior a seis años para poder utilizar la aplicación \tabularnewline \hline
        \end{tabularx}
        \captionof{table}{Requisito no funcional RFN-004}
        \label{tbl:table302}
    \end{figure}   

    \begin{figure}[H]
        \begin{tabularx}{\linewidth}{|c|X|} \hline
            \multicolumn{2}{|c|}{\textbf{Identificador: RNF-005}} \tabularnewline \hline
            \textbf{Nombre} & Simplicidad en el diseño de la base de datos \tabularnewline \hline
            \textbf{Prioridad} & Alta \tabularnewline \hline
            \textbf{Descripción} & La estructura de la base de datos debe ser sencilla para realizar el mínimo número de consultas posible\tabularnewline \hline
            \textbf{Verificación} & Las operaciones de lectura y escritura en la base de datos deben ser reducidas, verificando este requisito en las estadísticas del uso de dicha base de datos ofrecidas por Firebase \tabularnewline \hline
        \end{tabularx}
        \captionof{table}{Requisito no funcional RFN-005}
        \label{tbl:rnf005}
    \end{figure}   
	
	\section{Diseño de la estructura de la base de datos}
	Con el objetivo de implementar lo estipulado en el RNF-005, y debido a que la estructura de la base de datos es de crucial importancia para un funcionamiento correcto de la aplicación, se ha realizado el diseño que debe tener la base de datos para cumplir con dichos objetivos. Firestore tiene dos estructuras de datos principales para su base de datos: los documentos y las colecciones, siendo estas últimas una colección de documentos. Los documentos son las estructuras que contienen los verdaderos datos de la aplicación. Estos guardan todo tipo de estructuras de datos que se pueden pasar a objetos tipo Java usando la función \texttt{documentToObject()} que ofrece Firebase. Cuentan con un nombre de documento, que en el caso de los usuarios registrados de la aplicación es su identificador alfanumérico único. En la figura se muestran los documentos en azul y las colecciones en rojo:
	\begin{figure}[H]
	\centering
	\captionsetup{justification=centering}
	\begin{tikzpicture}[
		every node/.style = {rectangle, 
			font=\bfseries,
			scale=0.5,
			draw=black,
			text width=4cm,
			align=center,
			inner sep = 1.5mm}
		]
		\node[fill=red, text=white](p1) {Courses\\Organization};
		\node[fill=blue, text=white](p2)[below=of p1] {CourseDoc};
		\node[fill=red, text=white](p3)[below=of p2] {Subjects};
		\node[fill=blue, text=white](p4)[below=of p3] {SubjectDoc};
		\node(p5)[fill=red, text=white][below=of p4] {Collective\\Groups};
		\node(p6)[fill=red, text=white][below right=of p4] {Individual\\Groups};
		\node(p7)[fill=blue, text=white][below=of p5] {GroupDoc};
		\node(p8)[fill=red, text=white][below=of p7] {ChatRoom\\WithTeacher};
		\node(p9)[fill=red, text=white][right=of p8] {ChatRoom\\WithoutTeacher};
		\node(p10)[fill=red, text=white][right=of p9] {Interactivity\\Cards};
		\node(p11)[fill=red, text=white][left=of p8] {Storage\\WithTeacher};
		\node(p12)[fill=red, text=white][left=of p11] {Storage\\WithoutTeacher};
		\node(p13)[fill=red, text=white][left=of p5] {Petitions};
		\node(p14)[fill=blue, text=white][below=of p12] {StorageDoc};
		\node(p15)[fill=blue, text=white][below=of p11] {StorageDoc};
		\node(p16)[fill=blue, text=white][below=of p8] {MessageDoc};
		\node(p17)[fill=blue, text=white][below=of p9] {MessageDoc};
		\node(p18)[fill=blue, text=white][below=of p10] {InteractivityDoc};
		\node(p19)[fill=blue, text=white][below= 0.5cm of p13] {PetitionDoc};
		
		\draw
		
		(p1) -- (p2)
		(p2) -- (p3)
		(p3) -- (p4)
		(p4) -- (p5)
		(p4) -| (p13)
		(p4) -| (p6)
		(p5) -- (p7)
		(p7) -- (p8)
		(p7) -| (p9)
		(p7) -| (p10)
		(p7) -| (p11)
		(p7) -| (p12)
		(p12) -- (p14)
		(p11) -- (p15)
		(p8) -- (p16)
		(p9) -- (p17)
		(p10) -- (p18)
		(p13) -- (p19)
		;
		
		\draw [dots, draw=black, line width=1pt] (p6.south) -- ++(0,-0.6cm);
	\end{tikzpicture}
	\caption{Estructura de la colección principal de documentos}
	\label{fig:dbStructure}
    \end{figure}
	
	La colección principal se llama \textit{CoursesOrganization}. Esta colección contiene todas las subcolecciones más importantes de la base de datos. En esta colección se guarda toda la información referente a la asignatura salvo los datos de las personas usuarias (actividades, mensajes, etc). Solo hay una excepción, que son los archivos que se suben en las salas de chat, ya que se guardan en una base de datos independiente que no forma parte de Firestore, aunque en Firestore se guardan las referencias a dichos archivos. Los nombres de las colecciones son descriptivos en cuanto a la información que contienen. Por ejemplo, en la colección \textit{ChatRoomWithTeacher} se guardan los mensajes de la sala de chat en la que está la persona docente con las personas estudiantes de un grupo. La colección \textit{IndividualGroups}, que es donde se encuentran los documentos de grupo (\textit{GroupDoc}) privados con la persona docente, solo cuenta con dos subcolecciones: \textit{ChatRoomWithTeacher} y \textit{StorageWithTeacher}, ya que a un grupo privado con una única persona estudiante no se pueden enviar interactividades, ni tiene una sala de chat privada con otras personas estudiantes.\par
	
	Las colecciones que guardan la información de las personas usuarias de la aplicación tienen la siguiente forma. Estas colecciones tienen la estructura más básica, y permiten buscar la información de una persona usuaria con una sola lectura de la base de datos con el objetivo de buscar la información de una persona usuaria de forma rápida y directa:
	
	\begin{figure}[H]
	\centering
	\captionsetup{justification=centering}
	\begin{tikzpicture}[
		every node/.style = {rectangle, 
			font=\bfseries,
			scale=0.5,
			draw=black,
			text width=4cm,
			align=center,
			inner sep = 1.5mm}
		]
		\node[fill=red, text=white](p1) {Teachers};
		\node[fill=blue, text=white](p2)[below=of p1] {TeacherDoc};
		\node[fill=red, text=white](p3)[right=of p1] {Students};
		\node[fill=blue, text=white](p4)[below=of p3] {StudentsDoc};
		
		\draw
		
		(p1) -- (p2)
		(p3) -- (p4)
		;
		
	\end{tikzpicture}
		\caption{Estructura de la colección con la información de los usuarios}
    \end{figure}
	
	\section{Diseño de la interfaz gráfica}
	
	Definidos los requisitos y la estructura de la base de datos se puede comenzar con el desarrollo de la aplicación, empezando por el diseño de su interfaz gráfica.\par
	Cualquier aplicación desarrollada para Android necesita un logotipo. Los logotipos tienen que seguir unas pautas de diseño, como ser simples y que sean fáciles de recordar e identificar. Con esto en cuenta, se diseñó un logotipo para CoordinApp:
	
	\begin{figure}[H]
		\captionsetup{justification=centering}
		\begin{subfigure}{.5\textwidth}
			\centering
			\includegraphics[width=.5\linewidth]{imagenes/cap3/3001.png}
			\caption{Logotipo con tipografía}
		\end{subfigure}%
		\begin{subfigure}{.5\textwidth}
			\centering
			\includegraphics[width=.5\linewidth]{imagenes/cap3/3002.png}
			\caption{Logotipo sin tipografía}
		\end{subfigure}%
		\caption{Logotipo de CoordinApp}
		\label{fig:301}
	\end{figure}
	
	Por otra parte, al crearlo se pensó en representar la unión (propia de los EVAs o de cualquier plataforma de comunicación entre individuos) junto con un símbolo que transmitiese sentimientos positivos, de ahí el puzle. La paleta de colores está compuesta de tonalidades azules, colores que transmiten calma, además de blanco y negro para contraste. La aplicación ha sido desarrollada siempre que se ha podido con componentes gráficos de la librería \textit{Material Design} de Google, una librería moderna y ampliamente utilizada en el diseño de aplicaciones Android, recordando estos componentes al papel y que les dota de una sensación de profundidad.\par
	
	El proyecto está diseñado para que existan tres tipos de roles:
	
	\begin{itemize}
		\item \textbf{Administrador}: Se encarga de gestionar la base de datos y de registrar a las personas usuarias en ella para que las personas usuarias puedan iniciar sesión en la aplicación. También se encarga de organizar los cursos, las asignaturas de cada curso y de asignar a dichas asignaturas las personas docentes y estudiantes que las cursen. Para dicha tarea no utiliza la aplicación móvil, sino una API de REST. Tiene el control de todos los datos de la aplicación.
		
		\item \textbf{Docente}: Se encarga de impartir la asignatura asignada por el administrador. Tiene el control total de la gestión de los grupos de estudiantes de las asignaturas que la persona administradora le haya asignado y de todas las funcionalidades de la app.
		
		\item \textbf{Estudiante}: Tiene cierto control sobre la creación de grupos, siempre con la aprobación de la persona docente. Dentro del rol de estudiante hay dos categorías: portavoz y no portavoz.
	\end{itemize}
	
	Para los dos perfiles que utilizan la app (docente y estudiante) la pantalla de inicio es la misma:
	
	\begin{figure}[H]
		\captionsetup{justification=centering}
		\centering
		\includegraphics[width=.25\linewidth]{imagenes/cap3/3003.jpg}
		\caption{Pantalla de inicio}
		\label{fig:302}
	\end{figure}
	
	Se puede comprobar que la pantalla de inicio no cuenta con un sistema de registro. Esto se debe a que toda la información necesaria para el acceso de las personas usuarias en la app ya ha sido gestionada por la persona administradora con el objetivo de que una persona usuaria ajena a la institución o centro educativo que no cuente con una cuenta validada por la persona administradora no pueda tener acceso a la app.\par
	
	Las personas docentes tienen un e-mail con el formato docenteX@coordinapp.es y las personas estudiantes un e-mail con el formato estudianteX@coordinapp.es, siendo X el identificativo de la persona usuaria, pudiendo tener otro formato diferente (por ejemplo, el nombre propio de la persona usuaria correspondiente terminado en @coordinapp.es o un número de identificación de la institución). Durante el resto del capítulo se usará a Docente1 para mostrar el diseño de las funcionalidades de la persona docente y a EstudianteX para mostrar el diseño de las funcionalidades de las personas estudiantes.\par
	
	En el momento de iniciar sesión ya se sea persona docente u estudiante, ambas serán redirigidas a la siguiente pantalla. En dicha pantalla se tendrá que seleccionar en el menú de la parte superior derecha la asignatura del curso sobre la que se quiere mostrar información. Nótese que a la persona docente no le aparece la misma información que a la persona estudiante. Esto es porque la persona estudiante se encuentra inscrita en dos asignaturas mientras que la persona docente solo está inscrita en una de ellas (la que imparte, en este caso biología de 2º de bachillerato), siendo otra persona docente la encargada de impartir la otra asignatura (en este caso, matemáticas).
	
	\begin{figure}[H]
		\captionsetup{justification=centering}
		\begin{subfigure}{.5\textwidth}
			\centering
			\includegraphics[width=.5\linewidth]{imagenes/cap3/3004.jpg}
			\caption{Información de Docente1}
		\end{subfigure}%
		\begin{subfigure}{.5\textwidth}
			\centering
			\includegraphics[width=.5\linewidth]{imagenes/cap3/3005.jpg}
			\caption{Información de Estudiante1}
		\end{subfigure}%
		\caption{Cursos y asignaturas de los diferentes usuarios}
		\label{fig:303}
	\end{figure}
	
	Una vez seleccionada la asignatura ambas personas serán redirigidos a la pantalla de inicio. Tanto docente como estudiante disponen de cuatro secciones (o pantallas) por las que navegar, aunque ambas personas no realizan las mismas acciones en ellas. 
	
	\begin{figure}[H]
		\captionsetup{justification=centering}
		\centering
		\begin{subfigure}{\linewidth}
			\centering
			\includegraphics[width=.5\linewidth]{imagenes/cap3/3006.jpg}
			\caption{Sección Interactividad}
		\end{subfigure} %
		
		\hfill
		
		\begin{subfigure}{\linewidth}
			\centering
			\includegraphics[width=.5\linewidth]{imagenes/cap3/3007.jpg}
			\caption{Sección Grupos}
		\end{subfigure}
		
		\hfill
		
		
		\begin{subfigure}{\linewidth}
			\centering
			\includegraphics[width=.5\linewidth]{imagenes/cap3/3008.jpg}
			\caption{Sección Archivos}
		\end{subfigure}
		
		\hfill
		
		\begin{subfigure}{\linewidth}
			\centering
			\includegraphics[width=.5\linewidth]{imagenes/cap3/3009.jpg}
			\caption{Sección Inicio}
		\end{subfigure} 
		
		\caption{Secciones de los usuarios de la aplicación}
		\label{fig:304}
	\end{figure}
	
	\subsection{Rol de docente}
	
	\subsubsection{Pantalla Inicio}
	
	La persona docente es redirigida a la pantalla de inicio. Esta pantalla cuenta con tres secciones, las cuales se pueden seleccionar en la barra deslizante superior. La primera sección es la sección de estadísticas. Lo primero que se muestra es una lista con las personas estudiantes de la asignatura ordenada por orden alfabético. Al seleccionar una persona estudiante se mostrarán estadísticas individuales de la misma, las cuales solo es posible obtener si la persona estudiante está al menos en un grupo. Como no es el caso del ejemplo ya que todavía no se ha creado ninguno, al seleccionar a la persona estudiante se muestra un mensaje de aviso. La persona docente puede buscar a una persona estudiante concreto introduciendo su nombre en la barra de búsqueda de la parte superior.
	
	\begin{figure}[H]
		\captionsetup{justification=centering}
		\begin{subfigure}{.5\textwidth}
			\centering
			\includegraphics[width=.5\linewidth]{imagenes/cap3/3010.jpg}
			\caption{Lista de estudiantes de la asignatura}
		\end{subfigure}%
		\begin{subfigure}{.5\textwidth}
			\centering
			\includegraphics[width=.5\linewidth]{imagenes/cap3/3011.jpg}
			\caption{Búsqueda de un estudiante concreto}
		\end{subfigure}%
		\caption{Sección de estadísticas}
		\label{fig:305}
	\end{figure}
	
	La siguiente sección es la de peticiones. Las personas estudiantes pueden crear peticiones de creación de grupos con las personas integrantes que deseen. La persona docente puede comprobar el estatus de las peticiones de las personas estudiantes viendo la lista de las personas participantes incluidas en la petición. Un símbolo amarillo indica que la persona estudiante no ha reaccionado a la petición, un símbolo verde indica que la persona estudiante ha aceptado la petición (la persona estudiante que ha realizado la petición siempre muestra un símbolo verde) y un símbolo rojo indica que la persona estudiante ha rechazado la petición. Este sistema de símbolos tiene mero carácter informativo para la persona docente, con el que puede hacerse una idea de si las personas estudiantes están de acuerdo en la creación de tal grupo o no, sin afectar la acción de las personas estudiantes sobre la petición al control que tiene la persona docente de la creación o no de dicho grupo.\par
	
	Si la persona docente decide crear el grupo sugerido por la persona Estudiante1 se creará con las personas integrantes que componen la petición al pulsar la acción \enquote{Aceptar}. Si se acepta, se creará un grupo compuesto de las personas participantes incluidas en la petición y aparecerá en la pantalla Grupos.
	En caso contrario, puede rechazar la petición pulsando el botón \enquote{Eliminar}, lo cual eliminará la petición para toas las personas participantes incluidas en ella. 
	
	\begin{figure}[H]
		\captionsetup{justification=centering}
		\begin{subfigure}[t]{.33\textwidth}
			\centering
			\includegraphics[width=.5\linewidth]{imagenes/cap3/3012.jpg}
			\caption{Sección de peticiones sin ninguna petición}
		\end{subfigure}%
		\begin{subfigure}[t]{.33\textwidth}
			\centering
			\includegraphics[width=.5\linewidth]{imagenes/cap3/3013.jpg}
			\caption{Sección de peticiones con una petición}
		\end{subfigure}%
		\begin{subfigure}[t]{.33\textwidth}
			\centering
			\includegraphics[width=.5\linewidth]{imagenes/cap3/3014.jpg}
			\caption{Estatus de las peticiones}
		\end{subfigure}%
		\caption{Sección de peticiones}
		\label{fig:306}
	\end{figure}
	
	La última sección de la pantalla inicio es la de eventos. En ella aparecen eventos creados por la persona docente. Un evento está compuesto de un título, una descripción, un lugar y una fecha. Los eventos se envían a grupos específicos de estudiantes. En este caso, como todavía no se ha creado ningún grupo, la persona docente no puede enviar eventos. Se volverá a esta sección cuando haya un grupo creado a modo de demostración.
	
	\begin{figure}[H]
		\captionsetup{justification=centering}
		\centering
		\includegraphics[width=.25\linewidth]{imagenes/cap3/3015.jpg}
		\caption{Sección de eventos}
		\label{fig:307}
	\end{figure}
	
	\subsubsection{Pantalla Grupos}
	
	Esta es la pantalla más importante de la aplicación, ya que si no se realizan acciones en ella la persona docente no podrá utilizar las funcionalidades de la aplicación. Desde aquí la persona docente puede crear y administrar los grupos de trabajo de estudiantes. Un grupo es un área de trabajo donde las personas estudiantes pueden debatir sobre las respuestas a actividades creadas por la persona docente, organizar las actividades de sus eventos o ayudarse y colaborar. Esta pantalla cuenta con dos secciones: grupales e individuales. La sección grupal se ha diseñado para incluir a los grupos de más de dos estudiantes, mientras que la sección individual incluye los grupos formados únicamente por una persona estudiante para asuntos privados de dicha persona estudiante con la persona docente. Los grupos, sean del tipo que sean, siempre incluyen a la persona docente.\par
	
	En la sección grupal, la persona docente puede crear un grupo pulsando sobre el menú en la parte inferior derecha de la pantalla, el cual al desplegarse mostrará tres opciones distintas. La primera opción empezando por la parte inferior abre un menú de creación de un único grupo.
	
	\begin{figure}[H]
		\captionsetup{justification=centering}
		\begin{subfigure}{.5\textwidth}
			\centering
			\includegraphics[width=.5\linewidth]{imagenes/cap3/3016.jpg}
			\caption{Sección grupal con menú desplegado}
		\end{subfigure}%
		\begin{subfigure}{.5\textwidth}
			\centering
			\includegraphics[width=.5\linewidth]{imagenes/cap3/3017.jpg}
			\caption{Primera opción del menú: Crear un único grupo}
		\end{subfigure}%
		\caption{Pantalla Grupos}
		\label{fig:308}
	\end{figure}
	
	En este menú se indica a la persona docente que seleccione a las personas estudiantes que quiere incluir en el grupo, teniendo la opción de seleccionar a todas las personas estudiantes de la asignatura a la vez. Si selecciona a una única persona estudiante y pulsa la acción \textit{Crear} se creará el grupo individual con la persona estudiante seleccionada, apareciendo esta en la sección Individuales de la pantalla Grupos. 
	
	\begin{figure}[H]
		\captionsetup{justification=centering}
		\begin{subfigure}{.5\textwidth}
			\centering
			\includegraphics[width=.5\linewidth]{imagenes/cap3/3018.jpg}
			\caption{Selección de un único estudiante}
		\end{subfigure}%
		\begin{subfigure}{.5\textwidth}
			\centering
			\includegraphics[width=.5\linewidth]{imagenes/cap3/3019.jpg}
			\caption{Formato de un grupo privado con el docente}
		\end{subfigure}%
		\caption{Creación de un grupo privado con un estudiante}
		\label{fig:309}
	\end{figure}
	
	La persona docente puede ocultar un grupo individual pulsando en la acción \textit{Ocultar} debido a que, siendo necesario un grupo de este tipo para cada estudiante de la asignatura, podría resultar poco práctico tener un número elevado de grupos individuales mostrados al mismo tiempo. Al ocultar un grupo individual no se borra la información que contiene, pudiendo volver a ser mostrado realizando los mismos pasos de la forma en la que fue creado. Una persona estudiante puede hablar a la persona docente sin necesidad de que esta cree el grupo individual debido a que las personas estudiantes pueden necesitar hablar con la persona docente sobre un asunto privado en cualquier momento. Por otra parte, se considera que la información que se habla con dicha persona estudiante, al no haber más testigos en la conversación que ambas, puede que ser requerida por la dirección del centro educativo en caso de surgir algún tipo de problema, por lo que los grupos individuales no pueden ser borrados. \par

	Si se decide seleccionar más de una persona estudiante aparecerá un botón a la izquierda de la casilla de selección de las personas estudiantes seleccionados. Este botón es de obligada selección en este caso, e indica quién será la persona portavoz del grupo que se vaya a crear. Si se ha seleccionado una persona portavoz pero se cambia de opinión y se selecciona otra persona estudiante como nueva persona portavoz se desmarcará la persona estudiante que había sido seleccionada portavoz inicialmente. La aplicación no deja crear el grupo si no se ha seleccionado una persona portavoz.
	
	\begin{figure}[H]
		\captionsetup{justification=centering}
		\begin{subfigure}{.5\textwidth}
			\centering
			\includegraphics[width=.5\linewidth]{imagenes/cap3/3020.jpg}
			\caption{Aviso de necesidad de selección de portavoz}
		\end{subfigure}%
		\begin{subfigure}{.5\textwidth}
			\centering
			\includegraphics[width=.5\linewidth]{imagenes/cap3/3021.jpg}
			\caption{Portavoz seleccionado}
		\end{subfigure}%
		\caption{Creación de grupo con dos o más estudiantes}
		\label{fig:310}
	\end{figure}
	
	Al pulsar la acción \textit{Crear} aparecerá el grupo en la sección Grupales. Los grupos formados por dos o más estudiantes desde el punto de vista de la persona docente tienen como propiedades principales un nombre (identificativo único no modificable elegido por la aplicación basándose en el mayor número identificativo existente, siendo 1 en el caso de que no exista ningún grupo aún), una lista de participantes (integrantes del grupo) en la que siempre está incluida la persona docente y una acción para cambiar a la persona portavoz del grupo. La persona portavoz del grupo es una persona estudiante con mayor control sobre el grupo que una persona estudiante no portavoz. Un grupo solo puede tener una única persona portavoz, la cual es asignada por la persona docente de forma manual como se ha mostrado o de forma aleatoria, como se indicará más adelante. Se muestra quién es la persona portavoz debajo del nombre del grupo. La persona docente tiene la potestad de cambiar a la persona portavoz del grupo pulsando el la acción \textit{Cambiar portavoz} y mostrar la lista de participantes del grupo.
	
	\begin{figure}[H]
		\captionsetup{justification=centering}
		\begin{subfigure}[t]{.33\textwidth}
			\centering
			\includegraphics[width=.5\linewidth]{imagenes/cap3/3022.jpg}
			\caption{Representación gráfica de un grupo desde el punto de vista del docente}
		\end{subfigure}%
		\begin{subfigure}[t]{.33\textwidth}
			\centering
			\includegraphics[width=.5\linewidth]{imagenes/cap3/3023.jpg}
			\caption{Cambio de portavoz}
		\end{subfigure}%
		\begin{subfigure}[t]{.33\textwidth}
			\centering
			\includegraphics[width=.5\linewidth]{imagenes/cap3/3024.jpg}
			\caption{Lista con los nombres de los participantes del grupo}
		\end{subfigure}%
		\caption{Grupo y sus acciones}
		\label{fig:311}
	\end{figure}
	
    Con el fin de crear un canal de comunicación entre las personas participantes de un grupo se dota a estas de salas de mensajería, de ahora en adelante llamadas salas de chat. Los grupos privados con una persona estudiante cuenta con una única sala de chat, mientras que los grupos de dos o más estudiantes cuentan con dos salas de chat, una en la que está incluida la persona docente y otra en la que no. En los grupos de dos o más personas estudiantes, la persona docente tiene únicamente acceso a la sala de chat en la que está incluida, y la única información que se le muestra sobre la sala de chat en la que están incluidas las personas estudiantes es el número de mensajes enviados en dicha sala, indicados debajo del nombre de la persona portavoz del grupo. Se ha tomado esta decisión de diseño debido a que se pretende que toda la comunicación de los grupos ocurra dentro de la aplicación desarrollada. Si a las personas estudiantes no se les da la opción de tener un espacio privado en el que comunicarse podrían optar por aplicaciones de mensajería externas. Esto se quiere evitar a toda costa ya que como se verá más adelante el número de mensajes enviados por las personas estudiantes es importante.
	La persona docente puede eliminar los grupos de dos o más estudiantes si así lo desea pulsando el la acción \textit{Eliminar}. Esta acción tiene que realizase con precaución, puesto que eliminará todos los datos relacionados con el grupo.\par
	
	La persona docente puede acceder a la sala de chat del grupo pulsando sobre él:
	
	\begin{figure}[H]
		\captionsetup{justification=centering}
		\centering
		\includegraphics[width=.25\linewidth]{imagenes/cap3/3025.jpg}
		\caption{Sala de chat sin ningún mensaje}
		\label{fig:312}
	\end{figure}
	
	Una sala de chat está formada por un contenedor de los mensajes del chat, una entrada de texto para enviarlos y dos acciones, una con la que se pueden seleccionar archivos del dispositivo móvil (acción con el símbolo de archivo) y una acción para enviar el mensaje de texto introducido en la entrada de texto (acción con el símbolo de avión de papel). Los formatos de archivos soportados por el chat son tres: PDF, JPEG (JPG) y PNG. Hay tres tipos de diseño de mensajes: uno los mensajes propios (enviados por la persona usuaria que ha iniciado sesión) de una tonalidad de azul oscuro y con alineamiento a la izquierda, uno para los mensajes recibidos de otras personas usuarias con una tonalidad de azul claro y con alineamiento a la derecha y uno para los mensajes que contienen archivos con una tonalidad gris y que ocupa todo el ancho de la pantalla, independientemente de quién los envíe. Los mensajes cuentan con la fecha y hora en la que se envió el archivo. El chat soporta enlaces a recursos externos, permitiendo a la persona docente compartir con las personas estudiantes contenido en línea. A continuación se muestra un ejemplo de una conversación entre la persona docente y las personas estudiantes del grupo:
	
	\begin{figure}[H]
		\captionsetup{justification=centering}
		\begin{subfigure}[t]{.5\textwidth}
			\centering
			\includegraphics[width=.5\linewidth]{imagenes/cap3/3026.jpg}
			\caption{Intercambio de mensajes entre docente y estudiantes}
		\end{subfigure}%
		\begin{subfigure}[t]{.5\textwidth}
			\centering
			\includegraphics[width=.5\linewidth]{imagenes/cap3/3026-1.jpg}
			\caption{Envío de recursos por enlace externo}
		\end{subfigure}%
		\caption{Conversación entre docente y estudiantes de un grupo}
		\label{fig:313}
	\end{figure}
	
	Los archivos adjuntos a los mensajes se pueden descargar pulsando el botón con el icono de descarga.\par
	Por otra parte, las personas estudiantes pueden hablar en su grupo de chat con el objetivo de organizarse en la tarea enviada por la persona docente. La persona docente puede comprobar en tiempo real en número de mensajes enviados entre estudiantes por su sala de chat, apareciendo dicho número debajo del nombre de la persona portavoz del grupo:
	
	\begin{figure}[H]
		\captionsetup{justification=centering}
		\begin{subfigure}[t]{.5\textwidth}
			\centering
			\includegraphics[width=.5\linewidth]{imagenes/cap3/3027.jpg}
			\caption{Sala de chat de los estudiantes}
		\end{subfigure}%
		\begin{subfigure}[t]{.5\textwidth}
			\centering
			\includegraphics[width=.5\linewidth]{imagenes/cap3/3028.jpg}
			\caption{Vista del docente de la actividad entre estudiantes}
		\end{subfigure}%
		\caption{Interacciones entre los estudiantes de un grupo}
		\label{fig:314}
	\end{figure}
	
	Del mismo modo, si la persona docente selecciona un grupo individual con una persona estudiante será redirigida a la sala de chat privada con esa persona estudiante. Este es un ejemplo de conversación privada entre docente y estudiante:
	
	\begin{figure}[H]
		\captionsetup{justification=centering}
		\centering
		\includegraphics[width=.25\linewidth]{imagenes/cap3/3029.jpg}
		\caption{Conversación privada entre docente y estudiante}
		\label{fig:315}
	\end{figure}
	
	Hemos visto la creación de un único grupo por parte de la persona docente. Si bien esta funcionalidad es útil en caso de que sea la persona docente quien quiera tener el control sobre qué estudiantes integran el grupo esto puede convertirse en una tarea poco práctica en caso de tener una cantidad elevada de estudiantes en la asignatura. Es por esto que la segunda opción con la que cuenta la persona docente en el menú desplegable de la pantalla Grupos es el menú de creación de grupos. Esta opción tiene dos modos de creación de grupos: indicando el número de estudiantes deseado por grupo o indicando el número de grupos deseados. 
	
	\begin{figure}[H]
		\captionsetup{justification=centering}
		\centering
		\includegraphics[width=.25\linewidth]{imagenes/cap3/3030.jpg}
		\caption{Menú de creación de grupos de forma automática}
		\label{fig:316}
	\end{figure}
	
	Seleccionando el primer modo de esta funcionalidad se pide introducir el número de estudiantes deseado por grupo. La asignatura de ejemplo tiene 15 estudiantes, por lo que si se decide crear grupos de 5 estudiantes se crearán 3 grupos. Si se decide crear grupos de un número de estudiantes que no es múltiplo del número total de estudiantes de la asignatura, como por ejemplo de 4 estudiantes, se crearán 4 grupos (3 grupos de 4 estudiantes y un grupo de 3 estudiantes). Si se marca la casilla del final, se crearán 3 grupos (2 grupos de 4 estudiantes y 1 grupo de 7 estudiantes). En este ejemplo creamos 2 grupos más de 7 estudiantes cada uno, lo que resulta en un grupo de 7 estudiantes y un grupo de 8 estudiantes. En este caso en el que la división dé una persona estudiante restante marcar la casilla o no no aplica ningún efecto puesto que una persona estudiante no puede estar sola en un grupo, por lo que se le incluirá automáticamente en un grupo de un tamaño mayor al especificado.
	
	\begin{figure}[H]
		\captionsetup{justification=centering}
		\begin{subfigure}[t]{.5\textwidth}
			\centering
			\includegraphics[width=.5\linewidth]{imagenes/cap3/3031.jpg}
			\caption{Primer modo de creación de grupos automático}
		\end{subfigure}%
		\begin{subfigure}[t]{.5\textwidth}
			\centering
			\includegraphics[width=.5\linewidth]{imagenes/cap3/3032.jpg}
			\caption{Grupos creados con el primer modo}
		\end{subfigure}%
		\caption{Primer modo del menú de creación de grupos}
		\label{fig:317}
	\end{figure}
	
	\begin{figure}[H]
		\captionsetup{justification=centering}
		\begin{subfigure}[t]{.5\textwidth}
			\centering
			\includegraphics[width=.5\linewidth]{imagenes/cap3/3033.jpg}
			\caption{Participantes del Grupo 2}
		\end{subfigure}%
		\begin{subfigure}[t]{.5\textwidth}
			\centering
			\includegraphics[width=.5\linewidth]{imagenes/cap3/3034.jpg}
			\caption{Participantes del grupo 3}
		\end{subfigure}%
		\caption{Participantes de los grupos creados con el primer modo}
		\label{fig:318}
	\end{figure}
	
	Antes de crear un grupo, se comprueba si el grupo que se va a crear ya existe, y en caso de que sea así no lo crea de nuevo. Las personas participantes del lote o \textit{batch} de grupos creados con este sistema son asignadas a los grupos de forma aleatoria, así como la elección de las personas portavoces.\par
	
	Con el segundo modo de creación de grupos se puede crear un número fijo de grupos. Siempre se va a crear el número especificado de grupos, con independencia del resto que resulte de la operación. Si se decide crear 2 grupos sería equivalente al caso del ejemplo anterior, por lo que se crearía un grupo de 7 estudiantes y un grupo de 8 estudiantes.
	
	\begin{figure}[H]
		\captionsetup{justification=centering}
		\begin{subfigure}{.5\textwidth}
			\centering
			\includegraphics[width=.5\linewidth]{imagenes/cap3/3035.jpg}
			\caption{Segundo modo de creación de grupos automático}
		\end{subfigure}%
		\begin{subfigure}{.5\textwidth}
			\centering
			\includegraphics[width=.5\linewidth]{imagenes/cap3/3036.jpg}
			\caption{Grupos creados con el segundo modo}
		\end{subfigure}%
		\caption{Segundo modo de creación de grupos}
		\label{fig:319}
	\end{figure}
	
	\begin{figure}[H]
		\captionsetup{justification=centering}
		\begin{subfigure}{.5\textwidth}
			\centering
			\includegraphics[width=.5\linewidth]{imagenes/cap3/3037.jpg}
			\caption{Participantes del Grupo 4}
		\end{subfigure}%
		\begin{subfigure}{.5\textwidth}
			\centering
			\includegraphics[width=.5\linewidth]{imagenes/cap3/3038.jpg}
			\caption{Participantes del Grupo 5}
		\end{subfigure}%
		\caption{Participantes de los grupos creados con el segundo modo}
		\label{fig:319}
	\end{figure}
	
	La tercera opción con la que cuenta la persona docente en el menú de la pantalla Grupos es el administrador de grupos. Esta opción permite a la persona docente modificar los grupos como desee, moviendo las personas estudiantes de un grupo a otro.
	
	\begin{figure}[H]
		\captionsetup{justification=centering}
		\centering
		\includegraphics[width=.25\linewidth]{imagenes/cap3/3039-1.jpg}
		\caption{Administrador de grupos}
		\label{fig:320}
	\end{figure}
	
	La persona docente selecciona los dos grupos que quiera modificar, momento en el que aparecerán dos listas, la primera correspondiente a los integrantes del primer grupo y la segunda correspondiente a los integrantes del segundo grupo. La persona docente selecciona las personas estudiantes que quiere mover de ambos grupos (o una sola) y realiza la acción Intercambiar. Esto cambiará las listas de las personas participantes de cada grupo. Al realizar la acción Modificar Grupos se cambiarán las personas participantes anteriores por las personas participantes de las nuevas listas. Si se mueve a la persona portavoz del grupo el sistema elegirá una nueva persona portavoz de forma aleatoria. Se muestra un ejemplo moviendo a Estudiante1 y Estudiante2 del Grupo 1 al Grupo 2 y moviendo a Estudiante7 del Grupo 2 al Grupo 1.
	
	\begin{figure}[H]
		\captionsetup{justification=centering}
		\begin{subfigure}[t]{.5\textwidth}
			\centering
			\includegraphics[width=.5\linewidth]{imagenes/cap3/3039.jpg}
			\caption{Porción de listas de los participantes sin modificar}
		\end{subfigure}%
		\begin{subfigure}[t]{.5\textwidth}
			\centering
			\includegraphics[width=.5\linewidth]{imagenes/cap3/3040.jpg}
			\caption{Porción de listas de los participantes modificadas}
		\end{subfigure}%
		\caption{Modificación de los participantes de dos grupos}
		\label{fig:320}
	\end{figure}
	
	\begin{figure}[H]
		\captionsetup{justification=centering}
		\begin{subfigure}{.5\textwidth}
			\centering
			\includegraphics[width=.5\linewidth]{imagenes/cap3/3041.jpg}
			\caption{Grupo 1}
		\end{subfigure}%
		\begin{subfigure}{.5\textwidth}
			\centering
			\includegraphics[width=.5\linewidth]{imagenes/cap3/3042.jpg}
			\caption{Grupo 2}
		\end{subfigure}%
		\caption{Participantes de los grupos antes de su modificación}
		\label{fig:321}
	\end{figure}
	
	\begin{figure}[H]
		\captionsetup{justification=centering}
		\begin{subfigure}{.5\textwidth}
			\centering
			\includegraphics[width=.5\linewidth]{imagenes/cap3/3043.jpg}
			\caption{Grupo 1}
		\end{subfigure}%
		\begin{subfigure}{.5\textwidth}
			\centering
			\includegraphics[width=.5\linewidth]{imagenes/cap3/3044.jpg}
			\caption{Grupo 2}
		\end{subfigure}%
		\caption{Participantes de los grupos después de su modificación}
		\label{fig:322}
	\end{figure}
	
	%TODO incluir la creación de un grupo con una petición de un estudiante
	
	Por último, si la persona docente quiere comprobar en el grupo o grupos que está una persona estudiante puede hacerlo buscando su nombre en la barra de búsqueda superior. Por ejemplo, Estudiante6 se encuentra en 2 grupos, el Grupo 3 y el Grupo 5.
	
	\begin{figure}[H]
		\captionsetup{justification=centering}
		\centering
		\includegraphics[width=.25\linewidth]{imagenes/cap3/3045.jpg}
		\caption{Búsqueda de los grupos en los que está un estudiante concreto}
		\label{fig:323}
	\end{figure}
	
	
	\subsubsection{Pantalla Archivos}
	
	Esta pantalla consiste en un resumen de los archivos que se han enviado por todos las salas de chat a las que la persona docente tiene acceso. Consta de dos secciones: la sección Grupales, en la que se muestran los archivos de los grupos con más de dos estudiantes, y la sección Individuales, donde aparecen los archivos de los grupos privados de las personas estudiantes con la persona docente. Pulsando en la acción Ver archivos se mostrarán los archivos enviados por cada grupo, que constan del nombre de la persona usuaria que lo ha subido, el nombre del archivo y la fecha de subida en caso de que se quiera comprobar si un archivo ha sido subido con anterioridad a la fecha límite de una entrega, por ejemplo. De igual forma que en los chats, los archivos se pueden descargar pulsando en la correspondiente acción a la derecha de la información sobre el archivo.
	
	\begin{figure}[H]
		\captionsetup{justification=centering}
		\begin{subfigure}[t]{.5\textwidth}
			\centering
			\includegraphics[width=.5\linewidth]{imagenes/cap3/3046.jpg}
			\caption{Archivos de los grupos con más de dos estudiantes}
		\end{subfigure}%
		\begin{subfigure}[t]{.5\textwidth}
			\centering
			\includegraphics[width=.5\linewidth]{imagenes/cap3/3047.jpg}
			\caption{Archivos de los grupos privados con el estudiante}
		\end{subfigure}%
		\caption{Pantalla Archivos}
		\label{fig:324}
	\end{figure}
	
	\subsubsection{Pantalla Interactividad}
	
	Esta es la segunda pantalla más importante de la aplicación después de la pantalla Grupos. Es en esta pantalla donde la persona docente puede crear actividades para los grupos. Se incluyen dos tipos de actividades: de tipo entrada de texto y de tipo multirrespuesta y un menú de creación de eventos. Las actividades solo se pueden enviar a los grupos formados por dos o más estudiantes, no siendo posible enviar actividades a los grupos individuales de la persona docente junto con la persona estudiante. La persona docente no puede enviar una actividad si no ha creado ningún grupo. En caso de que exista al menos un grupo, se mostrará un menú en la parte inferior derecha de esta pantalla, el cual al presionarlo se mostrarán tres nuevas acciones.
	
	\begin{figure}[H]
		\captionsetup{justification=centering}
		\begin{subfigure}[t]{.5\textwidth}
			\centering
			\includegraphics[width=.5\linewidth]{imagenes/cap3/3048.jpg}
			\caption{Pantalla sin haber sido creado ningún grupo}
		\end{subfigure}%
		\begin{subfigure}[t]{.5\textwidth}
			\centering
			\includegraphics[width=.5\linewidth]{imagenes/cap3/3049.jpg}
			\caption{Pantalla con grupos creados sin haber enviado ninguna actividad. Menú de creación de actividades desplegado}
		\end{subfigure}%
		\caption{Pantalla Interactividad}
		\label{fig:325}
	\end{figure}
	
	Las actividades tienen dos tipos de opciones. Atendiendo a la función pedagógica que deben de cumplir los EVAs, se definen varias modalidades de actividades recogidas en la siguiente tabla:
	
	\begin{figure}[H]
		\centering
		\captionsetup{justification=centering}
		\begin{tabular}{|c|c|c|} \hline
			\textbf{Actividad evaluable} & \textbf{Actividad grupal} & \textbf{Tipo de actividad} \\ \hline
			Sí &  Sí & Actividad evaluable grupal\\ \hline
			No & Sí & Actividad no evaluable grupal\\ \hline
			Sí & No & Actividad evaluable no grupal\\ \hline
			No & No & Actividad no evaluable no grupal \\ \hline
		\end{tabular}
		\captionof{table}{Modalidades de una actividad}
	\end{figure}
	
	La actividades evaluables grupales permiten una evaluación numérica de los grupos. Este tipo de actividades admiten únicamente una respuesta conjunta del equipo. Dicha respuesta será introducida por la persona portavoz del grupo, la cual se supone que cuenta con la confianza de la persona docente y de las personas estudiantes que conforman el mismo. Las actividades no evaluables grupales tienen mero carácter informativo, es decir, la persona docente creará una actividad de este tipo cuando quiera saber la opinión de los equipos sobre cierto tema, siendo la persona portavoz del equipo quien dé la opinión a cierta cuestión. Las actividades evaluables no grupales se envían de forma individual a cada persona miembro del equipo, y cada persona estudiante contesta a ellas de forma individual. Este tipo de actividades están pensadas para que la persona docente compruebe si el desempeño de las personas estudiantes es mejor en grupo o de forma individual, pudiendo comparar los resultados de estas actividades con las actividades evaluables grupales. Las actividades no evaluables no grupales se envían de forma individual a cada persona miembro del equipo, contestando cada persona estudiante de forma individual. Este tipo de actividades, al igual que las actividades no evaluables grupales, tienen mero carácter informativo, y la persona docente debe de usarlas cuando quiera comprobar si las opiniones de las personas estudiantes que componen el grupo distan mucho las unas entre las otras.\par
	
	Para cada tipo de actividad se irán mostrando las diferentes modalidades. El primer tipo de actividad que se ofrece es de tipo entrada de texto.
	La primera acción empezando desde arriba del menú de creación de actividades muestra un menú de creación de una actividad de tipo entrada de texto. 
	
	\begin{figure}[H]
		\captionsetup{justification=centering}
		\centering
		\includegraphics[width=.25\linewidth]{imagenes/cap3/3050.jpg}
		\caption{Menú de creación de tipo entrada de texto}
		\label{fig:326}
	\end{figure}
	
	La persona docente selecciona los grupos a los que desea enviar la actividad. Después, introduce el título de la actividad o pregunta que quiera realizar a las personas estudiantes. Con motivos de demostración, se han borrado los grupos existentes y creado dos nuevos con el mismo número de estudiantes. En el Grupo 1 se encuentran las personas estudiantes con nombre EstudianteX, donde X va del 1 al 5, ambos inclusive, y en el Grupo 2 se encuentran las personas estudiantes con nombre EstudianteY, donde Y va desde el 6 al 10, ambos inclusive. La persona portavoz del Grupo 1 es Estudiante1 y la persona portavoz del Grupo 2 es Estudiante6.\par 
	
	Primero se mostrarán las modalidad de la actividad de tipo entrada de texto en el siguiente orden: evaluables grupales, evaluables no grupales, no evaluables grupales y no evaluables no grupales. Después se mostrarán las actividades de tipo multirrespuesta en el mismo orden de modalidad.\par
	
	Si la persona docente elige que la actividad es de tipo evaluable grupal deberá marcar ambas casillas: 
	
	\begin{figure}[H]
		\captionsetup{justification=centering}
		\centering
		\includegraphics[width=.25\linewidth]{imagenes/cap3/3051.jpg}
		\caption{Ejemplo de la creación de una actividad de tipo entrada de texto}
		\label{fig:327}
	\end{figure}
	
	Al crear la actividad aparecerán dos carpetas incluyendo las actividades enviadas a cada grupo. Dichas carpetas (tarjetas) cuentan con una acción que permite mostrar las estadísticas únicamente de las actividades que se hayan marcado como evaluables, tanto grupales como individuales. Si no se ha enviado ninguna actividad, se avisará a la persona docente que no hay actividades evaluadas de momento. También cuenta con una acción llamada \textit{Mostrar ocultas}. Esto es así porque las actividades, si son borradas, desaparecen las estadísticas asignadas con ellas, por lo que se le da la opción a la persona docente de ocultarlas para evitar un exceso de actividades mostradas, pudiendo ocultar las que estén completadas o las que desee. Se pueden mostrar las actividades asignadas a cada grupo pulsando en la acción \textit{Ver actividades}.
	
	\begin{figure}[H]
		\captionsetup{justification=centering}
		\begin{subfigure}[t]{.5\textwidth}
			\centering
			\includegraphics[width=.5\linewidth]{imagenes/cap3/3052.jpg}
			\caption{Representación de una actividad de tipo entrada de texto}
		\end{subfigure}%
		\begin{subfigure}[t]{.5\textwidth}
			\centering
			\includegraphics[width=.5\linewidth]{imagenes/cap3/3053.jpg}
			\caption{Estadísticas de las actividades del grupo}
		\end{subfigure}%
		\caption{Acciones del contenedor de actividades de un grupo}
		\label{fig:328}
	\end{figure}
	
	Las actividades cuentan con el tipo de actividad que es (de tipo entrada de texto o multirrespuesta) el título de la actividad, la modalidad de la actividad y en caso de que sea grupal un mensaje indicando si la persona portavoz del grupo ha respondido a la actividad o no.
	Una vez la persona portavoz ha contestado, la persona docente recibe unos avisos en la actividad que indican \enquote{El grupo ha contestado} y \enquote{Evalúa la respuesta del grupo} en caso de no haber sido evaluada todavía. Puede mostrar la respuesta pulsando en la acción Ver respuesta. Debido a que las repuestas a estas preguntas tienen un grado de subjetividad, la persona docente se encarga de introducir la nota manualmente. Para ello, tiene un deslizador a la izquierda que va desde el 0 al 10 en intervalos de 0.5. Una vez ha seleccionado la nota, la persona docente pulsa la acción \textit{Poner nota}. La calificación aparecerá en la actividad junto con la nota máxima de la misma, pudiendo ver u ocultar la respuesta del grupo para futuras comprobaciones.
	
	\begin{figure}[H]
		\captionsetup{justification=centering}
		\begin{subfigure}[t]{.33\textwidth}
			\centering
			\includegraphics[width=.5\linewidth]{imagenes/cap3/3054.jpg}
			\caption{Aviso de respuesta del portavoz recibida}
		\end{subfigure}%
		\begin{subfigure}[t]{.33\textwidth}
			\centering
			\includegraphics[width=.5\linewidth]{imagenes/cap3/3055.jpg}
			\caption{Calificación de la actividad}
		\end{subfigure}%
		\begin{subfigure}[t]{.33\textwidth}
			\centering
			\includegraphics[width=.5\linewidth]{imagenes/cap3/3056.jpg}
			\caption{Actividad con calificación}
		\end{subfigure}%
		\caption{Calificación de una actividad de tipo entrada de texto}
		\label{fig:329}
	\end{figure}
	
	Como inciso, la idea de los chats de estudiantes es que puedan debatir una respuesta en común a la pregunta. Como grupo, se debe aprender a trabajar en equipo y a debatir, y que los y las estudiantes comprendan que sus acciones como grupo tienen consecuencias en sus calificaciones, por lo que se les debe motivar a trabajar en equipo. Este es un ejemplo de interacción esperado entre las personas estudiantes:
	
	\begin{figure}[H]
		\captionsetup{justification=centering}
		\centering
		\includegraphics[width=.25\linewidth]{imagenes/cap3/3057.jpg}
		\caption{Debate de los estudiantes sobre la actividad}
		\label{fig:330}
	\end{figure}
	
	La persona docente puede comprobar la actividad de las personas estudiantes mirando el contador de mensajes del grupo de estudiantes e intervenir en la actividad.
	
	\begin{figure}[H]
		\captionsetup{justification=centering}
		\begin{subfigure}[t]{.33\textwidth}
			\centering
			\includegraphics[width=.5\linewidth]{imagenes/cap3/3058.jpg}
			\caption{Comprobación del docente de mensajes enviados entre estudiantes}
		\end{subfigure}%
		\begin{subfigure}[t]{.33\textwidth}
			\centering
			\includegraphics[width=.5\linewidth]{imagenes/cap3/3059.jpg}
			\caption{Mensaje del docente al Grupo 1}
		\end{subfigure}%
		\begin{subfigure}[t]{.33\textwidth}
			\centering
			\includegraphics[width=.5\linewidth]{imagenes/cap3/3060.jpg}
			\caption{Mensaje del docente al Grupo 2}
		\end{subfigure}%
		\caption{Interacciones del docente con estudiantes estudiantes esperadas}
		\label{fig:331}
	\end{figure}
	
	Se ha enviado otra actividad igual para ambos grupos y se han calificado ambas. La persona docente puede comprobar la media de las notas de este tipo de actividades pulsando en la acción anteriormente mencionada para cada equipo. El Grupo 1 ha obtenido unas calificaciones de 10 y 9, por lo que su media en esta modalidad de actividad es de 9.5. El Grupo 2 ha obtenido unas calificaciones de 6 y 5.5, por lo que su media es de 5.75 en esta modalidad de actividad. Dependiendo del rango de la nota esta se mostrará con un color distinto (rojo si es inferior a 5, amarillo si es mayor o igual a 5 y menor que 7, verde de tipo 1 si es mayor o igual a 7 y menor que 9 y verde de tipo 2 si es mayor o igual a 9).\par
	
	\begin{figure}[H]
		\captionsetup{justification=centering}
		\begin{subfigure}{.5\textwidth}
			\centering
			\includegraphics[width=.5\linewidth]{imagenes/cap3/3061.jpg}
			\caption{Media de actividades del Grupo 1}
		\end{subfigure}%
		\begin{subfigure}{.5\textwidth}
			\centering
			\includegraphics[width=.5\linewidth]{imagenes/cap3/3062.jpg}
			\caption{Media de actividades del Grupo 2}
		\end{subfigure}%
		\caption{Estadísticas de ambos grupos}
		\label{fig:332}
	\end{figure}
	
	En cuanto a las actividades evaluables no grupales, la persona docente crea la actividad de la misma forma, ahora sin seleccionar la casilla grupal. Aparecerá el mensaje \enquote{Ningún estudiante ha contestado todavía} en caso de que ningún estudiante del grupo haya contestado a la pregunta.
	
	\begin{figure}[H]
		\captionsetup{justification=centering}
		\centering
		\includegraphics[width=.25\linewidth]{imagenes/cap3/3063.jpg}
		\caption{Actividad de tipo entrada de texto evaluable no grupal}
		\label{fig:333}
	\end{figure}
	
	En la actividad aparecerá un contador con las personas estudiantes que han contestado y los que no. Una vez hayan contestado, aparecerá un mensaje indicándolo, junto con otro mensaje que indica las respuestas que quedan por evaluar por parte de la persona docente:
	
	\begin{figure}[H]
		\captionsetup{justification=centering}
		\begin{subfigure}{.5\textwidth}
			\centering
			\includegraphics[width=.5\linewidth]{imagenes/cap3/3064.jpg}
			\caption{Actividad no grupal contestada por unos cuantos estudiantes}
		\end{subfigure}%
		\begin{subfigure}{.5\textwidth}
			\centering
			\includegraphics[width=.5\linewidth]{imagenes/cap3/3065.jpg}
			\caption{Actividad no grupal contestada por todos los estudiantes}
		\end{subfigure}%
		\caption{Información de actividad de tipo entrada de texto evaluable no grupal}
		\label{fig:334}
	\end{figure}
	
	La persona docente se encargará de introducir las notas una a una. Nótese que no aparece el nombre de la persona estudiante que ha contestado la pregunta. El sistema de evaluación se ha diseñado así con el objetivo de evitar sesgos a la hora de evaluar una respuesta. Se irá mostrando la media de las respuestas evaluadas mostrando cuántas personas estudiantes quedan por evaluar. Una vez hayan sido todas evaluadas, se mostrará un mensaje indicándolo.
	
	\begin{figure}[H]
		\captionsetup{justification=centering}
		\begin{subfigure}[t]{.5\textwidth}
			\centering
			\includegraphics[width=.5\linewidth]{imagenes/cap3/3066.jpg}
			\caption{Respuestas de los estudiantes del grupo a la actividad}
		\end{subfigure}%
		\begin{subfigure}[t]{.5\textwidth}
			\centering
			\includegraphics[width=.5\linewidth]{imagenes/cap3/3067.jpg}
			\caption{Nota media de todas las respuestas}
		\end{subfigure}%
		\caption{Evaluación de actividad evaluable no grupal}
		\label{fig:335}
	\end{figure}
	
	Se ha evaluado también al Grupo 2. Además, se ha realizado otra pregunta no grupal con carácter evaluable. El Grupo 1 ha obtenido unas calificaciones de 7.6 y 6.4, dando una media de 7, mientras que el Grupo 2 ha obtenido unas calificaciones de 8 y 9, dando una media de 8.5. Esto se puede comprobar por parte de la persona docente viendo las estadísticas de cada grupo:
	
	\begin{figure}[H]
		\captionsetup{justification=centering}
		\begin{subfigure}{.5\textwidth}
			\centering
			\includegraphics[width=.5\linewidth]{imagenes/cap3/3068.jpg}
			\caption{Medias del Grupo 1}
		\end{subfigure}%
		\begin{subfigure}{.5\textwidth}
			\centering
			\includegraphics[width=.5\linewidth]{imagenes/cap3/3069.jpg}
			\caption{Medias del Grupo 2}
		\end{subfigure}%
		\caption{Medias de las actividades evaluables de tipo entrada de texto}
		\label{fig:336}
	\end{figure}
	
	Con esta información, la persona docente puede llegar a la conclusión de que el Grupo 1 trabaja mejor en grupo que el Grupo 2, mientras que las personas miembros del Grupo 2 obtienen mejor calificaciones al trabajar de forma individual que el Grupo 1. Con esto en cuenta, la persona docente puede tomar las medidas que considere oportunas (modificar el grupo, eliminar el grupo, cambiar a la persona portavoz del grupo, etc).\par
	
	En cuanto a las actividades no evaluables grupales, estas funcionan de igual forma que las actividades evaluables grupales, salvo que en este caso la persona docente recibe la respuesta de la persona portavoz del grupo y no se le da la opción de evaluarla. Es por esto que las actividades no evaluables son meramente informativas como ya se ha comentado, y no recopilan estadísticas.
	
	\begin{figure}[H]
		\captionsetup{justification=centering}
		\begin{subfigure}{.5\textwidth}
			\centering
			\includegraphics[width=.5\linewidth]{imagenes/cap3/3070.jpg}
			\caption{Enunciado de actividad no evaluable grupal}
		\end{subfigure}%
		\begin{subfigure}{.5\textwidth}
			\centering
			\includegraphics[width=.5\linewidth]{imagenes/cap3/3071.jpg}
			\caption{Respuesta a actividad no evaluable grupal}
		\end{subfigure}%
		\caption{Ejemplo de actividad no evaluable grupal}
		\label{fig:337}
	\end{figure}
	
	De igual forma, la actividades no evaluables no grupales funcionan de igual forma que las actividades evaluables no grupales salvo que a la persona docente no se le da la opción de evaluar las respuestas de las personas estudiantes. Esta modalidad de actividad puede ser usada por la persona docente para conocer la opinión de las personas integrantes del grupo de forma individual sobre cierta cuestión y comprobar si distan mucho entre sí para saber el grado de afinidad entre las personas estudiantes del mismo grupo.
	
	\begin{figure}[H]
		\captionsetup{justification=centering}
		\begin{subfigure}{.5\textwidth}
			\centering
			\includegraphics[width=.5\linewidth]{imagenes/cap3/3072.jpg}
			\caption{Enunciado de actividad no evaluable no grupal}
		\end{subfigure}%
		\begin{subfigure}{.5\textwidth}
			\centering
			\includegraphics[width=.5\linewidth]{imagenes/cap3/3073.jpg}
			\caption{Respuestas individuales a actividad no evaluable}
		\end{subfigure}%
		\caption{Ejemplo de actividad de tipo entrada de texto no evaluable no grupal}
		\label{fig:338}
	\end{figure}
	
	Por otra parte tenemos las actividades de tipo multirrespuesta. La persona docente puede crearlas pulsando en la segunda acción del menú de creación de actividades. En ese momento aparecerá un menú con el que se pueden crear dichas actividades. De igual forma que las actividades de tipo entrada de texto, la persona docente introduce el título de la actividad y la modalidad de la actividad. A continuación, introduce las opciones que conformarán la actividad. Las actividades multirrespuesta deben de tener un mínimo de dos opciones y un máximo de 5. La persona docente debe introducir una a una pulsando en la acción con el símbolo \enquote{más}. 
	
	\begin{figure}[H]
		\captionsetup{justification=centering}
		\begin{subfigure}{.5\textwidth}
			\centering
			\includegraphics[width=.5\linewidth]{imagenes/cap3/3074.jpg}
			\caption{Menú de creación de una actividad tipo multirrespuesta}
		\end{subfigure}%
		\begin{subfigure}{.5\textwidth}
			\centering
			\includegraphics[width=.5\linewidth]{imagenes/cap3/3075.jpg}
			\caption{Ejemplo de creación de una actividad multirrespuesta}
		\end{subfigure}%
		\caption{Ejemplo de creación de una actividad multirrespuesta}
		\label{fig:339}
	\end{figure}
	
	En el caso de que la actividad sea de tipo multirrespuesta evaluable se debe seleccionar la respuesta correcta en el listado de opciones añadidas que aparecerá una vez se agregue la primera opción.
	
	\begin{figure}[H]
		\captionsetup{justification=centering}
		\begin{subfigure}{.5\textwidth}
			\centering
			\includegraphics[width=.5\linewidth]{imagenes/cap3/3076.jpg}
			\caption{Listado de opciones añadidas a la actividad}
		\end{subfigure}%
		\begin{subfigure}{.5\textwidth}
			\centering
			\includegraphics[width=.5\linewidth]{imagenes/cap3/3077.jpg}
			\caption{Selección de la respuesta correcta de la actividad}
		\end{subfigure}%
		\caption{Ejemplo de creación de una actividad multirrespuesta evaluable}
		\label{fig:340}
	\end{figure}
	
	La actividad aparecerá en las actividades enviadas a los grupos seleccionados. La información que se muestra en las actividades multirrespuesta de tipo evaluable grupal es la que ya se incluía en las de tipo entrada de texto (tipo de actividad, título de la actividad, modalidad de la actividad), añadiendo la opción de la respuesta correcta a la actividad en caso de que sea evaluable. De nuevo, la persona portavoz del grupo será la encargada de contestar la actividad. Cuando la persona portavoz conteste se informará a la persona docente de ello junto con la opción escogida. En las actividades de tipo multirrespuesta la persona docente no tiene que introducir la nota, puesto que la aplicación se encarga de comprobar si la actividad ha sido correctamente contestada o no. Las actividades de este tipo erróneamente contestadas no restan puntuación.
	
	\begin{figure}[H]
		\captionsetup{justification=centering}
		\begin{subfigure}[t]{.5\textwidth}
			\centering
			\includegraphics[width=.5\linewidth]{imagenes/cap3/3078.jpg}
			\caption{Actividad multirrespuesta evaluable grupal sin contestar}
		\end{subfigure}%
		\begin{subfigure}[t]{.5\textwidth}
			\centering
			\includegraphics[width=.5\linewidth]{imagenes/cap3/3079.jpg}
			\caption{Actividad multirrespuesta evaluable grupal contestada}
		\end{subfigure}%
		\caption{Ejemplo de una actividad multirrespuesta evaluable grupal}
		\label{fig:341}
	\end{figure}
	
	En las estadísticas del grupo aparece la tasa de acierto de las actividades de tipo multirrespuesta. Se ha realizado otra actividad de tipo evaluable grupal y enviado a ambos grupos. El Grupo 1 ha contestado ambas correctamente, por lo que su tasa de acierto es del 100\%. EL Grupo 2 ha contestado una bien y otra no, por lo que su tasa de acierto es del 50\%
	
	\begin{figure}[H]
		\captionsetup{justification=centering}
		\begin{subfigure}{.5\textwidth}
			\centering
			\includegraphics[width=.5\linewidth]{imagenes/cap3/3080.jpg}
			\caption{Estadísticas del Grupo 1}
		\end{subfigure}%
		\begin{subfigure}{.5\textwidth}
			\centering
			\includegraphics[width=.5\linewidth]{imagenes/cap3/3081.jpg}
			\caption{Estadísticas del Grupo 2}
		\end{subfigure}%
		\caption{Estadísticas de las actividades de tipo multirrespuesta}
		\label{fig:342}
	\end{figure}
	
	Las actividades evaluables no grupales se muestran a todas las personas estudiantes por igual. Las respuestas irán apareciendo en la actividad de docente, junto con el porcentaje de estudiantes del grupo que la ha contestado. Una vez todas las personas estudiantes hayan contestado se avisará a la persona docente de ello.
	
	\begin{figure}[H]
		\captionsetup{justification=centering}
		\begin{subfigure}{.5\textwidth}
			\centering
			\includegraphics[width=.5\linewidth]{imagenes/cap3/3082.jpg}
			\caption{Actividad multirrespuesta evaluable no grupal sin contestar}
		\end{subfigure}%
		\begin{subfigure}{.5\textwidth}
			\centering
			\includegraphics[width=.5\linewidth]{imagenes/cap3/3083.jpg}
			\caption{Actividad multirrespuesta evaluable no grupal contestada}
		\end{subfigure}%
		\caption{Ejemplo de actividad multirrespuesta evaluable no grupal}
		\label{fig:343}
	\end{figure}
	
	Se han evaluado dos actividades de este tipo para ambos grupos. El Grupo 1 ha obtenido una tasa de acierto en la primera actividad del 40\% y en la segunda de una tasa de acierto en la segunda actividad del 60\%, por lo que la tasa media total de acierto es del 50\%. El Grupo 2 ha obtenido una tasa de acierto en la primera actividad del 40\% y en la segunda actividad del 80\%, por lo que la tasa media total de acierto es del 60\%. Esto puede indicar que el Grupo 1 trabaja mejor en equipo mientras que el Grupo 2 trabaja mejor de forma individual, como ya se comprobó con las actividades de tipo entrada de texto.
	
	\begin{figure}[H]
		\captionsetup{justification=centering}
		\begin{subfigure}{.5\textwidth}
			\centering
			\includegraphics[width=.5\linewidth]{imagenes/cap3/3084.jpg}
			\caption{Estadísticas del Grupo 1}
		\end{subfigure}%
		\begin{subfigure}{.5\textwidth}
			\centering
			\includegraphics[width=.5\linewidth]{imagenes/cap3/3085.jpg}
			\caption{Estadísticas del Grupo 2}
		\end{subfigure}%
		\caption{Estadísticas de todos los tipos de actividades evaluables}
		\label{fig:344}
	\end{figure}
	
	Las actividades no evaluables grupales tienen el siguiente formato:
	
	\begin{figure}[H]
		\captionsetup{justification=centering}
		\begin{subfigure}{.5\textwidth}
			\centering
			\includegraphics[width=.5\linewidth]{imagenes/cap3/3086.jpg}
			\caption{Actividad multirrespuesta no evaluable grupal sin contestar}
		\end{subfigure}%
		\begin{subfigure}{.5\textwidth}
			\centering
			\includegraphics[width=.5\linewidth]{imagenes/cap3/3087.jpg}
			\caption{Actividad multirrespuesta no evaluable grupal contestada}
		\end{subfigure}%
		\caption{Ejemplo de actividad multirrespuesta no evaluable grupal}
		\label{fig:345}
	\end{figure}
	
	Las actividades no evaluables no grupales tienen el siguiente formato:
	
	\begin{figure}[H]
		\captionsetup{justification=centering}
		\begin{subfigure}{.5\textwidth}
			\centering
			\includegraphics[width=.5\linewidth]{imagenes/cap3/3088.jpg}
			\caption{Actividad multirrespuesta no evaluable no grupal sin contestar}
		\end{subfigure}%
		\begin{subfigure}{.5\textwidth}
			\centering
			\includegraphics[width=.5\linewidth]{imagenes/cap3/3089.jpg}
			\caption{Actividad multirrespuesta no evaluable no grupal contestada}
		\end{subfigure}%
		\caption{Ejemplo actividad multirrespuesta no evaluable no grupal}
		\label{fig:346}
	\end{figure}
	
	La última actividad de la que dispone la persona docente es la de crear eventos, lo que se puede hacer pulsando el último botón del menú. Esta es una actividad especial, puesto que no aparece en la pantalla de interactividad con las demás actividades, sino en la pantalla de Inicio, en la sección Eventos. Al pulsar en la acción aparecerá una pantalla de creación de eventos. La persona docente tiene que seleccionar el grupo o grupos a los que va dirigido el evento, agregar un título descriptivo del evento, la descripción del evento, el lugar del evento y la fecha del evento, la cual se selecciona de un calendario que aparece al pulsar la acción marcada en azul dentro de la pantalla. Si es un evento para toda la clase, se recomienda tener un grupo en el que estén incluidos todas las personas estudiantes. La persona docente puede eliminar los eventos si así lo desea.
	
	\begin{figure}[H]
		\captionsetup{justification=centering}
		\begin{subfigure}{.5\textwidth}
			\centering
			\includegraphics[width=.5\linewidth]{imagenes/cap3/3090.jpg}
			\caption{Menú de creación de eventos}
		\end{subfigure}%
		\begin{subfigure}{.5\textwidth}
			\centering
			\includegraphics[width=.5\linewidth]{imagenes/cap3/3091.jpg}
			\caption{Evento creado por el docente}
		\end{subfigure}%
		\caption{Creación de un evento por parte del docente}
		\label{fig:347}
	\end{figure}
	
	Una vez se han creado grupos y obtenido estadísticas, si la persona docente ha realizado pruebas de evaluación individuales y debido a que no puede saber los nombres de las personas estudiantes a los que está evaluando puede consultar los resultados de una persona estudiante específica dirigiéndose a la pantalla de inicio en la sección de estadísticas. Las estadísticas van acompañadas de un encabezado en el que se indica en qué grupo han sido recogidas.
	
	\begin{figure}[H]
		\captionsetup{justification=centering}
		\centering
		\includegraphics[width=.25\linewidth]{imagenes/cap3/3092.jpg}
		\caption{Estadísticas de un estudiante}
		\label{fig:348}
	\end{figure}
	
	\subsection{Rol de estudiante}
	
	\subsubsection{Pantalla Inicio}
	
	La pantalla de inicio de las personas estudiantes tiene tres secciones al igual que la persona docente. En la primera, llamada clase, las personas estudiantes pueden ver un listado de todas las personas participantes de la asignatura y su información, incluida la información de la persona docente. Las personas estudiantes no pueden buscar a alguien en concreto (pueden buscar a una persona estudiante en particular apoyándose en que la lista está ordenada por orden alfabético), ni pueden ver las estadísticas propias (ya que esta información podría resultar desmotivante en caso de que los resultados fuesen negativos) ni de las demás personas estudiantes. 
	
	
	\begin{figure}[H]
		\captionsetup{justification=centering}
		\centering
		\includegraphics[width=.25\linewidth]{imagenes/cap3/3093.jpg}
		\caption{Lista de todos los participantes de una clase}
		\label{fig:349}
	\end{figure}
	
	En la sección de peticiones la persona estudiante puede eliminar las peticiones de creación de grupo que haya realizado y aceptar o rechazar las peticiones de otras personas estudiantes en caso de que quiera avisar al docente sobre su elección. 
	
	\begin{figure}[H]
		\captionsetup{justification=centering}
		\begin{subfigure}{.5\textwidth}
			\centering
			\includegraphics[width=.5\linewidth]{imagenes/cap3/3094.jpg}
			\caption{Sección Peticiones vacía}
		\end{subfigure}%
		\begin{subfigure}{.5\textwidth}
			\centering
			\includegraphics[width=.5\linewidth]{imagenes/cap3/3095.jpg}
			\caption{Sección Peticiones con dos peticiones}
		\end{subfigure}%
		\caption{Sección Peticiones}
		\label{fig:350}
	\end{figure}
	
	La última sección es la de eventos. Se diferencia de la de la persona docente en que cuenta con dos subsecciones llamadas De portavoces y Del docente. Esto se debe a que las personas portavoces de los equipos son las únicas que tienen la potestad de enviar eventos a sus grupos y de administrarlos. Estos eventos creados por las personas portavoces no aparecen en la sección Del docente. Como esta persona estudiante era portavoz del Grupo 1, puede crear eventos para su grupo.
	
	\begin{figure}[H]
		\captionsetup{justification=centering}
		\begin{subfigure}{.5\textwidth}
			\centering
			\includegraphics[width=.5\linewidth]{imagenes/cap3/3096.jpg}
			\caption{Sección Eventos mostrando los eventos de los portavoces (en este caso no hay eventos que mostrar)}
		\end{subfigure}%
		\begin{subfigure}{.5\textwidth}
			\centering
			\includegraphics[width=.5\linewidth]{imagenes/cap3/3097.jpg}
			\caption{Sección Eventos mostrando los eventos enviados por el docente}
		\end{subfigure}%
		\caption{Ejemplo de creación de una actividad multirrespuesta}
		\label{fig:351}
	\end{figure}
	
	\subsubsection{Pantalla Grupos}
	
    En esta pantalla aparecen los grupos en los que se encuentra la persona estudiante. Las tarjetas de los grupos muestran el nombre del grupo, el nombre de su portavoz (en este caso es la propia persona estudiante, lo cual se indica con un mensaje), la lista de las personas integrantes del grupo y un desplegable que muestra las salas de chat que la persona estudiante tiene disponible, que a diferencia de la persona docente son dos: la sala en la que se encuentran las personas estudiantes y la persona docente y la sala en la que solo se encuentran las personas estudiantes.\par
	
	\begin{figure}[H]
		\captionsetup{justification=centering}
		\begin{subfigure}{.5\textwidth}
			\centering
			\includegraphics[width=.5\linewidth]{imagenes/cap3/3098.jpg}
			\caption{Grupo con salas de chats in mostrarse}
		\end{subfigure}%
		\begin{subfigure}{.5\textwidth}
			\centering
			\includegraphics[width=.5\linewidth]{imagenes/cap3/3099.jpg}
			\caption{Grupo con salas de chat mostrándose}
		\end{subfigure}%
		\caption{Pantalla Grupos}
		\label{fig:352}
	\end{figure}
	
	La persona estudiante no cuenta con un menú como la persona docente, sino que cuenta con tres acciones: creación de petición de grupo, abrir grupo privado con la persona docente y creación de eventos. La primera empezando por la parte de abajo abre el menú para crear una petición:
	
	\begin{figure}[H]
		\captionsetup{justification=centering}
		\centering
		\includegraphics[width=.25\linewidth]{imagenes/cap3/3100.jpg}
		\caption{Menú de creación de una petición}
		\label{fig:353}
	\end{figure}
	
	Si la persona estudiante intenta solicitar la creación de un grupo con las mismas personas participantes de un grupo que ya está creado y en la que está incluida le saldrá un aviso. Además, una persona estudiante no puede tener creadas más de tres peticiones a la vez que no hayan sido aceptadas por la persona docente aún.
	
	\begin{figure}[H]
		\captionsetup{justification=centering}
		\begin{subfigure}[t]{.5\textwidth}
			\centering
			\includegraphics[width=.5\linewidth]{imagenes/cap3/3101.jpg}
			\caption{Intento de petición de creación de un grupo ya existente}
		\end{subfigure}%
		\begin{subfigure}[t]{.5\textwidth}
			\centering
			\includegraphics[width=.5\linewidth]{imagenes/cap3/3102.jpg}
			\caption{Intento de creación de más de tres peticiones}
		\end{subfigure}%
		\caption{Acciones no permitidas en la creación de peticiones}
		\label{fig:354}
	\end{figure}
	
	Pulsando en la segunda opción de las acciones de la pantalla se abre la sala de chat privada con la persona docente. Si esta sala de chat no ha sido creada de forma manual por la persona docente, la sala se creará en el momento que la persona estudiante pulse la acción.
	
	\begin{figure}[H]
		\captionsetup{justification=centering}
		\centering
		\includegraphics[width=.25\linewidth]{imagenes/cap3/3103.jpg}
		\caption{Sala de chat privada con el docente dese el punto de vista del estudiante}
		\label{fig:355}
	\end{figure}
	
	En caso de que la persona estudiante sea portavoz de algún grupo la última de las opciones es equivalente a la actividad de creación de eventos de la persona docente, solo que esta vez en la lista de grupos disponibles a los que enviar el evento solo aparecen los grupos de los que la persona estudiante es portavoz. En caso de que la persona estudiante no sea portavoz de ningún grupo aparecerá un mensaje de error.
	
	\begin{figure}[H]
		\captionsetup{justification=centering}
		\begin{subfigure}[t]{.33\textwidth}
			\centering
			\includegraphics[width=.5\linewidth]{imagenes/cap3/3105.jpg}
			\caption{Menú de creación de eventos en caso de ser portavoz}
		\end{subfigure}%
		\begin{subfigure}[t]{.33\textwidth}
			\centering
			\includegraphics[width=.5\linewidth]{imagenes/cap3/3106.jpg}
			\caption{Mensaje de error en caso de no ser portavoz}
		\end{subfigure}%
		\begin{subfigure}[t]{.33\textwidth}
			\centering
			\includegraphics[width=.5\linewidth]{imagenes/cap3/3110.jpg}
			\caption{Evento creado por un estudiante portavoz}
		\end{subfigure}%
		\caption{Creación de eventos por parte del estudiante}
		\label{fig:356}
	\end{figure}
	
	De igual forma que la persona docente, la persona estudiante puede buscar los grupos en los que esté una persona estudiante concreta, pero solo de los grupos en los que se encuentra incluida la propia persona estudiante.
	
	\begin{figure}[H]
		\captionsetup{justification=centering}
		\begin{subfigure}{.5\textwidth}
			\centering
			\includegraphics[width=.5\linewidth]{imagenes/cap3/3107.jpg}
			\caption{Estudiante que se encuentra en un grupo donde se encuentra también el estudiante que lo busca}
		\end{subfigure}%
		\begin{subfigure}{.5\textwidth}
			\centering
			\includegraphics[width=.5\linewidth]{imagenes/cap3/3108.jpg}
			\caption{Estudiante que no se encuentra en un grupo con el estudiante que lo busca}
		\end{subfigure}%
		\caption{Búsqueda de un estudiante}
		\label{fig:358}
	\end{figure}
	
	\subsubsection{Pantalla Archivos}
	
	De igual forma que la persona docente, la pantalla de archivos de la persona estudiante consiste en un resumen de los archivos enviados por los grupos en los que se encuentra, tanto del chat con la persona docente como en el que no. A diferencia que la persona docente, la persona estudiante no tiene una sección en la que le aparece un resumen de los archivos enviados por su chat privado con la persona docente, por lo que en caso de necesitar de un archivo en concreto que le haya enviado la persona docente o viceversa deberá buscarlo en el propio chat y descargarlo desde ahí.
	
	\begin{figure}[H]
		\captionsetup{justification=centering}
		\centering
		\includegraphics[width=.25\linewidth]{imagenes/cap3/3109.jpg}
		\caption{Pantalla Archivos}
		\label{fig:359}
	\end{figure}
	
	\subsubsection{Pantalla Interactividad}
	
	Aquí le aparecen las preguntas que la persona docente envía a los grupos. Si la persona estudiante es portavoz y la actividad es grupal, la opción de contestar se le da solo a esta, mostrando al resto de estudiantes del grupo un mensaje que les invita a debatir la respuesta. Si la actividad es individual, se les dará la opción tanto a portavoces como a no portavoces la opción de contestar. 
	
	\begin{figure}[H]
		\captionsetup{justification=centering}
		\begin{subfigure}{.5\textwidth}
			\centering
			\includegraphics[width=.5\linewidth]{imagenes/cap3/3112.jpg}
			\caption{Formato de respuesta a la actividad en caso de ser individual o de ser portavoz}
		\end{subfigure}%
		\begin{subfigure}{.5\textwidth}
			\centering
			\includegraphics[width=.5\linewidth]{imagenes/cap3/3113.jpg}
			\caption{Formato de mensaje en caso de ser una actividad grupal y no ser el portavoz}
		\end{subfigure}%
		\caption{Vista de las actividades de tipo entrada de texto por los estudiantes}
		\label{fig:360}
	\end{figure}
	
	\begin{figure}[H]
		\captionsetup{justification=centering}
		\begin{subfigure}{.5\textwidth}
			\centering
			\includegraphics[width=.5\linewidth]{imagenes/cap3/3114.jpg}
			\caption{Formato de respuesta a la actividad en caso de ser individual o de ser portavoz}
		\end{subfigure}%
		\begin{subfigure}{.5\textwidth}
			\centering
			\includegraphics[width=.5\linewidth]{imagenes/cap3/3115.jpg}
			\caption{Formato de mensaje en caso de ser una actividad grupal y no ser el portavoz}
		\end{subfigure}%
		\caption{Vista de las actividades de tipo multirrespuesta por los estudiantes}
		\label{fig:361}
	\end{figure}
	
	\chapter{\bfseries Implementación}
	
	Para la implementación del diseño se ha hecho uso de tres tecnologías principales:
	
	\begin{itemize}
	    \item API de REST para la gestión de la base de datos por parte de la persona administradora. Consiste en un cliente desarrollado con Spring Boot y Gradle en el lenguaje de programación Java para inscribir a las personas usuarias en las asignaturas correspondientes por parte de la persona administradora.
	    
	    \item Android Studio, el IDE oficial de Google para la creación de aplicaciones Android. Con este entorno de desarrollo se ha programado la aplicación en el lenguaje de programación Java. La interfaz gráfica se diseña mediante el lenguaje XML.
	    
	    \item Google Firestore, la base de datos de la plataforma Google Firebase. Se utiliza además la base de datos Google Storage de la plataforma Firebase para el almacenamiento de los archivos del chat.
	\end{itemize}
	
	Todo el código o funciones que se mencionan en esta sección se encuentra disponible de forma pública en \url{https://github.com/marmatsan/CoordinApp}. 
    \section{Registro de personas usuarias de la aplicación en el sistema}	
	
	La persona administradora se encarga de todo lo referente a este apartado. Utilizará la API de REST desarrollada para la comunicación directa con la base de datos. Esta API está desarrollada en una subcarpeta que viene dentro de la carpeta del proyecto llamada \texttt{admin\textbackslash coordinapprest}. En dicha carpeta es necesaria la clave privada en formato JSON de la base de datos, llamada \texttt{serviceAccountKey.json}, y que se puede descargar de la propia plataforma Firebase. Esta clave contiene todos los \textit{tokens} y credenciales necesarios para que la comunicación con la base de datos sea posible. La API consiste de varios módulos que han sido diseñados exclusivamente para este proyecto, y que contienen, entre otras cosas, los modelos de datos principales (clases de Java) como el modelo del curso, la asignatura y los modelos que contienen información sobre los usuarios, los cuales se encuentran en la carpeta \texttt{src} de la API.\par
	
	Primero, el administrador debe registrar a las personas usuarias en la plataforma de manera individual, introduciendo su e-mail y su contraseña, la cual no puede ser menor de seis dígitos. Al hacerlo, se otorgará un identificador alfanumérico único de cada persona usuaria registrada. Después, debe crear dos colecciones, una llamada \textit{Teachers} y otra llamada \textit{Students}, donde creará un documento por cada persona usuaria conteniendo su nombre completo y su e-mail. La persona usuaria será docente si se encuentra en la colección \textit{Teachers} y estudiante si se encuentra en la colección \textit{Students}. Una vez hecho esto, debe crear un archivo JSON con la siguiente estructura:
	
\makeatletter
\AtBeginEnvironment{minted}{\dontdofcolorbox}
\def\dontdofcolorbox{\renewcommand\fcolorbox[4][]{##4}}
\makeatother
	
\begin{minted}[fontsize=\small, linenos = false, escapeinside=||, mathescape=true, frame = single,framesep=0pt,framerule = 1pt]{JSON}
{
    "courseName": "Nombre del curso",
    "courseParticipantsIDs": {
        "allParticipantsIDs": [
            "ID Estudiante1",
            "ID Estudiante2",
                ......
            "ID EstudianteN",
            "ID Docente1",
            "ID Docente2",
                ......
            "ID DocenteM",

        ]
    },
    "courseSubjects": [
        {
            "subjectName": "Nombre de la asignatura 1",
            "teacherID": "ID Docente1",
            "studentIDs": [
                "ID EstudianteJ-1",
                "ID EstudianteJ-2",
                       ......
                "ID EstudianteJ-I"
            ]
        },
        {
            "subjectName": "Nombre de la asignatura 2",
            "teacherID": "ID Docente2",
            "studentIDs": [
                "ID EstudianteP-1",
                "ID EstudianteP-2",
                       ......
                "ID EstudianteP-K"
            ]
        },
                       ......
        {
            "subjectName": "Nombre de la asignatura M",
            "teacherID": "ID DocenteM",
            "studentIDs": [
                "ID EstudianteZ-1",
                "ID EstudianteZ-2",
                       ......
                "ID EstudianteZ-W"
            ]
        }
    ]
}
\end{minted}
	
    Es decir, cada docente solo puede impartir una asignatura y las asignaturas pueden tener el conjunto entero de estudiantes o un subconjunto del mismo. Una vez hecho esto, la persona administradora deberá iniciar la API usando el comando \texttt{gradle bootRun} dentro de la carpeta. Es entonces cuando la persona administradora deberá usar un software de terceros para realizar una petición HTTP de tipo POST con cuerpo el archivo JSON creado. Es en ese momento cuando se creará la colección \textit{CoursesOrganization} y se podrá empezar a escribir datos desde la aplicación Android por parte de los usuarios inscritos por la persona administradora.\par
    
    En cuanto a la escritura en la base de datos desde la aplicación Android, todas las peticiones tanto de escritura como de lectura se realizan indicando la colección sobre la que se quiere escuchar o sobre la que se quiere escribir. También se permiten escuchas de documentos concretos en caso de que sus datos cambien.\par 
    
   \noindent Para obtener un documento en concreto:
    
\begin{minted}[fontsize=\small, linenos = false, escapeinside=||, mathescape=true, frame = single,framesep=0pt,framerule = 1pt]{Java}

FirebaseFirestore fStore = FirebaseFirestore.getInstance();

fStore
    .collection("Nombre de la colección")
    .document("Nombre del documento")
    .get()
    .addOnSuccessListener(requestedDocument -> {
         // Obtener los datos del documento y pasarlos a objeto de Java.
         CustomObject object = requestedDocument
                               .toObject(CustomObject.class);
    });
\end{minted}
    
  \noindent  Para obtener todos los documentos de una colección:
    
\begin{minted}[fontsize=\small, linenos = false, escapeinside=||, mathescape=true, frame = single,framesep=0pt,framerule = 1pt]{Java}

FirebaseFirestore fStore = FirebaseFirestore.getInstance();

fStore
    .collection("Nombre de la colección")
    .get()
    .addOnSuccessListener(requestedDocuments -> {
         for (DocumentSnapshot requestedDocument : requestedDocuments) {
            CustomObject object = requestedDocument
                                  .toObject(CustomObject.class);
         }
    });
\end{minted}    

  \noindent  Para escuchar un documento de una colección:

\begin{minted}[fontsize=\small, linenos = false, escapeinside=||, mathescape=true, frame = single,framesep=0pt,framerule = 1pt]{Java}

FirebaseFirestore fStore = FirebaseFirestore.getInstance();

fStore
    .collection("Nombre de la colección")
    .document("Nombre del documento")
    .addSnapshotListener((documentSnapshot, error) -> {
        if (error != null) {
            return;
        } else if (documentSnapshot == null) {
            return;
        }
        CustomObject object = documentSnapshot
                              .toObject(CustomObject.class);
        });
\end{minted}    
    
  \noindent  Para escuchar una colección entera:

\begin{minted}[fontsize=\small, linenos = false, escapeinside=||, mathescape=true, frame = single,framesep=0pt,framerule = 1pt]{Java}

FirebaseFirestore fStore = FirebaseFirestore.getInstance();

fStore
    .collection("Nombre de la colección")
    .addSnapshotListener((documentSnapshots, error) -> {
        if (error != null) {
            return;
        } else if (documentSnapshots == null) {
            return;
        }
        for (DocumentSnapshot documentSnapshot : documentSnapshots) {
            CustomObject object = documentSnapshot
                                  .toObject(CustomObject.class);
         }
        });
\end{minted} 
    
    \section{Diagramas de secuencia UML}
    
    En esta sección se muestran unos diagramas de secuencia UML donde se explican los procesos llevados a cabo para la implementación de las funciones de la aplicación. Toda la comprobación de estas funcionalidades se encuentran en el capítulo \ref{chapter3} de este trabajo. Los diagramas muestran a las diferentes personas usuarias de la aplicación como hilos que tienen una longitud que ocupa todo el alto de la figura y los procesos de código se muestran como hilos discontinuos. Las flechas indican una acción, y los recuadros muestran conjuntos de acciones con una finalidad en común.\vspace{0.2cm}

    \begin{figure}[H]
        \captionsetup{justification=centering}
        \centering
        \begin{sequencediagram}
        	\newthread{esx}{:EstudianteX}
        	\newinst{a}{:CrearPetición}
        	\newinst[1]{b}{:Firestore}
        	\newthreadShift{d}{:Docente}{1cm}
        	\newthread{esy}{:EstudianteY}
        	
        	\begin{sdblock}{Crear petición}{El estudiante X crea la petición}
        			\postlevel
        	\begin{call}{esx}{\shortstack{Seleccionar\\ estudiantes}}{a}{Comprobante}
        		\begin{call}{a}{\shortstack{Crear\\ petición}}{b}{Comprobante}
        		\end{call}
        	\end{call}
        	\end{sdblock}
        
        	\begin{call}{esx}{Lectura}{b}{Respuesta}
        	\end{call}
        
        	\begin{call}{d}{Lectura}{b}{Respuesta}
        	\end{call}
        
        	\begin{sdblock}{Acción de docente}{\shortstack{El docente \\ crea el grupo \\ o elimina la\\ petición}}
        		\postlevel
        		\postlevel
        	\begin{messcall}{d}{\shortstack{Crear grupo \\ o eliminar \\ petición}}{b}{respuesta}
        	\end{messcall}
        	\end{sdblock}
        
        	\begin{call}{esy}{Lectura}{b}{Respuesta}
        	\end{call}
        		
        	\begin{sdblock}{Acción de estudiante Y}{\shortstack{El estudiante Y\\ acepta o rechaza\\ la petición}}
        		\postlevel
        	\begin{messcall}{esy}{\shortstack{Aceptar petición o \\ rechazar petición}}{b}
        	\end{messcall}
        	\end{sdblock}
        \end{sequencediagram}
        \caption{Creación de grupo mediante una petición}
        \label{fig:diagr1}
    \end{figure}

    La figura \ref{fig:diagr1} muestra el funcionamiento del sistema de peticiones. Una persona estudiante puede crear una petición de creación de grupo, la cual será gestionada por el sistema. En esta gestión está comprobar si ya existe una petición igual a la que se está intentando crear, que en caso de que sea así la base de datos devolverá el ID de la petición y se avisará a la persona estudiante de que esto es así, y en caso contrario se creará. La persona estudiante está escuchando en todo momento la colección de Peticiones mencionada en la figura \ref{fig:dbStructure} para que se le puedan mostrar las peticiones tanto propias como de otras personas estudiantes (esta es una funcionalidad que se repite constantemente a lo largo de todo el proyecto, puesto que define la escucha en tiempo real de cambios en las colecciones). Una vez ha recibido la persona docente la petición mediante la escucha continua en la base de datos, esta podrá aceptarla o rechazarla. Si la rechaza, la petición se eliminará de la base de datos, y en caso contrario se creará un grupo con las personas participantes incluidas en la petición. Por otra parte, una persona estudiante que no ha creado la petición puede de igual forma aceptarla o rechazarla, salvo que estas acciones tendrán un efecto puramente estético de cara a que la persona docente pueda comprobar los estatus de las peticiones de las personas estudiantes.

    \begin{figure}[H]
        \captionsetup{justification=centering}
        \centering
        \begin{sequencediagram}
        	\newthread{d}{:Docente}
        	\newinst{a}{:CrearGrupos}
        	\newinst[2]{b}{:Firestore}
        	\newthreadShift{es}{:Estudiante}{2cm}
        	\begin{sdblock}{Crear grupo privado}{\shortstack{Crear un grupo privado\\ con un estudiante}}
        	\postlevel
        	\begin{call}{d}{\shortstack{Seleccionar\\ estudiante}}{a}{\shortstack{Abrir \\ grupo}}
        		\begin{call}{a}{\shortstack{Comprobar \\grupo}}{b}{Comprobante}
        		\end{call}
        		\begin{sdblock}{Si no existe}{}
        		        	\postlevel
        		 \begin{call}{a}{\shortstack{Crear \\grupo}}{a}{}
        		\end{call}
        		\end{sdblock}
        	\end{call}	
        	\end{sdblock}
        	
            \begin{sdblock}{Crear grupo privado}{\shortstack{El estudiante también\\ puede crear el grupo privado}}
            \postlevel
        	\begin{call}{es}{Abrir chat}{a}{Abrir grupo}
        	    \postlevel
    		    \begin{call}{a}{\shortstack{Comprobar \\grupo}}{b}{Comprobante}
        		\end{call}
        		\begin{sdblock}{Si no existe}{}
        		        	\postlevel
        		 \begin{call}{a}{\shortstack{Crear \\grupo}}{a}{}
        		\end{call}
        		\end{sdblock}
        	\end{call}
            \end{sdblock}

        	\begin{sdblock}{Crear un grupo}{\shortstack{Crear un grupo de\\ dos o más estudiantes}}
        		\postlevel
        	\begin{call}{d}{\shortstack{Seleccionar\\ estudiantes}}{a}{Respuesta}
        		\begin{call}{a}{\shortstack{Crear \\grupo}}{b}{Comprobante}
        		\end{call}
        	\end{call}
        	\end{sdblock}
        \end{sequencediagram}
        \caption{Creación de un único grupo}
        \label{fig:diagr2}
    \end{figure}

    La figura \ref{fig:diagr2} muestra los pasos seguidos para la creación de un grupo, una de las funcionalidades más importantes de la aplicación, empezando por la creación de los grupos privados. La persona docente, en caso de que quiera crear un grupo privado con una persona estudiante debe hacerlo explícitamente, seleccionando la acción de creación de grupo. El sistema se encargará de gestionar la creación de este grupo, comprobando en el proceso si el grupo existe. En caso de que exista significa que solo es necesario abrir el grupo (redirigir a la persona docente a la sala de chat). En caso de que no exista, el grupo se creará y de igual forma se redirigirá a la persona docente a la sala de chat de dicho grupo. La persona estudiante realiza el mismo proceso, solo que esta vez lo realiza pulsando sobre la acción de la que dispone en la pantalla Grupos, sin necesidad de un menú como la persona docente. Por otra parte, la persona docente es la única que tiene la potestad de crear un grupo seleccionando a las personas participantes, como muestra el diagrama en su parte inferior, comprobando en el proceso si el grupo que se intenta crear existe, en cuyo caso se devolverá el ID del grupo existente, y en caso contrario se creará con las personas participantes seleccionadas.

    \begin{figure}[H]
        \centering
        \captionsetup{justification=centering}
        \begin{sequencediagram}
        	\newthread{d}{:Docente}
        	\newinst[2]{a}{:CrearGrupos}
        	\newinst[2]{b}{:Firestore}
        	\newthreadShift{es}{:Estudiante}{2cm}
        	
        	\begin{sdblock}{Crear múltiples grupos}{\shortstack{Menú de creación\\ de grupos}}
        	\postlevel
        	\begin{call}{d}{\shortstack{Seleccionar modo\\ e introducir datos}}{a}{respuestas}
        		\begin{call}{a}{\shortstack{Calcular\\resto}}{a}{}
        		\end{call}
        		\postlevel
        		\begin{call}{a}{\shortstack{Comprobar modo\\ y casilla}}{a}{}
        		\end{call}
        		\begin{sdblock}{Loop}{}
        		        		\postlevel

        		\begin{call}{a}{\shortstack{Crear \\grupo}}{a}{}
        		\end{call}
        		\begin{call}{a}{\shortstack{Escribir \\grupo}}{b}{Comprobante}
        		\end{call}
        		\end{sdblock}
        	\end{call}	
        	\end{sdblock}
        
        	\begin{call}{es}{Lectura}{b}{Respuesta}
        	\end{call}
        
        \end{sequencediagram}
        \caption{Creación de múltiples grupos}
        \label{fig:diagr3}
    \end{figure}

    La figura \ref{fig:diagr3} muestra las funcionalidades del menú de creación de grupos de forma automática. La persona docente selecciona el modo e introduce los datos que requiere el modo seleccionado (número de personas estudiantes por grupo en caso de seleccionar el primer modo, número de grupos en caso de seleccionar el segundo modo). En ambos casos se calcula el resto de la división resultante con el objetivo de incluir a las personas estudiantes restantes en un grupo mayor al especificado o no en caso de ser haber seleccionado el primer modo de creación de grupos y la casilla que así lo indica, ya que en segundo modo siempre se incluirán, para conseguir el número de grupos deseado, por lo que funciona como si la casilla estuviese siempre seleccionada. Después, se van creando los grupos uno a uno, comprobando si el grupo existe o no, que en caso de que sea así no se creará y se avisará a la persona docente y en caso de no exista se creará, mostrándose en la pantalla. La persona estudiante recibirá estas actualizaciones de forma inmediata.

    \begin{figure}[H]
        \centering
        \captionsetup{justification=centering}
        \begin{sequencediagram}
        	\newthread{d}{:Docente}
        	\newinst[2]{a}{:AdministrarGrupos}
        	\newinst[2]{b}{:Firestore}
        	\newthreadShift{es}{:Estudiante}{1cm}
        	
        	\begin{sdblock}{Administrar grupos}{\shortstack{Administrador \\ de grupos}}
        	\postlevel
        		\postlevel
        	\begin{call}{d}{\shortstack{Seleccionar grupos\\ e interambiar\\ estudiantes}}{a}{Respuestas}
        		
        	\begin{call}{a}{\shortstack{Actualizar \\listas}}{a}{}
        	\end{call}
                		\postlevel

        	\begin{call}{a}{\shortstack{Actualizar \\grupos}}{a}{}
        	\end{call}
        
        	\begin{call}{a}{\shortstack{Modificar \\grupos}}{b}{Comprobante}
        	\end{call}
        
        
        	\end{call}	
        	\end{sdblock}
        
        	\begin{call}{es}{Lectura}{b}{Respuesta}
        	\end{call}
        
        \end{sequencediagram}
        \caption{Administración de grupos}
        \label{fig:diagr4}
    \end{figure}

    La figura \ref{fig:diagr4} muestra las fases llevadas a cabo para la administración de dos grupos. Primero, la persona docente selecciona dos grupos que quiera modificar, para seleccionar después a las personas estudiantes que quiere mover del primer grupo seleccionado al segundo grupo seleccionado y viceversa. Una vez realizada la acción de actualizar las listas con las personas integrantes de los grupos modificados, se comprobarán errores y si ya existe un grupo como el que se está intentando crear previo a la modificación. En caso de que no haya ningún error, los grupos se actualizarán de forma satisfactoria. La persona estudiante recibe estas actualizaciones de forma inmediata.

    \begin{figure}[H]
        \centering
        \captionsetup{justification=centering}
        \begin{sequencediagram}
        	\newthread{d}{:Docente}
        	\newinst[1.5]{a}{:CrearActividad}
        	\newinst[1]{b}{:Firestore}
        	\newthreadShift{es}{:Estudiante}{1cm}
        	\postlevel
        	
        	\begin{sdblock}{Crear actividad}{El docente crea una actividad}
        	\begin{call}{d}{Crear actividad}{a}{actividad creada}
        		\begin{call}{a}{\shortstack{Seleccionar tipo\\ de actividad}}{a}{}
        		\end{call}
        		\postlevel
        		\begin{call}{a}{\shortstack{Seleccionar\\ modalidad}}{a}{}
        		\end{call}
        		        		\postlevel
        		\begin{call}{a}{\shortstack{Escribir\\ actividad}}{b}{}
        		\end{call}
        	\end{call}
        	\end{sdblock}
        	
        	\begin{sdblock}{Respuesta}{El estudiante contesta la actividad}
        		\postlevel
        		\begin{call}{es}{Lectura}{b}{Respuesta}
        		\end{call}
        		\begin{call}{es}{\shortstack{Respuesta a\\ la actividad}}{es}{}
        		\end{call}
        		\postlevel
        		\begin{call}{es}{\shortstack{Envío de\\ respuesta}}{b}{\shortstack{Desvanecer\\ actividad}}
        			\postlevel
        			\begin{messcall}{b}{Obtener resultados}{d}
        			\end{messcall}
        		\end{call}
        		\begin{call}{d}{Cálculo de estadísticas}{d}{}
        		\end{call}
        	\end{sdblock}
        	
        \end{sequencediagram}
        \caption{Creación y respuesta de una actividad}
        \label{fig:diagr5}
    \end{figure}
    
    La figura \ref{fig:diagr5} muestra las acciones llevadas a cabo para la creación de una actividad. La persona docente tiene que introducir toda la información referente a la actividad, que en caso de tipo entrada de texto es un título y en caso de tipo multirrespuesta es el título, las respuestas y la respuesta correcta si la actividad es evaluable. Una vez creada le llegará a la persona estudiante, que está escuchando la base de datos. Si es una actividad grupal la persona portavoz será la encargada de contestar la actividad, mientras que si es una actividad individual todas las personas estudiantes tendrán que contestar. El sistema generará las estadísticas de la actividad basándose en las calificaciones de las respuestas si la actividad es evaluable, que en el caso de que sea de tipo entrada de texto serán introducidas por el docente y en caso de que sea de tipo entrada de texto serán calculadas por el sistema.

    \begin{figure}[H]
         \captionsetup{justification=centering}
        \centering
      \begin{sequencediagram}
	    \newthread{u1}{:Usuario}
        	\newinst[1.5]{a}{:CrearEvento}
        	\newinst[1]{b}{:Firestore}
        
        	\postlevel
        	
        	\begin{sdblock}{Crear evento}{El usuario crea un evento}
        
        	\begin{call}{u1}{Crear evento}{a}{\shortstack{Actualizar eventos}}
        		\begin{call}{a}{\shortstack{Proceso de datos\\ necesarios}}{a}{}
        		\end{call}
        		\postlevel
        
        		\begin{call}{a}{\shortstack{Escribir\\evento}}{b}{}
        		\end{call}
        	\end{call}
        	\end{sdblock}
        	\postlevel
        	\begin{call}{u1}{Leer eventos}{b}{Actualizar lista de eventos}
        	\end{call}
        	
        \end{sequencediagram}
        \caption{Creación de una actividad de tipo evento}
        \label{fig:diagr6}
    \end{figure}

    La figura \ref{fig:diagr6} muestra la creación de un evento, el cual es una actividad de carácter especial, puesto que sirve como forma indirecta de comunicación. La persona usuaria (ya sea docente o portavoz de un grupo, que son los que tienen la potestad de crear eventos) crea el evento añadiendo toda la información necesaria, que es un título, una descripción, un lugar y una fecha. El evento se creará y se mostrará a todas las personas usuarias a los que ha sido enviado de forma inmediata.
    
    \begin{figure}[H]
        \centering
        \captionsetup{justification=centering}
\begin{sequencediagram}
	\newthread{u1}{:Usuario}
	\newinst[1.5]{a}{:CrearMensaje}
	\newinst[1]{b}{:Firestore}

	\postlevel
	
	\begin{sdblock}{Crear mensaje}{El usuario manda un mensaje}
		\postlevel
		\postlevel
	\begin{call}{u1}{\shortstack{Enviar mensaje \\ y/o adjuntar\\ archivos}}{a}{\shortstack{Recibir mensaje\\ propio}}
		\begin{call}{a}{\shortstack{Cálculo de \\la fecha}}{a}{}
		\end{call}
		\postlevel
		\postlevel
		\begin{call}{a}{\shortstack{Subir archivo a\\ Storage si \\ lo hay}}{a}{}
		\end{call}
		\postlevel
		\begin{call}{a}{\shortstack{Escribir\\mensaje}}{b}{}
		\end{call}
	\end{call}
	\end{sdblock}
	\postlevel
	\begin{call}{u1}{Leer mensajes}{b}{Actualizar lista de mensajes}
	\end{call}
	
    \end{sequencediagram}
        \caption{Escritura de mensajes por chat}
        \label{fig:diagr7}
    \end{figure}

    La figura \ref{fig:diagr7} muestra el envío de mensajes entre personas usuarias por una sala de chat. Cualquier persona usuaria (docente, estudiante o estudiante portavoz) puede realizar las acciones que se muestran en la figura, entre las que se encuentran subir archivos. Las personas usuarias están escuchando a la base de datos en el momento, puesto que recibirán los mensajes enviados por otros usuarios de forma inmediata. Si el mensaje contiene archivos adjuntos, estos se subirán a la base de datos especial llamada \textit{Storage}, proveyendo un link para descargarlos.

	\chapter{\bfseries Evaluación y resultados}
	
	Con el objetivo de obtener una primera valoración de la aplicación se ha realizado una evaluación inicial a una muestra de población de 9 personas de entre 24 y 29 años. Estas personas recibieron una copia de la aplicación y se les pidió que la instalasen en sus dispositivos móviles. 
	
	\section{Metodología seguida}
	Se seleccionó a una de estas personas para que tuviese el rol de docente, mientras que las otras 8 tuvieron el rol de estudiante. Se crearon dos grupos de 4 personas estudiantes cada uno, las cuales probaron todas las funcionalidades de la aplicación. Una vez finalizada la prueba, se les pidió que realizasen un cuestionario.
	
	\section{Formato del cuestionario}
	El cuestionario enviado a las personas participantes tenía carácter anónimo. Se utilizó la escala Likert para ponderar las preguntas en una escala del 1 al 5, donde 1 quería decir \enquote{Totalmente en desacuerdo} y donde 5 quería decir \enquote{Totalmente de acuerdo}. El formulario cuenta con 4 secciones con preguntas adecuadas tanto con la investigación realizada en este Trabajo de Fin de Grado como con el uso de la aplicación:
	
	\begin{itemize}
	    \item \textbf{Sección 1}: Se introduce el formulario a las personas participantes.
	    \item \textbf{Sección 2}: Contiene preguntas sobre la experiencia con el uso de las TIC en el aula dependiendo de si son docentes de profesión o no.
	    \begin{itemize}
	        \item Si son docentes, se les realiza 9 preguntas sobre su relación con las TIC como docentes.
	        \item Si no son docentes, se les realiza 9 preguntas sobre su relación con las TIC como estudiantes.
	    \end{itemize}
	    
	    \item \textbf{Sección 3}: Contiene preguntas sobre el diseño de CoordinApp. Se realizan 7 preguntas más una pregunta de opinión dependiendo de si han tenido el rol de docente o de estudiante durante la prueba.
	    
	    \item \textbf{Sección 4}: Contiene preguntas sobre la experiencia con el uso de CoordinApp. Se realizan 2 preguntas independientemente del rol que han tenido durante la prueba y:
	    
	    \begin{itemize}
	        \item A la persona que ha tenido el rol de docente se le realizan 9 preguntas sobre su experiencia como docente más una pregunta de opinión.
	        \item A las personas que han tenido el rol de estudiantes se les realiza 7 preguntas, y:
	        \begin{itemize}
	            \item Si han tenido el rol de portavoz, dos preguntas más y una pregunta de opinión.
	            \item Si no han tenido el rol de portavoz, dos preguntas más y una pregunta de opinión.
	        \end{itemize}
	    \end{itemize}
	\end{itemize}
	
	\section{Resultados}
	
	\subsection{Experiencia personal del uso de las TIC en el aula}
	
	Se preguntó a las personas participantes si contaban con algún título que les acreditase como docentes y si habían ejercido la profesión:\par
	
	\noindent\fbox{\begin{minipage}{\dimexpr0.993\textwidth-2\fboxsep-2\fboxrule}
    P. \textit{¿Es usted docente/está en disposición de algún título que le acredite para ejercer la profesión y ha impartido clases?} \par\vspace{0.2cm}
    R.
        \begin{figure}[H]
        	\captionsetup{justification=centering}
            \centering
            \begin{tikzpicture}[scale=0.6]
            \pie {
            	44.4/No,
                55.6/Sí
            }
            \end{tikzpicture}
            \caption{Perfil de los participantes de la evaluación}
            \label{fig:my_label}
        \end{figure}
    \end{minipage}
    }
    
    Cabe destacar que el 55.6\% de las personas encuestadas es docente, por lo que han podido comprobar la utilidad de la aplicación a la hora de impartir clase.\par
    
    Si las personas encuestadas eran docentes, se les pidió contestar a las siguientes preguntas, de las cuales se obtuvieron los siguientes resultados:\vspace{0.2cm}
    
	 \noindent \begin{minipage}{\dimexpr0.4965\textwidth-2\fboxsep-2\fboxrule}
    	    \textit{Soy competente en el uso de las TIC} 	
    \end{minipage} \hspace*{0.3cm}
    \begin{minipage}{\dimexpr0.4965\textwidth-2\fboxsep-2\fboxrule}
    	\begin{tabularx}{\linewidth}{|Y|Y|Y|Y|Y|} \hline
    	     1 & 2 & \makecell[c]{3 \\ 20\%} & 4 & \makecell[c]{5 \\ 80\%} \tabularnewline \hline
    	\end{tabularx}
    \end{minipage}
	
    \noindent \begin{minipage}{\dimexpr0.4965\textwidth-2\fboxsep-2\fboxrule}
    	    \textit{El uso de las TIC es muy frecuente en mis clases} 	
    \end{minipage} \hspace*{0.3cm}
    \begin{minipage}{\dimexpr0.4965\textwidth-2\fboxsep-2\fboxrule}
    	\begin{tabularx}{\linewidth}{|Y|Y|Y|Y|Y|} \hline
    	     1 & 2 & \makecell[c]{3 \\ 40\%} & \makecell[c]{4 \\ 20\%} & \makecell[c]{5 \\ 40\%} \tabularnewline \hline
    	\end{tabularx}
    \end{minipage}
		
    \noindent \begin{minipage}{\dimexpr0.4965\textwidth-2\fboxsep-2\fboxrule}
    	    \textit{Las herramientas usadas en mi aula me han sido facilitadas por la administración y no he necesitado del uso de herramientas elegidas de forma personal} 	
    \end{minipage} \hspace*{0.3cm}
    \begin{minipage}{\dimexpr0.4965\textwidth-2\fboxsep-2\fboxrule}
    	\begin{tabularx}{\linewidth}{|Y|Y|Y|Y|Y|} \hline
    	    \makecell[c]{1 \\ 20\%} & \makecell[c]{2 \\ 40\%} & \makecell[c]{3 \\ 20\%} & \makecell[c]{4 \\ 20\%} & 5 \tabularnewline \hline
    	\end{tabularx}
    \end{minipage}
	
			
    \noindent \begin{minipage}{\dimexpr0.4965\textwidth-2\fboxsep-2\fboxrule}
    	    \textit{El uso de las TIC en mi aula fomenta la participación de los alumnos} 	
    \end{minipage} \hspace*{0.3cm}
    \begin{minipage}{\dimexpr0.4965\textwidth-2\fboxsep-2\fboxrule}
    	\begin{tabularx}{\linewidth}{|Y|Y|Y|Y|Y|} \hline
    	     1 & 2 & 3 & \makecell[c]{4 \\ 40\%} & \makecell[c]{5 \\ 60\%} \tabularnewline \hline
    	\end{tabularx}
    \end{minipage}
    
    		
    \noindent \begin{minipage}{\dimexpr0.4965\textwidth-2\fboxsep-2\fboxrule}
    	    \textit{Previo a la pandemia, el uso de las TIC en mi centro era habitual} 	
    \end{minipage} \hspace*{0.3cm}
    \begin{minipage}{\dimexpr0.4965\textwidth-2\fboxsep-2\fboxrule}
    	\begin{tabularx}{\linewidth}{|Y|Y|Y|Y|Y|} \hline
    	     1 & \makecell[c]{2 \\ 20\%} & \makecell[c]{3 \\ 40\%} & \makecell[c]{4 \\ 40\%} & 5 \tabularnewline \hline
    	\end{tabularx}
    \end{minipage}
    
    		
    \noindent \begin{minipage}{\dimexpr0.4965\textwidth-2\fboxsep-2\fboxrule}
    	    \textit{Soy consciente de la dimensión pedagógica de las TIC, más que relegar su uso a gestionar la asignatura} 	
    \end{minipage} \hspace*{0.3cm}
    \begin{minipage}{\dimexpr0.4965\textwidth-2\fboxsep-2\fboxrule}
    	\begin{tabularx}{\linewidth}{|Y|Y|Y|Y|Y|} \hline
    	     1 & 2 & 3 & \makecell[c]{4 \\ 40\%} & \makecell[c]{5 \\ 60\%} \tabularnewline \hline
    	\end{tabularx}
    \end{minipage}
	
    \noindent \begin{minipage}{\dimexpr0.4965\textwidth-2\fboxsep-2\fboxrule}
    	    \textit{La administración no ha escuchado las demandas de los profesores en cuanto al uso de las TIC durante la pandemia} 	
    \end{minipage} \hspace*{0.3cm}
    \begin{minipage}{\dimexpr0.4965\textwidth-2\fboxsep-2\fboxrule}
    	\begin{tabularx}{\linewidth}{|Y|Y|Y|Y|Y|} \hline
    	     1 & \makecell[c]{2 \\ 40\%} & \makecell[c]{3 \\ 40\%} & \makecell[c]{4 \\ 20\%} & 5 \tabularnewline \hline
    	\end{tabularx}
    \end{minipage}	
	
    \noindent \begin{minipage}{\dimexpr0.4965\textwidth-2\fboxsep-2\fboxrule}
    	    \textit{Un modelo educativo híbrido (en el que hubiese que ir 3 días a clase y 2 días desde casa) mejoraría mi calidad de vida} 	
    \end{minipage} \hspace*{0.3cm}
    \begin{minipage}{\dimexpr0.4965\textwidth-2\fboxsep-2\fboxrule}
    	\begin{tabularx}{\linewidth}{|Y|Y|Y|Y|Y|} \hline
    	     1 & \makecell[c]{2 \\ 40\%} & 3 & \makecell[c]{4 \\ 20\%} & \makecell[c]{5 \\ 40\%} \tabularnewline \hline
    	\end{tabularx}
    \end{minipage}		
	
    \noindent \begin{minipage}{\dimexpr0.4965\textwidth-2\fboxsep-2\fboxrule}
    	    \textit{Un modelo educativo híbrido (en el que hubiese que ir 3 días a clase y 2 días desde casa) resultaría en una mejora del aprendizaje de los alumnos} 	
    \end{minipage} \hspace*{0.3cm}
    \begin{minipage}{\dimexpr0.4965\textwidth-2\fboxsep-2\fboxrule}
    	\begin{tabularx}{\linewidth}{|Y|Y|Y|Y|Y|} \hline
    	     \makecell[c]{1 \\ 40\%} & 2 & \makecell[c]{3 \\ 20\%} & \makecell[c]{4 \\ 20\%} & \makecell[c]{5 \\ 20\%} \tabularnewline \hline
    	\end{tabularx}
    \end{minipage}		
    \vspace{0.2cm}
    
	Si los encuestados no eran docentes, se les pidió contestar a las siguientes preguntas, obteniendo los siguientes resultados:\vspace{0.2cm}
	
    \noindent \begin{minipage}{\dimexpr0.4965\textwidth-2\fboxsep-2\fboxrule}
    	    \textit{Soy competente en el uso de las TIC} 	
    \end{minipage} \hspace*{0.3cm}
    \begin{minipage}{\dimexpr0.4965\textwidth-2\fboxsep-2\fboxrule}
    	\begin{tabularx}{\linewidth}{|Y|Y|Y|Y|Y|} \hline
    	     1  & 2 & \makecell[c]{3 \\ 25\%} & \makecell[c]{4 \\ 50\%} & \makecell[c]{5 \\ 25\%} \tabularnewline \hline
    	\end{tabularx}
    \end{minipage}		
    
    \noindent \begin{minipage}{\dimexpr0.4965\textwidth-2\fboxsep-2\fboxrule}
    	    \textit{El uso de las TIC en el aula me motiva a participar en las actividades grupales propuestas por el profesor} 	
    \end{minipage} \hspace*{0.3cm}
    \begin{minipage}{\dimexpr0.4965\textwidth-2\fboxsep-2\fboxrule}
    	\begin{tabularx}{\linewidth}{|Y|Y|Y|Y|Y|} \hline
    	     \makecell[c]{1 \\ 25\%} & \makecell[c]{2 \\ 50\%} & 3 & 4 & \makecell[c]{5 \\ 25\%} \tabularnewline \hline
    	\end{tabularx}
    \end{minipage}
    
    \noindent \begin{minipage}{\dimexpr0.4965\textwidth-2\fboxsep-2\fboxrule}
    	    \textit{Me gusta trabajar en equipo porque me considero un igual entre los integrantes del grupo} 	
    \end{minipage} \hspace*{0.3cm}
    \begin{minipage}{\dimexpr0.4965\textwidth-2\fboxsep-2\fboxrule}
    	\begin{tabularx}{\linewidth}{|Y|Y|Y|Y|Y|} \hline
    	     \makecell[c]{1 \\ 25\%} & \makecell[c]{2 \\ 50\%} & 3 & 4 & \makecell[c]{5 \\ 25\%} \tabularnewline \hline
    	\end{tabularx}
    \end{minipage}		

    \noindent \begin{minipage}{\dimexpr0.4965\textwidth-2\fboxsep-2\fboxrule}
    	    \textit{Me gustaría que el uso de las TIC en el aula fuese mayor} 	
    \end{minipage} \hspace*{0.3cm}
    \begin{minipage}{\dimexpr0.4965\textwidth-2\fboxsep-2\fboxrule}
    	\begin{tabularx}{\linewidth}{|Y|Y|Y|Y|Y|} \hline
    	     \makecell[c]{1 \\ 25\%} & 2 & \makecell[c]{3 \\ 50\%} & 4 & \makecell[c]{5 \\ 25\%} \tabularnewline \hline
    	\end{tabularx}
    \end{minipage}		
    
    \noindent \begin{minipage}{\dimexpr0.4965\textwidth-2\fboxsep-2\fboxrule}
    	    \textit{Cuando trabajo en equipo pienso más en los resultados del equipo que en los míos propios} 	
    \end{minipage} \hspace*{0.3cm}
    \begin{minipage}{\dimexpr0.4965\textwidth-2\fboxsep-2\fboxrule}
    	\begin{tabularx}{\linewidth}{|Y|Y|Y|Y|Y|} \hline
    	     1 & \makecell[c]{2 \\ 25\%} & \makecell[c]{3 \\ 25\%} & \makecell[c]{4 \\ 25\%} & \makecell[c]{5 \\ 25\%} \tabularnewline \hline
    	\end{tabularx}
    \end{minipage}		    
	
    \noindent \begin{minipage}{\dimexpr0.4965\textwidth-2\fboxsep-2\fboxrule}
    	    \textit{Mis profesores utilizan las TIC de una forma que no considero correcta} 	
    \end{minipage} \hspace*{0.3cm}
    \begin{minipage}{\dimexpr0.4965\textwidth-2\fboxsep-2\fboxrule}
    	\begin{tabularx}{\linewidth}{|Y|Y|Y|Y|Y|} \hline
    	     1 & \makecell[c]{2 \\ 25\%} & \makecell[c]{3 \\ 25\%} & \makecell[c]{4 \\ 25\%} & \makecell[c]{5 \\ 25\%} \tabularnewline \hline
    	\end{tabularx}
    \end{minipage}		    
		    
		
    \noindent \begin{minipage}{\dimexpr0.4965\textwidth-2\fboxsep-2\fboxrule}
    	    \textit{Las TIC que usamos en clase están gamificadas, lo cual me ayuda a aprender} 	
    \end{minipage} \hspace*{0.3cm}
    \begin{minipage}{\dimexpr0.4965\textwidth-2\fboxsep-2\fboxrule}
    	\begin{tabularx}{\linewidth}{|Y|Y|Y|Y|Y|} \hline
    	     1 & \makecell[c]{2 \\ 25\%} & \makecell[c]{3 \\ 75\%} & 4 & 5 \tabularnewline \hline
    	\end{tabularx}
    \end{minipage}		    
		
    \noindent \begin{minipage}{\dimexpr0.4965\textwidth-2\fboxsep-2\fboxrule}
    	    \textit{Un modelo educativo híbrido (en el que hubiese que ir 3 días a clase y 2 días desde casa) resultaría en una mejora de mi calidad de vida} 	
    \end{minipage} \hspace*{0.3cm}
    \begin{minipage}{\dimexpr0.4965\textwidth-2\fboxsep-2\fboxrule}
    	\begin{tabularx}{\linewidth}{|Y|Y|Y|Y|Y|} \hline
    	     \makecell[c]{1 \\ 25\%} & \makecell[c]{2 \\ 25\%} & 3 & \makecell[c]{4 \\ 50\%} & 5 \tabularnewline \hline
    	\end{tabularx}
    \end{minipage}		    
		
    \noindent \begin{minipage}{\dimexpr0.4965\textwidth-2\fboxsep-2\fboxrule}
    	    \textit{Un modelo educativo híbrido (en el que hubiese que ir 3 días a clase y 2 días desde casa) resultaría en una mejora de mis resultados académicos} 	
    \end{minipage} \hspace*{0.3cm}
    \begin{minipage}{\dimexpr0.4965\textwidth-2\fboxsep-2\fboxrule}
    	\begin{tabularx}{\linewidth}{|Y|Y|Y|Y|Y|} \hline
    	     \makecell[c]{1 \\ 50\%} & 2 & \makecell[c]{3 \\ 25\%} & \makecell[c]{4 \\ 25\%} & 5\tabularnewline \hline
    	\end{tabularx}
    \end{minipage}		    
    \vspace{0.2cm}
    
    Observando los resultados de las personas que son docentes como profesión podemos observar que la mayoría de personas encuestadas se siente competente en el uso de las TIC, consideran que las TIC fomentan la participación de los estudiantes y son conscientes de la dimensión pedagógica de las mismas. Esto contrasta con el hecho de que no todas usan por igual las TIC en sus clases, pudiendo deberse a los problemas de formación de las personas docentes por parte de la administración expresados en los estudios citados por este trabajo. Hay un consenso general en que el uso de las TIC era habitual antes de la pandemia y una disparidad importante en cuanto a las opiniones sobre el modelo híbrido educativo, donde encontramos extremos en cuanto a las respuestas a las preguntas.\par
    
    En cuanto a los resultados de las personas que no son docentes y que han sido estudiantes podemos distinguir que los resultados son bastante dispares. De forma general se consideran competentes en el uso de las TIC, pero no hay un consenso general respecto a querer un mayor uso de las TIC en el aula. Por otra parte, tampoco hay consenso en que el trabajo en equipo sea deseable por parte de estas personas, desconfiando de las demás personas integrantes del grupo. Todo esto podría explicase por una mala experiencia con las TIC en el aula, y podría explicarse con que los resultados de los trabajos en equipo no siempre han sido favorables para las personas encuestadas. De forma general las TIC con las que tienen experiencia las personas encuestadas no están gamificadas, no teniendo así una buena experiencia con ellas en clase. Al igual que las personas docentes, hay disparidad sobre un modelo educativo híbrido, donde una mitad piensa que es beneficioso para ellas tanto en mejora de calidad de vida como en una mejora de resultados académicos y la otra mitad no.\par
    
    Podríamos concluir, en líneas generales, que las personas docentes se ven capacitadas para usar las TIC y que consideran que es beneficioso usarlas, y que las personas estudiantes se muestran reticentes a trabajar en equipo. Por último, un modelo educativo híbrido causa diversidad de opiniones, en donde generalmente una mitad piensa que tendría efectos positivos sobre su calidad de vida y los resultados académicos y la otra no.
    
	\subsection{Diseño de CoordinApp}
	
	Se preguntó a las personas encuestadas en referencia al diseño de la aplicación, obteniendo los siguientes resultados:
	
    \noindent \begin{minipage}{\dimexpr0.4\textwidth-2\fboxsep-2\fboxrule}
    	    \textit{El diseño de la aplicación me resulta atractivo} 	
    \end{minipage} \hspace*{0.3cm}
    \begin{minipage}{\dimexpr0.5930\textwidth-2\fboxsep-2\fboxrule}
    	\begin{tabularx}{\linewidth}{|Y|Y|Y|Y|Y|} \hline
    	     1 & 2 & \makecell[c]{3 \\ 44.4\%} & \makecell[c]{4 \\ 44.4\%} & \makecell[c]{5 \\ 11.1\%}\tabularnewline \hline
    	\end{tabularx}
    \end{minipage}			
	
    \noindent \begin{minipage}{\dimexpr0.4\textwidth-2\fboxsep-2\fboxrule}
    	    \textit{Las diferentes secciones de la aplicación están bien diferenciadas} 	
    \end{minipage} \hspace*{0.3cm}
    \begin{minipage}{\dimexpr0.5930\textwidth-2\fboxsep-2\fboxrule}
    	\begin{tabularx}{\linewidth}{|Y|Y|Y|Y|Y|} \hline
    	     1 & \makecell[c]{2 \\ 11.1\%} & 3 & \makecell[c]{4 \\ 55.6\%} & \makecell[c]{5 \\ 33.3\%}\tabularnewline \hline
    	\end{tabularx}
    \end{minipage}			
	
    \noindent \begin{minipage}{\dimexpr0.4\textwidth-2\fboxsep-2\fboxrule}
    	    \textit{Me resulta sencillo buscar la información que necesito} 	
    \end{minipage} \hspace*{0.3cm}
    \begin{minipage}{\dimexpr0.5930\textwidth-2\fboxsep-2\fboxrule}
    	\begin{tabularx}{\linewidth}{|Y|Y|Y|Y|Y|} \hline
    	     1 & \makecell[c]{2 \\ 11.1\%} & \makecell[c]{3 \\ 22.2\%} & \makecell[c]{4 \\ 22.2\%} & \makecell[c]{5 \\ 44.4\%}\tabularnewline \hline
    	\end{tabularx}
    \end{minipage}			
	
    \noindent \begin{minipage}{\dimexpr0.4\textwidth-2\fboxsep-2\fboxrule}
    	    \textit{La paleta de colores de la aplicación me transmite calma} 	
    \end{minipage} \hspace*{0.3cm}
    \begin{minipage}{\dimexpr0.5930\textwidth-2\fboxsep-2\fboxrule}
    	\begin{tabularx}{\linewidth}{|Y|Y|Y|Y|Y|} \hline
    	     1 & 2 & 3 & \makecell[c]{4 \\ 66.7\%} & \makecell[c]{5 \\ 33.3\%}\tabularnewline \hline
    	\end{tabularx}
    \end{minipage}			
	
    \noindent \begin{minipage}{\dimexpr0.4\textwidth-2\fboxsep-2\fboxrule}
    	    \textit{El diseño de la aplicación está adecuado a todas las edades} 	
    \end{minipage} \hspace*{0.3cm}
    \begin{minipage}{\dimexpr0.5930\textwidth-2\fboxsep-2\fboxrule}
    	\begin{tabularx}{\linewidth}{|Y|Y|Y|Y|Y|} \hline
    	     1 & \makecell[c]{2 \\ 11.1\%} & 3 & \makecell[c]{4 \\ 44.4\%} & \makecell[c]{5 \\ 44.4\%}\tabularnewline \hline
    	\end{tabularx}
    \end{minipage}			
	
    \noindent \begin{minipage}{\dimexpr0.4\textwidth-2\fboxsep-2\fboxrule}
    	    \textit{El diseño de la aplicación está adecuado a todas las personas independientemente de su familiaridad con la tecnología} 	
    \end{minipage} \hspace*{0.3cm}
    \begin{minipage}{\dimexpr0.5930\textwidth-2\fboxsep-2\fboxrule}
    	\begin{tabularx}{\linewidth}{|Y|Y|Y|Y|Y|} \hline
    	     1 & \makecell[c]{2 \\ 22.2\%} & \makecell[c]{3 \\ 22.2\%} & \makecell[c]{4 \\ 33.3\%} & \makecell[c]{5 \\ 22.2\%}\tabularnewline \hline
    	\end{tabularx}
    \end{minipage}			
	
    \noindent \begin{minipage}{\dimexpr0.4\textwidth-2\fboxsep-2\fboxrule}
    	    \textit{El diseño me parece innovador} 	
    \end{minipage} \hspace*{0.3cm}
    \begin{minipage}{\dimexpr0.5930\textwidth-2\fboxsep-2\fboxrule}
    	\begin{tabularx}{\linewidth}{|Y|Y|Y|Y|Y|} \hline
    	     1 & 2 & \makecell[c]{3 \\ 66.7\%} & \makecell[c]{4 \\ 22.2\%} & \makecell[c]{5 \\ 11.1\%}\tabularnewline \hline
    	\end{tabularx}
    \end{minipage}			
	\vspace{0.2cm}
		
	Por otra parte, se preguntó a las personas encuestadas ciertas preguntas de opinión. A la persona encuestada con rol de docente dentro de la aplicación se le hizo la siguiente pregunta:
    \begin{mdframed}
	\begin{minipage}{\dimexpr0.993\textwidth-2\fboxsep-2\fboxrule}
    P. \textit{En caso de que el diseño de la aplicación necesite de cambios y/o mejoras indícalas y cómo se debería hacer} \par\vspace{0.2cm}
    R.El diseño lo veo bastante bien, siempre hay cosas que pueden ser mejorables, sobretodo dependiendo de la situación para la que el docente necesite dicha aplicación, como por ejemplo un sistema de anotación para explicar a los alumnos la razón de su nota. Pero la verdad es que la aplicación está muy bien y es una gran idea.
    \end{minipage}
    \end{mdframed}
    
    \newpage
    
    A las personas encuestadas con rol de estudiante dentro de la aplicación se les realizó la misma pregunta:
    \begin{mdframed}
	\begin{minipage}{\dimexpr0.993\textwidth-2\fboxsep-2\fboxrule}
    P. \textit{En caso de que el diseño de la aplicación necesite de cambios y/o mejoras indícalas y cómo se debería hacer} \par\vspace{0.2cm}
    R1. Estaría mejor aprovechada si pudieras recibir notificaciones en el móvil respecto a los chats con el profesor, las modificaciones en las tareas o los nuevos archivos subidos. Por lo demás la aplicación está muy bien, especialmente el doble chat diferenciados por destinatarios entre profesor y compañeros.\par\vspace{0.2cm}
    
    R2. El diseño de la interfaz puede ser un poco enrevesado de primeras, sobre todo la selección inicial de asignatura, sería mas eficiente presentar directamente una lista de todas las asignaturas del alumno/profesor como pantalla de inicio y que cada una luego diera paso a sus diferentes pestañas con chats, archivos, etc.\par\vspace{0.2cm}
    
    R3. La organización jerárquica de los submenús desplegables y separados por grupos es algo liosa y un poco saturante si hay muchas cosas dentro. La presentación de los eventos podría ser más visual, por ejemplo usando un calendario marcado.\par\vspace{0.2cm}
    
    R4. Estaría bien tener apartados de cada asignatura tanto para la rama de ciencias como la de letras y artes y dar la oportunidad de ofrecer conocimiento de todas ellas de manera gradual. Es decir, como nivelar el conocimiento de cada asignatura según vas realizando formularios, preguntas de tipo test o de desarrollo.\par\vspace{0.2cm}
    
    R5. He podido apreciar un par de mejoras que me gustaría que se implementaran, como por ejemplo que se indicase mejor el botón para poder hablar directamente con el profesor y que los grupos como tal son algo confusos, habría que diferenciar bien las opciones que dan a parte de una simple pestaña o botón de acceso. Por lo demás, la usaría sin problemas en clase asiduamente.\par\vspace{0.2cm}
    
    R6. Me parece un diseño adecuado para una app de profesor-alumno.
    \end{minipage}
    \end{mdframed}

	Podemos observar que generalmente la aplicación ha resultado de agrado para los encuestados en cuanto a su diseño, sobre todo en cuanto a la paleta de colores y a la diferenciación de las secciones (pantallas) de la aplicación, además de ser adecuada para todas las edades. Hay una mayor disparidad en cuanto a si es apta para las personas con poca familiaridad con la tecnología, y en cuanto a la innovación del diseño no hay una postura claramente definida. Por otra parte, hay varias sugerencias a la hora de mejorar el diseño de la aplicación. Entre las más destacadas están la implementación de un sistema de notificaciones, la reducción de la cantidad de información que se muestra en una pantalla al mismo tiempo y una mejor descripción o más representativa de las funciones de la aplicación.  
	
	\subsection{Experiencia con el uso de CoordinApp}
	
	Se preguntó a las personas encuestadas en referencia a las funcionalidades de la aplicación, obteniendo los siguientes resultados:
	
    \noindent \begin{minipage}{\dimexpr0.4\textwidth-2\fboxsep-2\fboxrule}
    	    \textit{El desarrollo de la aplicación está justificado en el contexto de pandemia que nos encontramos} 	
    \end{minipage} \hspace*{0.3cm}
    \begin{minipage}{\dimexpr0.5930\textwidth-2\fboxsep-2\fboxrule}
    	\begin{tabularx}{\linewidth}{|Y|Y|Y|Y|Y|} \hline
    	     1 & 2 & \makecell[c]{3 \\ 11.1\%} & \makecell[c]{4 \\ 11.1\%} & \makecell[c]{5 \\ 77.8\%}\tabularnewline \hline
    	\end{tabularx}
    \end{minipage}			

    \noindent \begin{minipage}{\dimexpr0.4\textwidth-2\fboxsep-2\fboxrule}
    	    \textit{CoordinApp puede sustituir a un software que conozco con herramientas y funcionalidades parecidas que solo pudiese ser utilizado en un ordenador} 	
    \end{minipage} \hspace*{0.3cm}
    \begin{minipage}{\dimexpr0.5930\textwidth-2\fboxsep-2\fboxrule}
    	\begin{tabularx}{\linewidth}{|Y|Y|Y|Y|Y|} \hline
    	     1 & 2 & \makecell[c]{3 \\ 44.4\%} & \makecell[c]{4 \\ 33.3\%} & \makecell[c]{5 \\ 22.2\%}\tabularnewline \hline
    	\end{tabularx}
    \end{minipage}		
    \vspace{0.2cm}
    
    Para la persona encuestada con rol de docente se le hizo las siguientes preguntas:\vspace{0.2cm}
    
    \noindent \begin{minipage}{\dimexpr0.4\textwidth-2\fboxsep-2\fboxrule}
    	    \textit{La aplicación me permite coordinar alumnos de forma muy adecuada} 	
    \end{minipage} \hspace*{0.3cm}
    \begin{minipage}{\dimexpr0.5930\textwidth-2\fboxsep-2\fboxrule}
    	\begin{tabularx}{\linewidth}{|Y|Y|Y|Y|Y|} \hline
    	     1 & 2 & 3 & 4 & \makecell[c]{5 \\ $\bullet$}\tabularnewline \hline
    	\end{tabularx}
    \end{minipage}		
    
    \noindent \begin{minipage}{\dimexpr0.4\textwidth-2\fboxsep-2\fboxrule}
    	    \textit{Me gustaría tener el control de lo que hablan mis alumnos por el chat de grupo en el que están únicamente ellos} 	
    \end{minipage} \hspace*{0.3cm}
    \begin{minipage}{\dimexpr0.5930\textwidth-2\fboxsep-2\fboxrule}
    	\begin{tabularx}{\linewidth}{|Y|Y|Y|Y|Y|} \hline
    	     1 & 2 & 3 &  \makecell[c]{4 \\ $\bullet$} & 5\tabularnewline \hline
    	\end{tabularx}
    \end{minipage}		
    
    \noindent \begin{minipage}{\dimexpr0.4\textwidth-2\fboxsep-2\fboxrule}
    	    \textit{Las actividades de evaluación y la forma de evaluarlas son muy adecuadas} 	
    \end{minipage} \hspace*{0.3cm}
    \begin{minipage}{\dimexpr0.5930\textwidth-2\fboxsep-2\fboxrule}
    	\begin{tabularx}{\linewidth}{|Y|Y|Y|Y|Y|} \hline
    	     1 & 2 & 3 &  \makecell[c]{4 \\ $\bullet$} & 5\tabularnewline \hline
    	\end{tabularx}
    \end{minipage}		
    
    \noindent \begin{minipage}{\dimexpr0.4\textwidth-2\fboxsep-2\fboxrule}
    	    \textit{Las actividades realizadas en la aplicación pueden sustituir a las actividades realizadas de forma presencial} 	
    \end{minipage} \hspace*{0.3cm}
    \begin{minipage}{\dimexpr0.5930\textwidth-2\fboxsep-2\fboxrule}
    	\begin{tabularx}{\linewidth}{|Y|Y|Y|Y|Y|} \hline
    	     1 & 2 &  \makecell[c]{3 \\ $\bullet$} & 4 & 5\tabularnewline \hline
    	\end{tabularx}
    \end{minipage}		
    
    \noindent \begin{minipage}{\dimexpr0.4\textwidth-2\fboxsep-2\fboxrule}
    	    \textit{Las funcionalidades del gestor de grupos son muy útiles} 	
    \end{minipage} \hspace*{0.3cm}
    \begin{minipage}{\dimexpr0.5930\textwidth-2\fboxsep-2\fboxrule}
    	\begin{tabularx}{\linewidth}{|Y|Y|Y|Y|Y|} \hline
    	     1 & 2 & 3 &  \makecell[c]{4 \\ $\bullet$} & 5\tabularnewline \hline
    	\end{tabularx}
    \end{minipage}		
    
    \noindent \begin{minipage}{\dimexpr0.4\textwidth-2\fboxsep-2\fboxrule}
    	    \textit{Las estadísticas del rendimiento de los alumnos son muy útiles} 	
    \end{minipage} \hspace*{0.3cm}
    \begin{minipage}{\dimexpr0.5930\textwidth-2\fboxsep-2\fboxrule}
    	\begin{tabularx}{\linewidth}{|Y|Y|Y|Y|Y|} \hline
    	     1 & 2 & 3 &  \makecell[c]{4 \\ $\bullet$} & 5\tabularnewline \hline
    	\end{tabularx}
    \end{minipage}		
    
    \noindent \begin{minipage}{\dimexpr0.4\textwidth-2\fboxsep-2\fboxrule}
    	    \textit{Siento que tengo el control de todo lo que sucede en la aplicación} 	
    \end{minipage} \hspace*{0.3cm}
    \begin{minipage}{\dimexpr0.5930\textwidth-2\fboxsep-2\fboxrule}
    	\begin{tabularx}{\linewidth}{|Y|Y|Y|Y|Y|} \hline
    	     1 & 2 & 3 & 4 &  \makecell[c]{5 \\ $\bullet$}\tabularnewline \hline
    	\end{tabularx}
    \end{minipage}		
    
    \noindent \begin{minipage}{\dimexpr0.4\textwidth-2\fboxsep-2\fboxrule}
    	    \textit{Considero que el rol de portavoz discrimina a los otros integrantes del grupo} 	
    \end{minipage} \hspace*{0.3cm}
    \begin{minipage}{\dimexpr0.5930\textwidth-2\fboxsep-2\fboxrule}
    	\begin{tabularx}{\linewidth}{|Y|Y|Y|Y|Y|} \hline
    	      \makecell[c]{1 \\ $\bullet$} & 2 & 3 & 4 & 5\tabularnewline \hline
    	\end{tabularx}
    \end{minipage}		
    
    \noindent \begin{minipage}{\dimexpr0.4\textwidth-2\fboxsep-2\fboxrule}
    	    \textit{Usaría esta aplicación para organizar actividades con mis alumnos} 	
    \end{minipage} \hspace*{0.3cm}
    \begin{minipage}{\dimexpr0.5930\textwidth-2\fboxsep-2\fboxrule}
    	\begin{tabularx}{\linewidth}{|Y|Y|Y|Y|Y|} \hline
    	     1 & 2 & 3 &  \makecell[c]{4 \\ $\bullet$} & 5\tabularnewline \hline
    	\end{tabularx}
    \end{minipage}		
    
    \begin{mdframed}
	\begin{minipage}{\dimexpr0.993\textwidth-2\fboxsep-2\fboxrule}
    P. \textit{En caso de que haya que añadir funcionalidades y/o modificar las existentes indícalas y cómo se debería hacer} \par\vspace{0.2cm}
    R. Para el uso que puedo darle como docente yo no cambiaría nada.
    \end{minipage}
    \end{mdframed}
    
    A las personas encuestadas con rol de estudiante dentro de la aplicación se les realizó las siguientes preguntas:\vspace{0.2cm}
    
    \noindent \begin{minipage}{\dimexpr0.4\textwidth-2\fboxsep-2\fboxrule}
    	    \textit{La aplicación me ha resultado entretenida de usar} 	
    \end{minipage} \hspace*{0.3cm}
    \begin{minipage}{\dimexpr0.5930\textwidth-2\fboxsep-2\fboxrule}
    	\begin{tabularx}{\linewidth}{|Y|Y|Y|Y|Y|} \hline
    	     1 & 2 & \makecell[c]{3 \\ 12.5\%} & \makecell[c]{4 \\ 75\%} & \makecell[c]{5 \\ 12.5\%}\tabularnewline \hline
    	\end{tabularx}
    \end{minipage}		
    
    \noindent \begin{minipage}{\dimexpr0.4\textwidth-2\fboxsep-2\fboxrule}
    	    \textit{Las funciones que me permite realizar la aplicación son intuitivas y fáciles de usar} 	
    \end{minipage} \hspace*{0.3cm}
    \begin{minipage}{\dimexpr0.5930\textwidth-2\fboxsep-2\fboxrule}
    	\begin{tabularx}{\linewidth}{|Y|Y|Y|Y|Y|} \hline
    	     1 & \makecell[c]{2 \\ 12.5\%} & \makecell[c]{3 \\ 12.5\%} & \makecell[c]{4 \\ 25\%} & \makecell[c]{5 \\ 50\%}\tabularnewline \hline
    	\end{tabularx}
    \end{minipage}		
    
    \noindent \begin{minipage}{\dimexpr0.4\textwidth-2\fboxsep-2\fboxrule}
    	    \textit{He sentido que he tenido responsabilidad de los resultados de mi equipo} 	
    \end{minipage} \hspace*{0.3cm}
    \begin{minipage}{\dimexpr0.5930\textwidth-2\fboxsep-2\fboxrule}
    	\begin{tabularx}{\linewidth}{|Y|Y|Y|Y|Y|} \hline
    	     1 & \makecell[c]{2 \\ 25\%} & \makecell[c]{3 \\ 50\%} & 4 & \makecell[c]{5 \\ 25\%}\tabularnewline \hline
    	\end{tabularx}
    \end{minipage}		
    
    \noindent \begin{minipage}{\dimexpr0.4\textwidth-2\fboxsep-2\fboxrule}
    	    \textit{Me he sentido discriminado por los compañeros de mi equipo} 	
    \end{minipage} \hspace*{0.3cm}
    \begin{minipage}{\dimexpr0.5930\textwidth-2\fboxsep-2\fboxrule}
    	\begin{tabularx}{\linewidth}{|Y|Y|Y|Y|Y|} \hline
    	     \makecell[c]{1 \\ 87.5\%} & \makecell[c]{2 \\ 12.5\%} & 3 & 4 & 5\tabularnewline \hline
    	\end{tabularx}
    \end{minipage}		
    
    \noindent \begin{minipage}{\dimexpr0.4\textwidth-2\fboxsep-2\fboxrule}
    	    \textit{El profesor evalúa las respuestas sin saber el nombre del alumno/a que ha contestado ¿Consideras que este es un sistema de evaluación que te beneficia?} 	
    \end{minipage} \hspace*{0.3cm}
    \begin{minipage}{\dimexpr0.5930\textwidth-2\fboxsep-2\fboxrule}
    	\begin{tabularx}{\linewidth}{|Y|Y|Y|Y|Y|} \hline
    	     1 & 2 & 3 & \makecell[c]{4 \\ 37.5\%} & \makecell[c]{5 \\ 62.5\%}\tabularnewline \hline
    	\end{tabularx}
    \end{minipage}		
    
    \noindent \begin{minipage}{\dimexpr0.4\textwidth-2\fboxsep-2\fboxrule}
    	    \textit{En caso de ser necesario, me gustaría ser evaluado utilizando esta aplicación} 	
    \end{minipage} \hspace*{0.3cm}
    \begin{minipage}{\dimexpr0.5930\textwidth-2\fboxsep-2\fboxrule}
    	\begin{tabularx}{\linewidth}{|Y|Y|Y|Y|Y|} \hline
    	     1 & 2 & \makecell[c]{3 \\ 50\%} & \makecell[c]{4 \\ 12.5\%} & \makecell[c]{5 \\ 37.5\%}\tabularnewline \hline
    	\end{tabularx}
    \end{minipage}		
    
    \noindent \begin{minipage}{\dimexpr0.4\textwidth-2\fboxsep-2\fboxrule}
    	    \textit{Me gustaría tener más control sobre cómo se crean los grupos para elegir directamente con quién quiero estar} 	
    \end{minipage} \hspace*{0.3cm}
    \begin{minipage}{\dimexpr0.5930\textwidth-2\fboxsep-2\fboxrule}
    	\begin{tabularx}{\linewidth}{|Y|Y|Y|Y|Y|} \hline
    	     \makecell[c]{1 \\ 12.5\%} & 2 & \makecell[c]{3 \\ 50\%} & \makecell[c]{4 \\ 25\%} & \makecell[c]{5 \\ 12.5\%}\tabularnewline \hline
    	\end{tabularx}
    \end{minipage}		
    \vspace{0.2cm}
    
    Si la persona con el rol de estudiante fue portavoz, debía responder las siguientes preguntas:\vspace{0.2cm}
    
    \noindent \begin{minipage}{\dimexpr0.4\textwidth-2\fboxsep-2\fboxrule}
    	    \textit{Siento que he tenido más responsabilidad de los resultados del grupo que los integrantes del mismo} 	
    \end{minipage} \hspace*{0.3cm}
    \begin{minipage}{\dimexpr0.5930\textwidth-2\fboxsep-2\fboxrule}
    	\begin{tabularx}{\linewidth}{|Y|Y|Y|Y|Y|} \hline
    	     1 & 2 & 3 & 4 & \makecell[c]{5 \\ 100\%}\tabularnewline \hline
    	\end{tabularx}
    \end{minipage}		
    
    \noindent \begin{minipage}{\dimexpr0.4\textwidth-2\fboxsep-2\fboxrule}
    	    \textit{Considero que es necesario este rol} 	
    \end{minipage} \hspace*{0.3cm}
    \begin{minipage}{\dimexpr0.5930\textwidth-2\fboxsep-2\fboxrule}
    	\begin{tabularx}{\linewidth}{|Y|Y|Y|Y|Y|} \hline
    	     1 & 2 & 3 & 4 & \makecell[c]{5 \\ 100\%}\tabularnewline \hline
    	\end{tabularx}
    \end{minipage}		
    
    \begin{mdframed}
	\begin{minipage}{\dimexpr0.993\textwidth-2\fboxsep-2\fboxrule}
    P. \textit{En caso de que haya que añadir funcionalidades y/o modificar las existentes indícalas y cómo se debería hacer} \par\vspace{0.2cm}
    R1. Repetiría la sugerencia de notificaciones.
    \end{minipage}
    \end{mdframed}	
    
    Si la persona con el rol de estudiante no fue portavoz, debía responder las siguientes preguntas\vspace{0.2cm}
    
    \noindent \begin{minipage}{\dimexpr0.4\textwidth-2\fboxsep-2\fboxrule}
    	    \textit{El portavoz ha sido una persona responsable que se ha preocupado de los resultados del equipo} 	
    \end{minipage} \hspace*{0.3cm}
    \begin{minipage}{\dimexpr0.5930\textwidth-2\fboxsep-2\fboxrule}
    	\begin{tabularx}{\linewidth}{|Y|Y|Y|Y|Y|} \hline
    	     1 & 2 & \makecell[c]{3 \\ 33.3\%} & \makecell[c]{4 \\ 33.3\%} & \makecell[c]{5 \\ 33.3\%}\tabularnewline \hline
    	\end{tabularx}
    \end{minipage}		    
	
    \noindent \begin{minipage}{\dimexpr0.4\textwidth-2\fboxsep-2\fboxrule}
    	    \textit{La responsabilidad que se le otorga al portavoz del equipo podría convertirse en una desventaja para mí} 	
    \end{minipage} \hspace*{0.3cm}
    \begin{minipage}{\dimexpr0.5930\textwidth-2\fboxsep-2\fboxrule}
    	\begin{tabularx}{\linewidth}{|Y|Y|Y|Y|Y|} \hline
    	     \makecell[c]{1\\ 16.7\%} & \makecell[c]{2 \\ 16.7\%} & \makecell[c]{3 \\ 16.7\%} & \makecell[c]{4 \\ 50\%} & 5\tabularnewline \hline
    	\end{tabularx}
    \end{minipage}		 
    
    \begin{mdframed}
	\begin{minipage}{\dimexpr0.993\textwidth-2\fboxsep-2\fboxrule}
    P. \textit{En caso de que haya que añadir funcionalidades y/o modificar las existentes indícalas y cómo se debería hacer} \par\vspace{0.2cm}
    R1. Se podría relevar el cargo de portavoz para que cada alumno cargue con la misma responsabilidad e ir adquiriendo conocimientos tanto de la app como de los compañeros con los que se trabaja.
    \end{minipage}
    \end{mdframed}	
    Las personas participantes consideran que el desarrollo de la aplicación está justificado, y por tanto nos acerca al objetivo de este Trabajo de Fin de Grado. Por otra parte, la mayoría de las personas encuestadas considera que la aplicación podría sustituir a otros programas o plataformas existentes, lo cual puede ser significado de que se está siguiendo una buena dirección en cuanto a las funcionalidades implementadas.\par
    
    La experiencia de la persona docente con las funcionalidades de la aplicación ha sido positiva en todos los aspectos, si bien no muestra de forma concisa si las actividades de la aplicación pueden sustituir a las actividades realizadas de forma presencial en un aula. Además, considera que el rol de portavoz no discrimina a las demás personas estudiantes, por lo que esta podría haber sido una acertada elección de diseño.\par
    
    Las personas con el rol de estudiante muestran una satisfacción general con las funcionalidades de la aplicación y con el trabajo realizado en equipo, consideran de forma contundente que el no conocimiento de sus nombres por parte de la persona docente a la hora de evaluar una actividad les beneficia y en lineas generales les gustaría ser evaluados por las actividades de la aplicación. Sin embargo, de forma general les gustaría tener más control sobre con quién se encuentran en el grupo, reafirmando la reticencia de trabajar en equipo que mostraban las respuestas sobre la experiencia con las TIC. Las personas portavoces consideran que han asumido una responsabilidad mayor dentro del grupo, y aun así consideran que el rol es necesario. Esto puede mostrar que las personas participantes consideran que a la hora de trabajar en equipo es necesario que exista una figura con cierto liderazgo para coordinar el grupo. Las personas participantes que no han sido portavoces consideran que, si bien las personas portavoces de sus equipos han sido personas responsables, este rol es algo negativo, mostrando un participante su deseo de que este rol no existiese.
    
    \section{Conclusiones}
     
    Los resultados de la evaluación de CoordinApp indican que en este conjunto de evaluación las personas docentes confían en el uso de las TIC y se sienten preparados para un mayor uso en sus aulas de ellas, mientras que los estudiantes muestran reticencias a la hora de trabajar en equipo, siendo esto aliciente para una mayor investigación sobre el por qué de esta tendencia.\par
    
    Si bien la aplicación resulta sencilla de usar y clara en su estructura para las personas encuestadas, hay margen de mejora en ciertos aspecto de su diseño, siendo la mayor sugerencia una reestructuración de cómo se muestra la información en la aplicación.\par
    
    Por otra parte, los resultados obtenidos han sido positivos para la evaluación de CoordinApp y están alineados con los objetivos de este TFG. Las funcionalidades han sido generalmente del agrado de las personas encuestadas y se considera que el rol de portavoz podría ser algo negativo si no se controla con precaución.
     
	\chapter{\bfseries Gestión del proyecto}
	
	En este capítulo se detalla la planificación diseñada para la realización de este proyecto, así como un presupuesto de la realización del mismo.
	
	\section{Gestión del tiempo}
	
	El ciclo de vida del proyecto consta de tres fases: Planificación, ejecución y cierre.
	
	\begin{itemize}
		\item \textbf{Planificación}\par
		Es la fase inicial del proyecto.
		Consta de dos secciones: inicio y definición. En la primera se analiza el problema sobre el que se va a estudiar y se hace un pequeño estudio de la situación actual del mismo en nuestra sociedad. En la segunda se definen los requisitos funcionales que debe tener la aplicación para dar un intento de solución.
		
		\item \textbf{Ejecución}\par
		Es la fase intermedia del proyecto, en la que más tiempo se encuentra. Consta de tres secciones: diseño, realización y operación. En la primera se realiza un primer diseño con las especificaciones técnicas y la interfaz gráfica inicial para después pasar a las especificaciones técnicas y la interfaz gráfica final. En la segunda se implementa en los emuladores una primera versión de la aplicación con las especificaciones técnicas y la interfaz gráfica iniciales con el objetivo de tener un esqueleto de la aplicación para después implementar una versión final con todas las funcionalidades operativas, con su correspondiente período de corrección de fallos. En la tercera se procede a la prueba de la aplicación en dispositivos \textit{hardware} reales, la corrección de los fallos encontrados en dicha prueba y el lanzamiento de la versión 1.0 de la aplicación.
		
		\item \textbf{Cierre}\par
		Es la fase final del proyecto. Se procede a una evaluación de la aplicación con un grupo seleccionado de personas con el fin de obtener una valoración inicial del proyecto y la redacción de este trabajo.
	\end{itemize}

    \subsection{Definición de actividades}

	Las actividades de las fases tienen dependencia de las actividades de su correspondiente fase. Esto quiere decir que si una actividad tiene dependencia con otra actividad no puede comenzar hasta que no se haya finalizado la actividad de la que depende. Las actividades de fases distintas no tienen dependencia entre sí, puesto que las actividades de una nueva fase no pueden comenzar si no se han completado todas las actividades de una fase anterior. De igual forma, las actividades de una nueva sección de una fase no pueden comenzar si no se han completado las actividades de la sección anterior de dicha fase.\par
	
	Con estas reglas se ha procedido a realizar una tabla de actividades con su duración estimada teniendo en cuenta la fecha límite de entrega del proyecto:
	
	%TODO: Los márgenes de las celdas de actividades no son iguales para todas las celdas
	\begin{figure}[H]
	\centering
		\begin{tabular}{|c|c|c|c|c|c|}\cline{3-6}
			\multicolumn{2}{c|}{}&\textbf{ID} &\textbf{Descripción de la actividad}& \makecell{\textbf{Depende de}} & \textbf{Duración de}\\ \hline
			\multirow{8}{*}[-18ex]{\rotatebox{90}{ \textbf{Planificación}}}
			
			&
			\multirow{3}{*}[-6ex]{\rotatebox{90}{Inicio}}
			& 1.0 & \hspace*{-0.2cm}\makecell[l]{Idea inicial y debate con el\\ tutor sobre las líneas generales\\ del proyecto}& -- &1 semana\\ \cline{3-6}
			
			& & 1.1 & \hspace*{-0.2cm}\makecell[l]{Análisis del uso actual de las\\ TIC en educación y diferentes\\ plataformas} & -- & 4 días\\ \cline{3-6}
			& & 1.2 & \makecell[l]{Análisis de la situación socio-eco-\\nómica actual en España} & -- & 4 días\\ \cline{2-6}
			
			& 
			\multirow{4}{*}[-8ex]{\rotatebox{90}{Definición}}
			& 1.3 & \makecell[l]{Requisitos funcionales básicos ba-\\sándonos en plataformas\\ existentes} & -- & 1 semana \\ \cline{3-6}
			
			&  & 1.4 & \makecell[l]{Requisitos funcionales adiciona-\\les a plataformas existentes} & 1.3 & 1 semana \\ \cline{3-6}
			&  & 1.5 & \makecell[l]{Investigación de diferentes tecno-\\logías y herramientas\\ de desarrollo} & -- & 5 días \\ \cline{3-6}
			&  & 1.6 & \makecell[l]{Elección de las herramientas estu-\\diadas y puesta a punto en\\ el equipo de trabajo} & 1.5 & 2 días \\ \hline
			
			\multirow{6}{*}[-11ex]{\rotatebox{90}{ \textbf{Ejecución}}}
			&
			\multirow{2}{*}[-2ex]{\rotatebox{90}{Diseño}}
			
			& 2.0 & \makecell[l]{Especificaciones técnicas e inter-\\faz gráfica: diseño inicial} & -- & 2 semanas \\ \cline{3-6}
			& & 2.1 & \makecell[l]{Especificaciones técnicas e inter-\\faz gráfica: diseño final} & 2.0 & 4 semanas \\ \cline{2-6}
			&
            \multirow{2}{*}[-4ex]{\rotatebox{90}{Realización}}
			& 2.2 & \makecell[l]{Especificaciones técnicas e inter-\\faz gráfica: \\ implementación inicial} & -- & 10 semanas \\ \cline{3-6}
			&
			& 2.3 & \makecell[l]{Especificaciones técnicas e inter-\\faz gráfica: \\ implementación final} & 2.2 & 16 semanas \\ \cline{3-6}
			& & 2.4 & \makecell[l]{Primera versión: corrección de \phantom{}\\ fallos} & 2.3 & 2 semanas \\ \hline
		\end{tabular}
	\end{figure}
	
	\newpage
	
	\begin{figure}[H]
		\centering
    	\begin{tabular}{|c|c|c|c|c|c|}\cline{3-6}
			\multicolumn{2}{c|}{}&\textbf{ID} &\textbf{Descripción de la actividad}& \makecell{\textbf{Depende de}} & \textbf{Duración de}\\ \hline
			\multirow{3}{*}[-5ex]{\rotatebox{90}{ \textbf{Ejecución}}}
			&
			\multirow{2}{*}[-5ex]{\rotatebox{90}{Operación}}
			& 2.5 & \makecell[l]{Pruebas: Testeo con múltiples dis-\\po-sitivos reales} & -- & 2 días \\ \cline{3-6}
			& & 2.6 & \makecell[l]{Pruebas: Corrección de fallos en-\\contrados} & 2.7 & 2 semanas \\ \cline{3-6}
			& & 2.7 & \makecell[l]{Pequeñas modificaciones. Lanza-\\miento de la versión final de la\\ app} & 2.8 & 5 días \\ \hline
			\multirow{2}{*}{\rotatebox{90}{\textbf{Finalización}}}
			& 
			\multirow{2}{*}[-4ex]{\rotatebox{90}{Cierre}}
			& 3.0 & \makecell[l]{Evaluación de la aplicación por \\ parte de un grupo seleccionado de \\ personas} & -- & 1 semana \\ \cline{3-6}
			
			& & 3.1 & \makecell[l]{\phantom{}Investigación y redacción de la\\ memoria del Trabajo de Fin de \hspace*{0.35cm}\\ Grado} & -- & 4 semanas \\ \hline
		\end{tabular}
		\captionof{table}{Plan de actividades del proyecto}
		\label{tbl:table601}
	\end{figure}
	
	\subsection{Diagramas de Gantt}
	
	Con este plan inicial se ha elaborado el diagrama de Gantt de la figura \ref{fig:figure601} para representar visualmente las fechas límite de las diferentes actividades que componen las fases. Debido a que es un proyecto con una extensión de 12 meses, se ha dividido dicho diagrama en trimestres para una mejor visualización 
	
	\begin{figure}[H]
		\centering
		\begin{tabular}{|c|c|}\hline
			\textbf{P}& Planificación \\
			\textit{P.C} & Planificación-Comienzo\\
			\textbf{P.I} & Planificación-Inicio\\
			\textbf{P.D} & Planificación-Definición\\
			\textit{P.F/E.C} & Planifición-Final/Ejecución-Inicio\\
			\textbf{E} & Ejecución \\
			\textbf{E.D} & Ejecución-Diseño\\
			\textbf{E.R} & Ejecución-Realización\\
			\textbf{E.O} & Ejecución-Operación\\
			\textit{E.F/F.C} & Ejecución-Final/Finalización-Comienzo\\
			\textbf{F} & Finalización\\
			\textbf{F.C} & Finalización-Cierre\\
			\textit{F.F} & Finalización-Final\\ \hline
		\end{tabular}
		\captionof{table}{Resumen de las iniciales del diagrama de Gantt}
		\label{tbl:table602}
	\end{figure}
	
	\noindent No se muestran dependencias de las tareas con uniones debido a que las relaciones entre estas ya han sido definidas en la tabla \ref{tbl:table601}.
	
	\begin{figure}[H]
		\ganttset{%
			calendar week text={%
				\currentweek
			}%
		}
		\centering
		\hspace{-1.5cm}\begin{ganttchart}[
			hgrid,
			vgrid,
			x unit=1.6mm,
			y unit title = 0.7cm,
			y unit chart = 0.6cm,
			time slot format=isodate,
			time slot unit=day      
			]{2020-10-01}{2020-12-31}
			\gantttitlecalendar{year, month=name, week=1} \\
			\ganttgroup{P}{2020-10-11}{2020-10-31} \\
			\ganttmilestone{P.C}{2020-10-10}\\
			\ganttgroup{P.I}{2020-10-11}{2020-10-17} \\
			\ganttbar[bar/.append style={fill=blue}]{1.0}{2020-10-11}{2020-10-17}\ganttnewline
			\ganttbar[bar/.append style={fill=blue}]{1.1}{2020-10-13}{2020-10-16}\ganttnewline
			\ganttbar[bar/.append style={fill=blue}]{1.2}{2020-10-13}{2020-10-16}\ganttnewline
			
			\ganttgroup{P.D}{2020-10-18}{2020-10-31} \\
			\ganttbar[bar/.append style={fill=blue}]{1.3}{2020-10-18}{2020-10-24}\ganttnewline
			\ganttbar[bar/.append style={fill=blue}]{1.4}{2020-10-25}{2020-10-31}\ganttnewline
			\ganttbar[bar/.append style={fill=blue}]{1.5}{2020-10-20}{2020-10-24}\ganttnewline
			\ganttbar[bar/.append style={fill=blue}]{1.6}{2020-10-25}{2020-10-26}\ganttnewline
			\ganttset{%
				group right peak height = 0    
			}%
			\ganttgroup{E}{2020-11-01}{2020-12-31} \\
			\ganttmilestone{P.F/E.C}{2020-10-31}\\
			\ganttset{%
				group right peak height = 0.1    
			}%
			\ganttgroup{E.D}{2020-11-01}{2020-12-13} \\
			\ganttbar[bar/.append style={fill=red}]{2.0}{2020-11-1}{2020-11-14}\ganttnewline
			\ganttbar[bar/.append style={fill=red}]{2.1}{2020-11-15}{2020-12-13}\ganttnewline
			\ganttset{%
				group right peak height = 0    
			}%
			\ganttgroup{E.R}{2020-12-14}{2020-12-31} \\    
			\ganttbar[bar/.append style={fill=red}]{2.2}{2020-12-14}{2020-12-31}\ganttnewline
		\end{ganttchart}  
	\end{figure}
	
	\begin{figure}[H]
		\ganttset{%
			calendar week text={%
				\currentweek
			}%
		}
		\centering
		\hspace{-0.8cm}\begin{ganttchart}[
			hgrid,
			vgrid,
			x unit=1.6mm,
			y unit title = 0.7cm,
			y unit chart = 0.6cm,
			time slot format=isodate,
			time slot unit=day      
			]{2021-01-01}{2021-03-31}
			\gantttitlecalendar{year, month=name, week=15} \\
			\ganttset{%
				group right peak height = 0,
			    group left peak height = 0
			}%
			\ganttgroup{E}{2021-01-01}{2021-03-31} \\   
			\ganttgroup{E.R}{2021-01-01}{2021-03-31} \\    
			\ganttbar[bar/.append style={fill=red}]{2.2}{2021-01-01}{2021-02-21}\ganttnewline
			\ganttbar[bar/.append style={fill=red}]{2.3}{2021-02-22}{2021-03-31}\ganttnewline
		\end{ganttchart}  
	\end{figure}
	
	\newpage
	
	\begin{figure}[H]
		\ganttset{%
			calendar week text={%
				\currentweek
			}%
		}
		\centering
		\hspace{-0.85cm}\begin{ganttchart}[
			hgrid,
			vgrid,
			x unit=1.6mm,
			y unit title = 0.7cm,
			y unit chart = 0.6cm,
			time slot format=isodate,
			time slot unit=day      
			]{2021-04-01}{2021-06-30}
			\gantttitlecalendar{year, month=name, week=29} \\
			\ganttset{%
				group right peak height = 0,
			    group left peak height = 0
			}%
			\ganttgroup{E}{2021-04-01}{2021-06-30} \\
			\ganttset{%
				group right peak height = 0.1,
			    group left peak height = 0
			}%
			\ganttgroup{E.R}{2021-04-01}{2021-06-27} \\
			\ganttbar[bar/.append style={fill=red}]{2.3}{2021-04-01}{2021-06-13}\ganttnewline
			\ganttbar[bar/.append style={fill=red}]{2.4}{2021-06-14}{2021-06-27}\ganttnewline
			\ganttset{%
				group right peak height = 0,
				group left peak height = 0.1,
			}%
			\ganttgroup{E.O}{2021-06-28}{2021-06-30} \\
			\ganttbar[bar/.append
			style={fill=red}]{2.5}{2021-06-28}{2021-06-29}\ganttnewline
			\ganttbar[bar/.append
			style={fill=red}]{2.6}{2021-06-30}{2021-06-30}\ganttnewline
		\end{ganttchart}  
	\end{figure}
	
	\begin{figure}[H]
		\centering
		\captionsetup{justification=centering}
		\ganttset{%
			calendar week text={%
				\currentweek
			}%
		}
		\hspace{-1.55cm}\begin{ganttchart}[
			hgrid,
			vgrid,
			x unit=1.6mm,
			y unit title = 0.7cm,
			y unit chart = 0.6cm,
			time slot format=isodate,
			time slot unit=day      
			]{2021-07-01}{2021-09-30}
			\gantttitlecalendar{year, month=name, week=43} \\
			\ganttset{%
				group right peak height = 0.1,
			    group left peak height = 0
			}%
			\ganttgroup{E}{2021-07-01}{2021-07-18} \\
			\ganttgroup{E.O}{2021-07-01}{2021-07-18} \\
			\ganttbar[bar/.append style={fill=red}]{2.6}{2021-07-01}{2021-07-13}\ganttnewline
			\ganttbar[bar/.append style={fill=red}]{2.7}{2021-07-14}{2021-07-18}\ganttnewline
			\ganttset{%
				group right peak height = 0.1,
			    group left peak height = 0.1
			}%
			\ganttgroup{F}{2021-07-19}{2021-08-14} \\
			\ganttmilestone{E.F/F.C}{2021-07-18}\\
			\ganttgroup{F.C}{2021-07-19}{2021-08-14} \\
			\ganttbar[bar/.append style={fill=orange}]{3.0}{2021-07-19}{2021-07-24}\ganttnewline
			\ganttbar[bar/.append style={fill=orange}]{3.1}{2021-07-19}{2021-08-14}\ganttnewline
			\ganttmilestone{F.F}{2021-08-14}\\
			
		\end{ganttchart}  
		\captionof{figure}{Diagrama de Gantt con la planificación inicial del proyecto}
		\label{fig:figure601}
	\end{figure}
	
	Según el diagrama, el proyecto tendría una duración aproximada de once meses, con inicio el 11 de octubre de 2020 y finalización el 14 de agosto de 2021.\par
	A lo largo de la realización de todo el proyecto se ha medido el cumplimiento de los intervalos de tiempo. Con los resultados recogidos se volvió a realizar un diagrama de Gantt real del tiempo invertido en las actividades del proyecto, recogidos en la figura \ref{fig:figure602}.
	
	\newpage
	
	\begin{figure}[H]
		\ganttset{%
			calendar week text={%
				\currentweek
			}%
		}
		\centering
		\hspace{-1.5cm}\begin{ganttchart}[
			hgrid,
			vgrid,
			x unit=1.6mm,
			y unit title = 0.7cm,
			y unit chart = 0.6cm,
			time slot format=isodate,
			time slot unit=day      
			]{2020-10-01}{2020-12-31}
			\gantttitlecalendar{year, month=name, week=1} \\
			\ganttgroup{P}{2020-10-11}{2020-10-31} \\
			\ganttmilestone{P.C}{2020-10-10}\\
			\ganttgroup{P.I}{2020-10-11}{2020-10-17} \\
			\ganttbar[bar/.append style={fill=blue}]{1.0}{2020-10-11}{2020-10-17}\ganttnewline
			\ganttbar[bar/.append style={fill=blue}]{1.1}{2020-10-13}{2020-10-16}\ganttnewline
			\ganttbar[bar/.append style={fill=blue}]{1.2}{2020-10-13}{2020-10-16}\ganttnewline
			
			\ganttgroup{P.D}{2020-10-18}{2020-10-31} \\
			\ganttbar[bar/.append style={fill=blue}]{1.3}{2020-10-18}{2020-10-24}\ganttnewline
			\ganttbar[bar/.append style={fill=blue}]{1.4}{2020-10-25}{2020-10-31}\ganttnewline
			\ganttbar[bar/.append style={fill=blue}]{1.5}{2020-10-20}{2020-10-24}\ganttnewline
			\ganttbar[bar/.append style={fill=blue}]{1.6}{2020-10-25}{2020-10-26}\ganttnewline
			\ganttset{%
				group right peak height = 0    
			}%
			\ganttgroup{E}{2020-11-01}{2020-12-31} \\
			\ganttmilestone{P.F/E.C}{2020-10-31}\\
			\ganttgroup{E.D}{2020-11-01}{2020-12-31} \\
			\ganttbar[bar/.append style={fill=red}]{2.0}{2020-11-1}{2020-11-25}\ganttnewline
			\ganttbar[bar/.append style={fill=red}]{2.1}{2020-11-26}{2020-12-31}\ganttnewline
		\end{ganttchart}  
	\end{figure}
	
	\begin{figure}[H]
		\ganttset{%
			calendar week text={%
				\currentweek
			}%
		}
		\centering
		\hspace{-0.8cm}\begin{ganttchart}[
			hgrid,
			vgrid,
			x unit=1.6mm,
			y unit title = 0.7cm,
			y unit chart = 0.6cm,
			time slot format=isodate,
			time slot unit=day      
			]{2021-01-01}{2021-03-31}
			\gantttitlecalendar{year, month=name, week=15} \\
			\ganttset{%
				group right peak height = 0,
			    group left peak height = 0
			}%
			\ganttgroup{E}{2021-01-01}{2021-03-31} \\
			\ganttset{%
				group right peak height = 0.1,
			    group left peak height = 0
			}%
			\ganttgroup{E.D}{2021-01-01}{2021-02-25} \\
			\ganttbar[bar/.append style={fill=red}]{2.1}{2021-01-01}{2021-02-25}\ganttnewline
			\ganttset{%
				group right peak height = 0,
			    group left peak height = 0.1
			}%
			\ganttgroup{E.R}{2021-02-26}{2021-03-31} \\    
			\ganttbar[bar/.append style={fill=red}]{2.2}{2021-02-26}{2021-03-31}\ganttnewline
		\end{ganttchart}  
	\end{figure}
	
	\newpage
	
	\begin{figure}[H]
		\ganttset{%
			calendar week text={%
				\currentweek
			}%
		}
		\centering
		\hspace{-0.85cm}\begin{ganttchart}[
			hgrid,
			vgrid,
			x unit=1.6mm,
			y unit title = 0.7cm,
			y unit chart = 0.6cm,
			time slot format=isodate,
			time slot unit=day      
			]{2021-04-01}{2021-06-30}
			\gantttitlecalendar{year, month=name, week=29} \\
			\ganttset{%
				group right peak height = 0,
			    group left peak height = 0
			}%
			\ganttgroup{E}{2021-04-01}{2021-06-30} \\
			\ganttgroup{E.R}{2021-04-01}{2021-06-30} \\
			\ganttbar[bar/.append style={fill=red}]{2.2}{2021-04-01}{2021-05-25}\ganttnewline
			\ganttbar[bar/.append style={fill=red}]{2.3}{2021-05-26}{2021-06-30}\ganttnewline
		\end{ganttchart}  
	\end{figure}
	
	\begin{figure}[H]
		\centering
		\captionsetup{justification=centering}
		\ganttset{%
			calendar week text={%
				\currentweek
			}%
		}
		\hspace{-1.55cm}\begin{ganttchart}[
			hgrid,
			vgrid,
			x unit=1.6mm,
			y unit title = 0.7cm,
			y unit chart = 0.6cm,
			time slot format=isodate,
			time slot unit=day      
			]{2021-07-01}{2021-09-30}
			\gantttitlecalendar{year, month=name, week=43} \\
			\ganttset{%
			    group left peak height = 0
			}%
			\ganttgroup{E}{2021-07-01}{2021-08-15} \\
			\ganttgroup{E.R}{2021-07-01}{2021-08-06} \\
			\ganttset{%
			    group right peak height = 0.1,
			    group left peak height = 0.1
			}%
			\ganttbar[bar/.append style={fill=red}]{2.3}{2021-07-01}{2021-07-14}\ganttnewline
			\ganttbar[bar/.append style={fill=red}]{2.4}{2021-07-15}{2021-08-06}\ganttnewline
			\ganttgroup{E.O}{2021-08-07}{2021-08-15} \\
			\ganttbar[bar/.append style={fill=red}]{2.5}{2021-08-07}{2021-08-08}\ganttnewline
			\ganttbar[bar/.append style={fill=red}]{2.6}{2021-08-09}{2021-08-12}\ganttnewline
			\ganttbar[bar/.append style={fill=red}]{2.7}{2021-08-13}{2021-08-15}\ganttnewline
			
			\ganttgroup{F}{2021-08-16}{2021-09-06} \\
			\ganttmilestone{E.F/F.C}{2021-08-15}\\
			\ganttgroup{F.C}{2021-08-16}{2021-09-06} \\
			\ganttbar[bar/.append style={fill=orange}]{3.0}{2021-08-16}{2021-08-20}\ganttnewline
			\ganttbar[bar/.append style={fill=orange}]{3.1}{2021-08-16}{2021-09-06}\ganttnewline
			\ganttmilestone{F.F}{2021-09-06}
		\end{ganttchart}  
		\captionof{figure}{Diagrama de Gantt con la planificación final del proyecto}
		\label{fig:figure602}
	\end{figure}
	
	\subsection{Análisis de las planificaciones inicial y final}
	
	Se ha realizado un diagrama de barras comparando el tiempo requerido para cada fase con el tiempo real invertido en ellas con el fin de analizar la precisión de la planificación inicial:
	
	\begin{figure}[H]
		\centering
		\captionsetup{justification=centering}
		
		\begin{tikzpicture}
			\begin{axis} [
				xbar = .05cm,
				ybar = .05cm,
				bar width = 12pt,
				ymin = 0, ymax = 45,
				xmin = 0, xmax = 3,
				enlarge x limits = {value = .25, upper},
				enlarge y limits = {abs = .8},
				ylabel= Semanas,
				xlabel = Fases,
				xtick={1,2,3},
				ytick = {0, 10, 20, 30, 40},
				yticklabels = {0, 10, 20, 30, 40},
				xticklabels = {Fase 1, Fase 2, Fase 3},
				x label style = {at={(axis description cs:0.5,-0.05)},anchor=north},
				legend pos = outer north east
				]
				%TODO: cambiar
				\addplot coordinates {(1,5.14) (2,37) (3, 5)};
				\addplot coordinates {(1, 5.14) (2, 42.14) (3, 3.14)};
				
				\legend { Planificación inicial, Planificación final};
				
			\end{axis}
		\end{tikzpicture}
		\captionof{figure}{Comparación de la planificación inicial con la planificación final}
		\label{fig:figure603}
	\end{figure}
	
	La planificación inicial tenía una duración total de 43 semanas y 6 días. Sin embargo, la duración real del proyecto ha sido de 47 semanas y 1 día. Esto quiere decir que se ha tardado 3 semanas y 2 días más de lo planeado. Aun así, se ha cumplido en mayor medida con los plazos establecidos, habiendo una reducción de tiempo considerable en la fase 3 del proyecto de 2 semanas y un día para compensar los retrasos sufridos en la fase 2 del proyecto.\par
	
	Por lo tanto, el proyecto ha tenido una duración total de 330 días. Debido a que el tiempo dedicado a la realización del proyecto no ha sido proporcional todos los días, se ha estimado una media de 1 hora por día, lo que nos da un total de 330 horas de proyecto.
	
	\section{Presupuesto de la elaboración del TFG}
	
	Debido a que este Trabajo de Fin de Grado aborda un proyecto práctico y teniendo en cuenta el total de horas dedicadas a la realización de este proyecto se ha estimado un presupuesto del desarrollo del mismo por parte de un equipo de la compañía CoordinApp Project Inc. con cuatro perfiles profesionales distintos. El proyecto se llevará a cabo con metodologías ágiles, cada vez más demandadas en el mercado \cite{ref601}. Los perfiles profesionales elegidos son: Una persona Scrum Master encargada de debuggear el código, una persona desarrolladora backend para el desarrollo del código fuente y los modelos de la base de datos, una persona desarrolladora frontend para el desarrollo de la interfaz de usuario y una persona responsable de comunidad de internet o \enquote{community manager} para dar a conocer la aplicación en redes sociales y que potenciales clientes opten por ella descartando otras opciones. Por otra parte, se tienen en cuenta los costes materiales y costes indirectos.\par
	
	La parte de desarrollo la llevarán a cabo la persona Scrum Master, la persona desarrolladora backend y la persona desarrolladora frontend, mientras que la parte de publicidad será tarea de la persona community manager.
	
	\subsection{Coste de personal}
	
	Se ha realizado una investigación y se han obtenido los salarios brutos anuales de los perfiles profesionales necesarios para la realización de este proyecto, junto con las horas totales asignadas a cada empleado para completar el proyecto. Nótese que la persona community manager no participa en el desarrollo de la aplicación.
	
	\begin{figure}[H]
		\centering
		\captionsetup{justification=centering}
		\begin{tabular}{|c|c|c|c|}
			\hline
			\textbf{Identificador} & \textbf{Perfil profesional} & \textbf{Salario bruto anual (€)} & \textbf{Horas de trabajo}\\ \hline
			S.M & Scrum master & 37.186  & 90 \\
			D.B & Desarrollador backend & 34.744 & 140\\
			D.F & Desarrollador frontend & 29.774 & 100\\
			C.M & Community manager & 23.404 & 30 \\ \hline
		\end{tabular}
		\captionof{table}{Retribución bruta de los empleados}
		\label{tbl:table603}
	\end{figure}
	
	Con dichos datos se ha calculado el coste total que las personas trabajadoras supondrían para la empresa:
	
	\begin{figure}[H]
		\begin{tabular}{|c|c|c|c|c|}\cline{2-5}
			\multicolumn{1}{c}{} & \multicolumn{4}{|c|}{\textbf{Perfil profesional}}\\ \cline{2-5}
			\multicolumn{1}{c|}{} & S.M & D.B & D.F & C.M \\ \cline{2-5}
			\multicolumn{1}{c|}{} & \multicolumn{4}{|c|}{\textbf{Salario bruto mensual (€)}}\\ \cline{2-5}
			\multicolumn{1}{c|}{} & 3098.33 & 2895.33 & 2481.16 & 1950.33 \\  \cline{2-5}
			\multicolumn{1}{c|}{} & \multicolumn{4}{|c|}{\textbf{Deducciones (€)}} \\ \hline
			Contingencias comunes (4.7\%) & 145.62 & 136.08 & 116.61 & 91.66 \\
			Formación profesional (0.1\%)  & 3.1 & 2.9 & 2.48 & 1.95\\
			Desempleo (1.55\%)  & 48 & 44.87 & 38.45 & 30.23\\
			Tributación I.R.P.F (14.5\%)  & 449.25 & 419.82 & 359.76 & 282.79\\ \hline
			\multicolumn{1}{c|}{} & \multicolumn{4}{|c|}{\textbf{Seguros sociales (€)}} \\ \hline
			Contingencias comunes (23.6\%) & 731.2 & 683.29 & 585.55 & 460.27 \\
			Contingencias profesionales (1.5\%)  & 46.47 & 43.43 & 37.21 & 29.25\\
			Formación (0.6\%)  & 18.58 & 17.37 & 14.88  & 11.7\\
			FOGASA (0.2\%)  & 6.19 & 5.79 & 4.96 & 3.9\\
			Desempleo (6.7\%)  & 207.58 & 193.98 & 166.23 & 130.67\\ \hline
			\multicolumn{1}{c|}{}  & \multicolumn{4}{|c|}{\textbf{Coste mensual de cada empleado (€)}}\\ \cline{2-5}
			\multicolumn{1}{c|}{}  & 3462.38 & 3235.52 & 2772.69 & 2179.49 \\ \cline{2-5}
			\multicolumn{1}{c|}{}  & \multicolumn{4}{|c|}{\textbf{Horas laborales mensuales}}\\ \cline{2-5}
			\multicolumn{1}{c|}{} & 176 & 176 & 176 & 80 \\ \cline{2-5}
		\end{tabular}
	\end{figure}
	
	\begin{figure}[H]
	    \centering
		\begin{tabular}{|c|c|c|c|c|}\cline{2-5}
			\multicolumn{1}{c|}{} & \multicolumn{4}{|c|}{\textbf{Coste de empleado (€) por hora}} \\ \cline{2-5}
			\multicolumn{1}{c|}{}  & 19.67 & 18.38 & 15.75 & 27.24 \\ \cline{2-5}
			\multicolumn{1}{c|}{} & \multicolumn{4}{|c|}{\textbf{Coste total de cada empleado (€)}} \\ \cline{2-5}
			\multicolumn{1}{c|}{} & 1770.3 & 2573.2 & 1575 & 817.2 \\ \cline{2-5}
			\multicolumn{1}{c|}{} & \multicolumn{4}{|c|}{\textbf{Coste total de personal (€)}} \\ \cline{2-5}
			\multicolumn{1}{c|}{} & \multicolumn{4}{|c|}{6135.7} \\ \cline{2-5}
		\end{tabular}
		\captionof{table}{Coste total de los empleados para la empresa}
		\label{tbl:table604}
	\end{figure}
	
	\subsection{Coste de material}
	
	Debido a que el bien que produce la realización de este proyecto no es material sino digital no se necesitan materias primas para su realización.\par
	
	El coste de los equipos sobre los que se desarrollará la aplicación viene determinado por las especificaciones técnicas necesarias tanto de Android Studio \cite{ref114} como de Adobe XD para el desarrollador frontend \cite{ref602}. El software más exigente con las especificaciones es Android Studio, por lo que se compran tres equipos de torre (O.T1) que superen las especificaciones mínimas de este software y un equipo menos costoso para el community manager (O.T2). Por otra parte, se necesitan periféricos, entre los que se encuentran pantallas (P) y teclados más ratones (T.R). Al ser los recursos software gratuitos de usar, su coste no se tiene en cuenta.
	
	\begin{figure}[H]
		\centering
		\begin{tabular}{|c|c|c|c|c|} \hline
			\textbf{ID} & \textbf{Modelo} & \textbf{Precio unidad (€)} & \textbf{Unidades} & \textbf{Total (€)}\\ \hline
			O.T1 & PcCom Basic Elite Pro & 499,98 & 3 & 1499.94  \\
			O.T2 & PcCom Basic Home & 212,98 & 1 & 212.98 \\ \hline
			T & Lenovo L24e 30 238 & 199.98 & 4 & 799.92\\
			T.R & Logitech MK295 Combo & 22.89 & 4 & 91.56\\ \hline
			\multicolumn{5}{|c|}{\textbf{Coste total de material (€)}} \\\hline
			\multicolumn{5}{|c|}{2604.4} \\\hline
		\end{tabular}
		\captionof{table}{Coste de material del proyecto}
		\label{tbl:table605}
	\end{figure}
	
	\subsection{Costes indirectos}
	
	La pandemia ha demostrado que la mayoría de trabajos tecnológicos se pueden realizar desde casa, por lo que la modalidad de trabajo de la empresa será a distancia. Según la Ley 10/2021, que indica en su art.11: \enquote{Las personas que trabajan a distancia tendrán derecho a la dotación y mantenimiento adecuado por parte de la empresa  de todos los medios, equipos y herramientas necesarios para el desarrollo de la actividad [...]} \footnote{Ley 10/2021, de 9 de julio, de trabajo a distancia (BOE núm. 164, de 10 de julio de 2021)}. Es por eso que la empresa se encargará de subsanar el coste de la factura de la luz correspondiente a las horas de trabajo del empleado. Teniendo en cuenta el consumo medio mensual de un hogar en España es de 291 kWH (9.38 kWH por día, 0.39 kWH por hora) \cite{ref604} y el precio por kWH estándar \cite{ref1102} tenemos un precio de 0.01833 euros por hora de gasto de electricidad, que da como resultado:
	
	\begin{figure}[H]
		\centering
		\begin{tabular}{|c|c|} \hline
			\textbf{Horas totales de cada empleado} & \textbf{Precio de la luz por cada empleado (€)} \\ \hline
			90 &  1.67 \\
			140 & 2.56 \\
			100 & 1.83 \\
			30 & 0.54\\ \hline
			\multicolumn{2}{|c|}{\textbf{Coste total de gastos indirectos (€)}} \\\hline
			\multicolumn{2}{|c|}{6.6} \\\hline
		\end{tabular}
		\captionof{table}{Gastos indirectos del proyecto}
		\label{tbl:table606}
	\end{figure}
	
	\subsection{Coste total}
	
	\begin{figure}[H]
		\centering
		\begin{tabular}{|c|c|} \hline
			\textbf{Tipo de gastos} & \textbf{Coste calculado (€)} \\ \hline
			Personal &  6135.7 \\
			Materiales & 2604.4 \\
			Indirectos & 6.6 \\ \hline
			\multicolumn{2}{|c|}{\textbf{Coste total (€)}} \\\hline
			\multicolumn{2}{|c|}{8746.7} \\\hline
		\end{tabular}
		\captionof{table}{Gastos totales del proyecto}
		\label{tbl:table607}
	\end{figure}
	
	El presupuesto total del proyecto asciende a OCHO MIL SETECIENTOS CUARENTA Y SEIS CON SIETE EUROS (8746.7 €).
	
	\chapter{\bfseries Conclusiones y líneas futuras}
	
	\section{Conclusiones}
	
	La pandemia del COVID-19 ha supuesto un cambio sin precedentes en nuestra forma de vivir y de relacionarnos, así como graves consecuencias para nuestra sociedad, reflejadas en las millones de muertes producidas por la enfermedad alrededor del mundo, el aumento de las desigualdades sociales y económicas y la privación de un derecho fundamental como es el de la educación, recogido así en el art. 26 de la Declaración Universal de Derechos Humanos. La educación es la única forma de garantizar la plena libertad y autonomía de una persona, y en caso de interrumpirse (como en situaciones de pandemia) se tiene que dotar a los y las estudiantes de medios para que puedan continuar con su formación. Con todo ello, la pandemia también ha traído aspectos positivos, como un aumento en la innovación y en la búsqueda de soluciones a problemas y retos que no hubiesen podido surgir en otro contexto, como ha sido el caso de la educación a distancia. Los ingenieros y las ingenieras deben tener el compromiso de buscar soluciones a los problemas de la sociedad aunque ello no se traduzca en un beneficio económico inmediato, puesto que son quienes tienen el potencial de dar forma a las ideas que pueden traducirse en una mayor igualdad entre las personas. Es por esto que este Trabajo de Fin de Grado ha intentado buscar una solución a la falta de medios tecnológicos de una parte de la población teniendo en cuenta los recursos ya disponibles por esta y no buscar lucro por medio de la solución liberando el código y permitiendo que cualquier persona tenga derecho al proyecto. Aun así, la dotación de herramientas de comunicación para la enseñanza solo es una parte de la solución al problema, puesto que el uso de estas herramientas deben ir acompañadas de otras metodologías como lo es el aprendizaje colaborativo y su correcta inclusión en los Entornos Virtuales de Aprendizaje. La sociedad educativa muestra un creciente interés por el uso de estas tecnologías, consciente de que son una realidad y que resultan beneficiosas para el aprendizaje si se usan de una forma correcta, por lo que piden más medios y formación con ellas, aunque no siempre son escuchados por los órganos de poder del Estado. \par
	
	La realización de este Trabajo de Fin de Grado ha servido para conocer de una mejor forma el funcionamiento actual del sistema educativo en España y conocer las inquietudes y las peticiones de los y las docentes, y tener un mejor conocimiento de dónde se encuentran parte de los problemas y retos a los que se enfrenta la educación, pero también ha servido para saber que hay margen de mejora y que con las herramientas adecuadas se puede reducir una brecha que solo las familias con menos renta conocen. También ha servido para obtener un mayor conocimiento en cuanto a cómo realizar documentos con una estructura y unas normas definidas, lo cual es necesario a la hora de planificar y documentar cualquier proyecto de ingeniería.\par
	
    Se puede concluir de los resultados obtenidos de la realización de este Trabajo de Fin de Grado que crear una aplicación para dispositivos móviles que permitiese una educación en un modelo híbrido podría ser sostenible, además que es posible desarrollar una aplicación en un período relativamente corto de tiempo, siendo conscientes de que la planificación inicial de un proyecto no siempre será la planificación llevada a cabo finalmente debido a retos inesperados que puedan surgir.\par

	Se concluye con la idea de que un modelo educativo que combinase ambos sistemas, tanto presencial como a distancia como sería un modelo híbrido podría resultar en una mejora de ciertos aspectos de las vida de las personas. Por otra parte, la sociedad debe estar preparada para situaciones similares que ocurran en un futuro \cite{ref701}. Si se quiere que los y las estudiantes, sobre todo menores de edad, sigan teniendo acceso a la educación en estas situaciones de crisis, se debe contar con un sistema de educación preparado y con un plan de acción definido en todos sus aspectos, y en las situaciones en que no se pueden reunir muchas personas como es en un aula de forma presencial la educación a distancia o un modelo híbrido parecen las únicas alternativas viables.
	
	\section{Líneas futuras}
	
	Con la investigación realizada y los datos recogidos por el cuestionario realizado a las personas participantes se pueden definir líneas de actuación futuras para este proyecto:
	
	\begin{itemize}
	    \item Actualizar el diseño de la aplicación para que se muestre menos información de una sola vez en la pantalla atendiendo a las peticiones de las personas participantes en la evaluación de la aplicación. Realizar un diseño completamente funcional es costoso, especialmente para los perfiles profesionales que centran sus habilidades en escribir código. Es por ello que hay perfiles especializados en esto como los desarrolladores frontend.
	    
	    \item Implementar un sistema de notificaciones para que la persona usuaria pueda saber cuándo le han escrito o si tiene que reaccionar a alguna actividad, atendiendo a las peticiones de las personas participantes en la evaluación de la aplicación. Esta decisión de diseño de no implementar un sistema de notificaciones estaba planeada desde un principio, ya que se quería dar uso a la aplicación solo en las horas lectivas sin posibilidad de molestar a las personas usuarias en otras partes del día. Sin embargo, se ha comprobado que las personas usuarias tienen que saber cuándo les han hablado o si les queda actividades pendientes por contestar. De esta forma, la comunicación entre las diferentes personas usuarios de la aplicación será más fluida. 
	    
	    \item Permitir que más de una persona docente pueda dar una asignatura, pudiendo así utilizarse de una forma más general en las etapas universitarias que es donde las asignaturas suelen tener más de un docente. Al principio la aplicación se desarrolló de esta forma pensando en las etapas de primaria, secundaria y bachillerato, por lo que implementar esta funcionalidad en una etapa media del proyecto podría haberse traducido en un fallo en las otras funcionalidades. 
	    
	    \item Añadir actividades cruzadas entre grupos, permitiendo la colaboración entre ellos. Los grupos de la aplicación no interactúan entre ellos, y si bien el trabajo en equipo puede tener beneficios para los estudiantes como ya se habló en el Estado del Arte (aprendizaje colaborativo), es lógico que las interacciones entre estudiantes de distintos grupos también resultan beneficiosas. No se pensó en esta funcionalidad hasta las etapas finales del desarrollo de la aplicación, por lo que no fue implementada.
	    
	    \item Bloquear el chat individual entre estudiantes en caso de que la actividad sea evaluable. Las personas estudiantes pueden hablar entre ellas si están realizando una actividad evaluable individual, pudiendo resultar en la copia de resultados o respuestas, y aunque la persona docente puede comprobar si las personas estudiantes están hablando entre ellas, si tiene un número elevado de grupos podría no tener todo el control sobre ellos. No se pensó en esta funcionalidad hasta las etapas finales del desarrollo de la aplicación, por lo que no fue implementada.
	    
	    \item Diferenciar de una mejor forma la función que realiza cada acción. Las ilustraciones de las acciones pueden resultar confusas, no siempre reflejando la función que realizan, como se ha indicado por parte de las personas encuestadas. No se desarrolló un sistema más representativo visualmente debido a que este fallo en el diseño no se supo hasta finalizada la evaluación en forma de encuesta, y hubiese sido contraproducente cambiar el diseño de la aplicación cuando ya se había obtenido una valoración de este en la encuesta.
	    
	    \item Implementar un sistema de registro de las personas usuarias más accesible, para después subir la aplicación a Google Play y que cualquier centro educativo pueda usar CoordinApp como aplicación oficial de comunicación del centro. Esto no se implementó debido a que se prefirió mejorar las funcionalidades que ofrecía la aplicación, quedando esta funcionalidad en un segundo plano hasta el final del proyecto, donde se realizaron modificaciones mínimas en este aspecto pero no suficientes como para implementar un sistema nuevo sin poner en peligro la estabilidad del proyecto.
	    
	    \item Crear una versión de la aplicación en formato web, ya que aunque la aplicación está diseñada para ser usada en dispositivos móviles en forma de aplicación no se debe cerrar la puerta a que CoordinApp esté disponible en este formato, pudiendo llegar a dispositivos más variados. No se desarrolló porque no era objetivo de la realización de este Trabajo de Fin de Grado.
	    
	    \item Implementar un sistema de videollamadas para que la persona docente pueda impartir sus clases, pudiendo así facilitar una comunicación mucho más fluida y natural entre docente y estudiantes. No se desarrolló porque no era objetivo de la realización de este Trabajo de Fin de Grado.
	\end{itemize}
	
	\newpage
	
	%----------
	%	BIBLIOGRAFÍA
	%----------	
	
	\begin{comment}
		\nocite{*} % Si quieres que aparezcan en la bibliografía todos los documentos que la componen (también los que no estén citados en el texto) descomenta está línea
	\end{comment}
	
    \clearpage
	\renewcommand{\bibname}{\bfseries Lista de referencias}
    \addcontentsline{toc}{chapter}{Lista de referencias}
	\printbibliography

	
	%----------
	%	ANEXOS
	%----------	
	
	\appendix
	\pagenumbering{gobble} % Las páginas de los anexos no se numeran
	\addcontentsline{toc}{chapter}{Anexo A: Summary in English}
 	\chapter*{\bfseries Anexo A: Summary in English}
	\section*{1. Introduction}
	The main purpose of this document is to study the consequences that COVID-19 had in education and expose the plan to develop a Virtual Learning Environment called CoordinApp. 
	
	\subsection*{Context}
	The COVID-19 pandemic changed the way we live in all senses, and one of the sectors affected was education, where concern rised from 5.2 points to 11.8 points the months previous and after the pandemic started. There was a lack of electronic devices among Spanish families, as well as a lack of preparation for this situation. In addition, the existence of a common educational platform that could be used from mobile devices was required.
	
	\subsection*{Motivation}
	The motivation for carrying out this work arises during the COVID-19 pandemic due to the growing concern for the education of students in our country during the pandemic. Information and communication technologies (ICT) increase the desire to participate in class activities, despite the fact that society sees them as something negative, thus contradicting studies that say the opposite thing. Students must be taught to use these technologies in a responsible way.
	
	\subsection*{Objectives}
	The main objective of this document is to define the development plan and characteristics of an application named CoordinApp, and it is expected to be used as a practical case study for the implementation of a unified educational platform. The objectives to achieve this are supported by laws such as LO 3/2020, R.D 126/2014 and R.D 1105/2014.
	
	\subsection*{Laws applied}
	The developed application will be distributed first in Spain and then in the member countries of the European Union. It must be registered as an object of intellectual property (R.D 1/1996) but with a GNU license. Data protection law must be taken into account because the application makes use of personal data taking into account European and Californian legislation (GDPR and CCPA). Illustrations used by this project are free to use as long as their designer is mentioned.
	
	\subsection*{Socio-economic impact}
	The socio-economic impact of the implementation of this work has been measured, specifically its ecological, economic and social footprint. It has been verified that the project has a positive balance in all these dimensions (CO2 emissions, savings derived from the implementation of this project or impact on the lives of the people affected by it). Although there are some risks, mainly economic and social, the project has obtained a qualification of \enquote{very sustainable}.
	
	\subsection*{Document structure}
	The document has seven chapters. In the second, the state of the art is analyzed, in the third, the design of the application is analyzed, in the fourth, the implementation of the application, in the fifth, the results of a survey on the application are discussed, in the sixth, it is analyzed the planning followed and the budget for this work and the seventh talks about conclusions and future lines of the project.
	
	\section*{2. State of the art}
    
    \subsection*{COVID-19 pandemic}
	COVID-19 is one more in the family of coronaviruses already known together with SARS-CoV-2 and MERS-CoV, being the ones that cause the most serious diseases. The first two have a very high degree of mortality, being COVID-19 the one with the lowest mortality rate. Even so, it has affected millions of people, resulting in much more deaths. In Spain, the first case was detected on January 31, which led to a situation of confinement and leaving students without going to school. Progress has been made in vaccination until it is in phase 3, the last of the vaccination phases.
	
	\subsection*{Impact of the pandemic on education}
	94\% of students around the world were affected by this pandemic situation. The response to this situation was different depending on the income of the countries, showing inequalities, where 80 to 85\% of studentes in richer countries could continue with their studies, while counties with people with lower income could only attend classes in 50\% of the cases. Different ways of continuing with the classes were chosen (television, radio, etc.). In Latin America and the Caribbean, the situation was remarkably unequal, and this is stated in the CEPAL report. In the case of Spain, the problems were reflected in the different socio-economic situation of Spanish families, although it was decided to return to the classrooms for the 2020-2021 academic year in the form of a mixed model.
	
	\subsection*{Technology and education}
	Technologies encourage participation in the classroom and facilitate teachers' work. However, these technologies can't be an end in themselves, not only serving as a tool to organize courses. To use them correctly it has to be known what collaborative learning is and how to include it in these technologies.
	
	\subsection*{Collaborative Learning}
    Collaborative Learning dictates guidelines that group members have to follow, and if they are followed these methodology can be beneficial for students. There is also Computer Supported Collaborative Learning or CSCL, where technologies play a key role when applying this methodology. There exists platforms to help implementing this methodology in class, such as Classroomscreen, Padlet and Stormboard.
    
    \subsection*{M-learning}
    M-learning consists of the use of mobile devices to educate. In the context of the COVID-19 pandemic, the use of this methodology is considered a solution to the problems caused in education, since students that have it difficult to go to class can learn using these devices.
    
    \subsection*{Virtual Learning Environments}
    Virtual Learning Environments must make use of collaborative learning based on design guidelines. They are platforms where this collaborative learning can occur, as long as it is done correctly and as long as there exists tools within these platforms to control the bahavior of the students when working as a team. UNESCO recommends a series of platforms due to the COVID-19 situation, including Edmodo, ClassDojo and Moodle, the latter being the most used in Spain.
    
    \subsection*{Used technologies}
    The tools used to carry out this work have been selected based on certain criteria, such as choosing a widely used operating system, being endorsed by the developers of the chosen mobile operating system or that the programming language is documented. This is why Android was selected as the operating system, Java as the programming language or Google Firebase as the platform to store the data, since all of them are supported by Google, the owner of the operating system Android.
    
    \section*{3. Design}
    For the development of the application, its logo has been designed and the necessary roles for its implementation have been defined, such as the role of administrator, teacher and student. These last can use the application, where they have the same sections within it, although they perform different functions. The users can only use the application if they have been registered in the system by the administrator.
    
    \subsection*{Teacher role}
    The teacher can create groups efficiently and quickly by choosing their spokespersons and manage them by changing students from one group to another, talk to students through chat groups, send files and send them activities, both of the type of input from text as a questionnaire type, which have different modalities depending on whether you want to evaluate the groups or just know their opinion on certain topics. These activities allow the teacher to obtain statistics of the groups to compare their performance, as well as create activity events. The application has been designed for the teacher to have all control over the class, with the exception of not knowing what the students are talking about through their chat.
    
    \subsection*{Student role}
    Students can ask the teacher to create a group, talk with the teacher privately or among students in the groups to which they belong, respond to the activities and send events to their respective groups if they are spokespersons of their group.

    \subsection*{Design of database structure}
    The database has been designed taking into account how the application's functionalities are designed. The Firebase database has two main types of structures: documents and sets of documents called collections. Two collections called Students and Teachers are defined, where the information of the users is located, and a collection called CoursesOrganization, where the information of the subject is located, such as subjects, chat messages, activities sent by the teacher, etc. The files are kept in a separate database called Storage.
    
    \subsection*{Design requirements}
    Requirements that the application must have, both functional and non-functional, have been defined. The functional requirements can be of the teacher, student or common of both. These requirements have been designed taking into account the functionalities that the different roles can perform. The functional ones define everything that the teacher and the students can do, and the non-functional ones define design guidelines that the application must comply with. There is a total of 34 functional requirements and 4 non- functional requirements to be implemented.
    
    \section*{4. Implementation}
    For the development of the application, a REST API is used by the administrator to register users in the system using the private key of the database and a file in JSON format. Also, to write to the database from the application we have to connect to it using a Java object provided by Firebase, which allow us to listen in realtime to changes in the database. UML sequence diagrams have also been defined that explain the operation of the requests, the creation of events, groups by the teacher, their administration, the creation of activities, how the response to these works and how the chat works between the members of a group.
    
    \section*{5. Evaluation and results}
    
    A questionnaire was sent to a group of 9 people between the ages of 24 and 29 to test the application, most of whom were teachers. One person had the role of teacher while the others had the role of students. The questionnaire had three sections with different questions. The first section asked about the experience with information and communication technologies in the classroom. Teachers think they are trained to use them, while students, although they like working with these technologies, do not show much interest in working in groups. The second section asks about the design, concluding that the design of the application was correct, although the participants have indicated ways to improve certain aspects of it. The third section asks about the functionalities of the application. The person with the role of teacher showed his approval, while the rest, although they consider the correct functionalities, again showed that they do not like to work in a team.
    
    \section*{6. Project management}
    
    \subsection*{Time management}
    A time plan was designed to carry out the project. Preferences for the different activities were established, along its relations, and an initial Gantt diagram was made, which was adapted until the actual planning of the project was achieved. A comparison betweeen these two showed that the project took longer than planned.
    
    \subsection*{Budget}
    For the actual realization of the project by a company a budget was calculated, which included personnel, material and indirect costs. The total budget calculated was 8,746.7 euros.
    
    \section*{7. Conclusions and future lines}
    
    The pandemic has meant an unprecedented change in our way of life, and education has been one of the sectors on which the pandemic has had the most impact. Education is one of the most important stages of a person's life, as it will determine their future. The COVID-19 pandemic has made it clear that socio-economic differences make it impossible for the person to study, and therefore it is the duty of engineers to provide students with educational tools. This work has given the writer a better knowledge of how edication in Spain works and to write documents in a better way, and the result of this work has proved that an app can be developed in a relative short period of time. Among the future lines of the project we have: an improvement of the design and existing functionalities of the application and an implementation of new functionalities, such as bringing the platform to the web and distributing it to the different centers, even implementing video calls so that it can teach classes in a more functional way.
    
    \newpage
    
    \addcontentsline{toc}{chapter}{Anexo B: Cuestionario de evaluación inicial de CoordinApp}
 	\chapter*{\bfseries Anexo B: Cuestionario de evaluación inicial de CoordinApp}
    
    \section*{Sección 1: Introducción}
    
    \noindent Tiempo estimado: 15 minutos

    \noindent Este formulario se realiza con al intención de obtener una valoración inicial de la aplicación CoordinApp, proyecto de Trabajo de Fin de Grado de Martín Mateos Sánchez, estudiante de Ingeniería en Tecnologías de Telecomunicación de la UC3M.\par
    
    \noindent Contexto sobre el cuestionario:\par
    
    \noindent La idea de CoordinApp surge durante la pandemia causada por el nuevo coronavirus SARS-CoV-2. La situación de confinamiento supuso un cambio sin precedentes en todos los ámbitos de nuestra vida tanto laboral como social, afectando a todos los sectores de nuestra sociedad, entre ellos el de la educación.\par
    
    \noindent Los estudios muestran que las carencias de formación en las tecnologías tanto en alumnos como en profesores, sumado a la brecha socio-económica que sufren las familias españolas y utilizar las TIC como un fin en sí mismo ignorando la dimensión didáctica y pedagógica de las mismas ha supuesto un reto para la implementación de un sistema de formación completamente on-line. Por otra parte, la falta de escucha de las demandas de los profesores ha dado como resultado una falta de respuesta común por parte del Ministerio de Eduación, siendo en mayor parte  de forma individual los propios centros educativos los responsables de buscar herramientas para mitigar el impacto en la formación de los estudiantes.\par
    
    \noindent Este Trabajo de Fin de Grado tiene como objetivo atajar la brecha socio-económica disponiendo de los medios necesarios a todos los estudiantes independientemente de su situación económica aprovechando los recursos ya disponibles de los alumnos. El intento de solución propuesto es crear una aplicación para dispositivos móviles llamada CoordinApp, la cual de denomina como un Entorno Virtual de Aprendizaje o EVA. De esta forma, la gran mayoría de estudiantes podrían optar a disponer  de una plataforma por la que comunicarse con alumnos y profesores y desempeñar actividades educativas sin el uso de dispositivos más costosos como un ordenador o una tablet, ya que prácticamente la totalidad de los mayores de 16 años dispone de un dispositivo móvil y en el rango de edades de 10 a 14 años la no disponibilidad de estos dispositivos es del 37\%, pudiendo ser una solución optar a estos estudiantes con un dispositivo móvil por parte del Ministerio, con un coste mucho menor que los planes actuales de digitalización de los centros.\par
    
    \noindent La evaluación consta de cuatro secciones. En esta primera se introduce el cuestionario, en la segunda se pregunta por la experiencia personal del uso de las TIC en el ámbito educativo dependiendo del perfil del encuestado, en la tercera se pregunta en relación al diseño de la app y en la cuarta se pregunta por la experiencia personal del uso de la aplicación dependiendo del rol que ha desempeñado.\par
    
    \noindent Debido a que la presente encuesta recopila datos de forma anónima sin posibilidad de identificar a los participantes mediante su análisis el RGDP vigente de 26 de abril de 2016 no se aplica.
    
    \section*{Sección 2: Experiencia personal del uso de las TIC en el aula}
  
    \textit{¿Es usted docente/está en disposición de algún título que le acredite para ejercer la profesión y ha impartido clases?} \par\vspace{0.2cm}

\noindent    \textbf{Si la respuesta es Sí:}

	 \noindent \begin{minipage}{\dimexpr0.4965\textwidth-2\fboxsep-2\fboxrule}
    	    \textit{Soy competente en el uso de las TIC} 	
    \end{minipage} \hspace*{0.3cm}
    \begin{minipage}{\dimexpr0.4965\textwidth-2\fboxsep-2\fboxrule}
    	\begin{tabularx}{\linewidth}{|Y|Y|Y|Y|Y|} \hline
    	     1 & 2 & 3 & 4 & 5  \tabularnewline \hline
    	\end{tabularx}
    \end{minipage}
	
    \noindent \begin{minipage}{\dimexpr0.4965\textwidth-2\fboxsep-2\fboxrule}
    	    \textit{El uso de las TIC es muy frecuente en mis clases} 	
    \end{minipage} \hspace*{0.3cm}
    \begin{minipage}{\dimexpr0.4965\textwidth-2\fboxsep-2\fboxrule}
     	\begin{tabularx}{\linewidth}{|Y|Y|Y|Y|Y|} \hline
    	     1 & 2 & 3 & 4 & 5  \tabularnewline \hline
    	\end{tabularx}
    \end{minipage}
		
    \noindent \begin{minipage}{\dimexpr0.4965\textwidth-2\fboxsep-2\fboxrule}
    	    \textit{Las herramientas usadas en mi aula me han sido facilitadas por la administración y no he necesitado del uso de herramientas elegidas de forma personal} 	
    \end{minipage} \hspace*{0.3cm}
    \begin{minipage}{\dimexpr0.4965\textwidth-2\fboxsep-2\fboxrule}
     	\begin{tabularx}{\linewidth}{|Y|Y|Y|Y|Y|} \hline
    	     1 & 2 & 3 & 4 & 5  \tabularnewline \hline
    	\end{tabularx}
    \end{minipage}
	
			
    \noindent \begin{minipage}{\dimexpr0.4965\textwidth-2\fboxsep-2\fboxrule}
    	    \textit{El uso de las TIC en mi aula fomenta la participación de los alumnos} 	
    \end{minipage} \hspace*{0.3cm}
    \begin{minipage}{\dimexpr0.4965\textwidth-2\fboxsep-2\fboxrule}
     	\begin{tabularx}{\linewidth}{|Y|Y|Y|Y|Y|} \hline
    	     1 & 2 & 3 & 4 & 5  \tabularnewline \hline
    	\end{tabularx}
    \end{minipage}
    
    		
    \noindent \begin{minipage}{\dimexpr0.4965\textwidth-2\fboxsep-2\fboxrule}
    	    \textit{Previo a la pandemia, el uso de las TIC en mi centro era habitual} 	
    \end{minipage} \hspace*{0.3cm}
    \begin{minipage}{\dimexpr0.4965\textwidth-2\fboxsep-2\fboxrule}
     	\begin{tabularx}{\linewidth}{|Y|Y|Y|Y|Y|} \hline
    	     1 & 2 & 3 & 4 & 5  \tabularnewline \hline
    	\end{tabularx}
    \end{minipage}
    
    		
    \noindent \begin{minipage}{\dimexpr0.4965\textwidth-2\fboxsep-2\fboxrule}
    	    \textit{Soy consciente de la dimensión pedagógica de las TIC, más que relegar su uso a gestionar la asignatura} 	
    \end{minipage} \hspace*{0.3cm}
    \begin{minipage}{\dimexpr0.4965\textwidth-2\fboxsep-2\fboxrule}
    	\begin{tabularx}{\linewidth}{|Y|Y|Y|Y|Y|} \hline
    	     1 & 2 & 3 & 4 & 5  \tabularnewline \hline
    	\end{tabularx}
    \end{minipage}
	
    \noindent \begin{minipage}{\dimexpr0.4965\textwidth-2\fboxsep-2\fboxrule}
    	    \textit{La administración no ha escuchado las demandas de los profesores en cuanto al uso de las TIC durante la pandemia} 	
    \end{minipage} \hspace*{0.3cm}
    \begin{minipage}{\dimexpr0.4965\textwidth-2\fboxsep-2\fboxrule}
     	\begin{tabularx}{\linewidth}{|Y|Y|Y|Y|Y|} \hline
    	     1 & 2 & 3 & 4 & 5  \tabularnewline \hline
    	\end{tabularx}
    \end{minipage}	
	
    \noindent \begin{minipage}{\dimexpr0.4965\textwidth-2\fboxsep-2\fboxrule}
    	    \textit{Un modelo educativo híbrido (en el que hubiese que ir 3 días a clase y 2 días desde casa) mejoraría mi calidad de vida} 	
    \end{minipage} \hspace*{0.3cm}
    \begin{minipage}{\dimexpr0.4965\textwidth-2\fboxsep-2\fboxrule}
    	\begin{tabularx}{\linewidth}{|Y|Y|Y|Y|Y|} \hline
    	     1 & 2 & 3 & 4 & 5  \tabularnewline \hline
    	\end{tabularx}
    \end{minipage}		
	
    \noindent \begin{minipage}{\dimexpr0.4965\textwidth-2\fboxsep-2\fboxrule}
    	    \textit{Un modelo educativo híbrido (en el que hubiese que ir 3 días a clase y 2 días desde casa) resultaría en una mejora del aprendizaje de los alumnos} 	
    \end{minipage} \hspace*{0.3cm}
    \begin{minipage}{\dimexpr0.4965\textwidth-2\fboxsep-2\fboxrule}
     	\begin{tabularx}{\linewidth}{|Y|Y|Y|Y|Y|} \hline
    	     1 & 2 & 3 & 4 & 5  \tabularnewline \hline
    	\end{tabularx}
    \end{minipage}		
    \vspace{0.2cm}
    
\noindent    \textbf{Si la respuesta es No:}
	
    \noindent \begin{minipage}{\dimexpr0.4965\textwidth-2\fboxsep-2\fboxrule}
    	    \textit{Soy competente en el uso de las TIC} 	
    \end{minipage} \hspace*{0.3cm}
    \begin{minipage}{\dimexpr0.4965\textwidth-2\fboxsep-2\fboxrule}
     	\begin{tabularx}{\linewidth}{|Y|Y|Y|Y|Y|} \hline
    	     1 & 2 & 3 & 4 & 5  \tabularnewline \hline
    	\end{tabularx}
    \end{minipage}		
    
    \noindent \begin{minipage}{\dimexpr0.4965\textwidth-2\fboxsep-2\fboxrule}
    	    \textit{El uso de las TIC en el aula me motiva a participar en las actividades grupales propuestas por el profesor} 	
    \end{minipage} \hspace*{0.3cm}
    \begin{minipage}{\dimexpr0.4965\textwidth-2\fboxsep-2\fboxrule}
        \begin{tabularx}{\linewidth}{|Y|Y|Y|Y|Y|} \hline
    	     1 & 2 & 3 & 4 & 5  \tabularnewline \hline
    	\end{tabularx}
    \end{minipage}
    
    \noindent \begin{minipage}{\dimexpr0.4965\textwidth-2\fboxsep-2\fboxrule}
    	    \textit{Me gusta trabajar en equipo porque me considero un igual entre los integrantes del grupo} 	
    \end{minipage} \hspace*{0.3cm}
    \begin{minipage}{\dimexpr0.4965\textwidth-2\fboxsep-2\fboxrule}
    	\begin{tabularx}{\linewidth}{|Y|Y|Y|Y|Y|} \hline
    	     1 & 2 & 3 & 4 & 5  \tabularnewline \hline
    	\end{tabularx}
    \end{minipage}		

    \noindent \begin{minipage}{\dimexpr0.4965\textwidth-2\fboxsep-2\fboxrule}
    	    \textit{Me gustaría que el uso de las TIC en el aula fuese mayor} 	
    \end{minipage} \hspace*{0.3cm}
    \begin{minipage}{\dimexpr0.4965\textwidth-2\fboxsep-2\fboxrule}
     	\begin{tabularx}{\linewidth}{|Y|Y|Y|Y|Y|} \hline
    	     1 & 2 & 3 & 4 & 5  \tabularnewline \hline
    	\end{tabularx}
    \end{minipage}		
    
    \noindent \begin{minipage}{\dimexpr0.4965\textwidth-2\fboxsep-2\fboxrule}
    	    \textit{Cuando trabajo en equipo pienso más en los resultados del equipo que en los míos propios} 	
    \end{minipage} \hspace*{0.3cm}
    \begin{minipage}{\dimexpr0.4965\textwidth-2\fboxsep-2\fboxrule}
     	\begin{tabularx}{\linewidth}{|Y|Y|Y|Y|Y|} \hline
    	     1 & 2 & 3 & 4 & 5  \tabularnewline \hline
    	\end{tabularx}
    \end{minipage}		    
	
    \noindent \begin{minipage}{\dimexpr0.4965\textwidth-2\fboxsep-2\fboxrule}
    	    \textit{Mis profesores utilizan las TIC de una forma que no considero correcta} 	
    \end{minipage} \hspace*{0.3cm}
    \begin{minipage}{\dimexpr0.4965\textwidth-2\fboxsep-2\fboxrule}
    	\begin{tabularx}{\linewidth}{|Y|Y|Y|Y|Y|} \hline
    	     1 & 2 & 3 & 4 & 5  \tabularnewline \hline
    	\end{tabularx}
    \end{minipage}		    
		    
		
    \noindent \begin{minipage}{\dimexpr0.4965\textwidth-2\fboxsep-2\fboxrule}
    	    \textit{Las TIC que usamos en clase están gamificadas, lo cual me ayuda a aprender} 	
    \end{minipage} \hspace*{0.3cm}
    \begin{minipage}{\dimexpr0.4965\textwidth-2\fboxsep-2\fboxrule}
     	\begin{tabularx}{\linewidth}{|Y|Y|Y|Y|Y|} \hline
    	     1 & 2 & 3 & 4 & 5  \tabularnewline \hline
    	\end{tabularx}
    \end{minipage}		    
		
    \noindent \begin{minipage}{\dimexpr0.4965\textwidth-2\fboxsep-2\fboxrule}
    	    \textit{Un modelo educativo híbrido (en el que hubiese que ir 3 días a clase y 2 días desde casa) resultaría en una mejora de mi calidad de vida} 	
    \end{minipage} \hspace*{0.3cm}
    \begin{minipage}{\dimexpr0.4965\textwidth-2\fboxsep-2\fboxrule}
    	\begin{tabularx}{\linewidth}{|Y|Y|Y|Y|Y|} \hline
    	     1 & 2 & 3 & 4 & 5  \tabularnewline \hline
    	\end{tabularx}
    \end{minipage}		    
		
    \noindent \begin{minipage}{\dimexpr0.4965\textwidth-2\fboxsep-2\fboxrule}
    	    \textit{Un modelo educativo híbrido (en el que hubiese que ir 3 días a clase y 2 días desde casa) resultaría en una mejora de mis resultados académicos} 	
    \end{minipage} \hspace*{0.3cm}
    \begin{minipage}{\dimexpr0.4965\textwidth-2\fboxsep-2\fboxrule}
     	\begin{tabularx}{\linewidth}{|Y|Y|Y|Y|Y|} \hline
    	     1 & 2 & 3 & 4 & 5  \tabularnewline \hline
    	\end{tabularx}
    \end{minipage}		    

	\section*{\bfseries Sección 3: Diseño de CoordinApp}
	
    \noindent \begin{minipage}{\dimexpr0.4\textwidth-2\fboxsep-2\fboxrule}
    	    \textit{El diseño de la aplicación me resulta atractivo} 	
    \end{minipage} \hspace*{0.3cm}
    \begin{minipage}{\dimexpr0.5930\textwidth-2\fboxsep-2\fboxrule}
     	\begin{tabularx}{\linewidth}{|Y|Y|Y|Y|Y|} \hline
    	     1 & 2 & 3 & 4 & 5  \tabularnewline \hline
    	\end{tabularx}
    \end{minipage}			
	
    \noindent \begin{minipage}{\dimexpr0.4\textwidth-2\fboxsep-2\fboxrule}
    	    \textit{Las diferentes secciones de la aplicación están bien diferenciadas} 	
    \end{minipage} \hspace*{0.3cm}
    \begin{minipage}{\dimexpr0.5930\textwidth-2\fboxsep-2\fboxrule}
      	\begin{tabularx}{\linewidth}{|Y|Y|Y|Y|Y|} \hline
    	     1 & 2 & 3 & 4 & 5  \tabularnewline \hline
    	\end{tabularx}
    \end{minipage}			
	
    \noindent \begin{minipage}{\dimexpr0.4\textwidth-2\fboxsep-2\fboxrule}
    	    \textit{Me resulta sencillo buscar la información que necesito} 	
    \end{minipage} \hspace*{0.3cm}
    \begin{minipage}{\dimexpr0.5930\textwidth-2\fboxsep-2\fboxrule}
     	\begin{tabularx}{\linewidth}{|Y|Y|Y|Y|Y|} \hline
    	     1 & 2 & 3 & 4 & 5  \tabularnewline \hline
    	\end{tabularx}
    \end{minipage}			
	
    \noindent \begin{minipage}{\dimexpr0.4\textwidth-2\fboxsep-2\fboxrule}
    	    \textit{La paleta de colores de la aplicación me transmite calma} 	
    \end{minipage} \hspace*{0.3cm}
    \begin{minipage}{\dimexpr0.5930\textwidth-2\fboxsep-2\fboxrule}
     	\begin{tabularx}{\linewidth}{|Y|Y|Y|Y|Y|} \hline
    	     1 & 2 & 3 & 4 & 5  \tabularnewline \hline
    	\end{tabularx}
    \end{minipage}			
	
    \noindent \begin{minipage}{\dimexpr0.4\textwidth-2\fboxsep-2\fboxrule}
    	    \textit{El diseño de la aplicación está adecuado a todas las edades} 	
    \end{minipage} \hspace*{0.3cm}
    \begin{minipage}{\dimexpr0.5930\textwidth-2\fboxsep-2\fboxrule}
       	\begin{tabularx}{\linewidth}{|Y|Y|Y|Y|Y|} \hline
    	     1 & 2 & 3 & 4 & 5  \tabularnewline \hline
    	\end{tabularx}
    \end{minipage}			
	
    \noindent \begin{minipage}{\dimexpr0.4\textwidth-2\fboxsep-2\fboxrule}
    	    \textit{El diseño de la aplicación está adecuado a todas las personas independientemente de su familiaridad con la tecnología} 	
    \end{minipage} \hspace*{0.3cm}
    \begin{minipage}{\dimexpr0.5930\textwidth-2\fboxsep-2\fboxrule}
     	\begin{tabularx}{\linewidth}{|Y|Y|Y|Y|Y|} \hline
    	     1 & 2 & 3 & 4 & 5  \tabularnewline \hline
    	\end{tabularx}
    \end{minipage}			
	
    \noindent \begin{minipage}{\dimexpr0.4\textwidth-2\fboxsep-2\fboxrule}
    	    \textit{El diseño me parece innovador} 	
    \end{minipage} \hspace*{0.3cm}
    \begin{minipage}{\dimexpr0.5930\textwidth-2\fboxsep-2\fboxrule}
      	\begin{tabularx}{\linewidth}{|Y|Y|Y|Y|Y|} \hline
    	     1 & 2 & 3 & 4 & 5  \tabularnewline \hline
    	\end{tabularx}
    \end{minipage}			
	\vspace{0.2cm}
		
\noindent	\textbf{Si ha tenido el rol de profesor dentro de la app}\par

    \textit{En caso de que el diseño de la aplicación necesite de cambios y/o mejoras indícalas y cómo se debería hacer}\par

\noindent	\textbf{Si ha tenido el rol de estudiante dentro de la app}\par

    \textit{En caso de que el diseño de la aplicación necesite de cambios y/o mejoras indícalas y cómo se debería hacer}

	\section*{\bfseries Sección 4: Experiencia con el uso de CoordinApp}
	
    \noindent \begin{minipage}{\dimexpr0.4\textwidth-2\fboxsep-2\fboxrule}
    	    \textit{El desarrollo de la aplicación está justificado en el contexto de pandemia que nos encontramos} 	
    \end{minipage} \hspace*{0.3cm}
    \begin{minipage}{\dimexpr0.5930\textwidth-2\fboxsep-2\fboxrule}
     	\begin{tabularx}{\linewidth}{|Y|Y|Y|Y|Y|} \hline
    	     1 & 2 & 3 & 4 & 5  \tabularnewline \hline
    	\end{tabularx}
    \end{minipage}			

    \noindent \begin{minipage}{\dimexpr0.4\textwidth-2\fboxsep-2\fboxrule}
    	    \textit{CoordinApp puede sustituir a un software que conozco con herramientas y funcionalidades parecidas que solo pudiese ser utilizado en un ordenador} 	
    \end{minipage} \hspace*{0.3cm}
    \begin{minipage}{\dimexpr0.5930\textwidth-2\fboxsep-2\fboxrule}
     	\begin{tabularx}{\linewidth}{|Y|Y|Y|Y|Y|} \hline
    	     1 & 2 & 3 & 4 & 5  \tabularnewline \hline
    	\end{tabularx}
    \end{minipage}		
    \vspace{0.2cm}
    
\noindent    \textbf{Si ha tenido el rol de profesor dentro de la app}
    
    \noindent \begin{minipage}{\dimexpr0.4\textwidth-2\fboxsep-2\fboxrule}
    	    \textit{La aplicación me permite coordinar alumnos de forma muy adecuada} 	
    \end{minipage} \hspace*{0.3cm}
    \begin{minipage}{\dimexpr0.5930\textwidth-2\fboxsep-2\fboxrule}
     	\begin{tabularx}{\linewidth}{|Y|Y|Y|Y|Y|} \hline
    	     1 & 2 & 3 & 4 & 5  \tabularnewline \hline
    	\end{tabularx}
    \end{minipage}		
    
    \noindent \begin{minipage}{\dimexpr0.4\textwidth-2\fboxsep-2\fboxrule}
    	    \textit{Me gustaría tener el control de lo que hablan mis alumnos por el chat de grupo en el que están únicamente ellos} 	
    \end{minipage} \hspace*{0.3cm}
    \begin{minipage}{\dimexpr0.5930\textwidth-2\fboxsep-2\fboxrule}
      	\begin{tabularx}{\linewidth}{|Y|Y|Y|Y|Y|} \hline
    	     1 & 2 & 3 & 4 & 5  \tabularnewline \hline
    	\end{tabularx}
    \end{minipage}		
    
    \noindent \begin{minipage}{\dimexpr0.4\textwidth-2\fboxsep-2\fboxrule}
    	    \textit{Las actividades de evaluación y la forma de evaluarlas son muy adecuadas} 	
    \end{minipage} \hspace*{0.3cm}
    \begin{minipage}{\dimexpr0.5930\textwidth-2\fboxsep-2\fboxrule}
    	\begin{tabularx}{\linewidth}{|Y|Y|Y|Y|Y|} \hline
    	     1 & 2 & 3 & 4 & 5  \tabularnewline \hline
    	\end{tabularx}
    \end{minipage}		
    
    \noindent \begin{minipage}{\dimexpr0.4\textwidth-2\fboxsep-2\fboxrule}
    	    \textit{Las actividades realizadas en la aplicación pueden sustituir a las actividades realizadas de forma presencial} 	
    \end{minipage} \hspace*{0.3cm}
    \begin{minipage}{\dimexpr0.5930\textwidth-2\fboxsep-2\fboxrule}
     	\begin{tabularx}{\linewidth}{|Y|Y|Y|Y|Y|} \hline
    	     1 & 2 & 3 & 4 & 5  \tabularnewline \hline
    	\end{tabularx}
    \end{minipage}		
    
    \noindent \begin{minipage}{\dimexpr0.4\textwidth-2\fboxsep-2\fboxrule}
    	    \textit{Las funcionalidades del gestor de grupos son muy útiles} 	
    \end{minipage} \hspace*{0.3cm}
    \begin{minipage}{\dimexpr0.5930\textwidth-2\fboxsep-2\fboxrule}
     	\begin{tabularx}{\linewidth}{|Y|Y|Y|Y|Y|} \hline
    	     1 & 2 & 3 & 4 & 5  \tabularnewline \hline
    	\end{tabularx}
    \end{minipage}		
    
    \noindent \begin{minipage}{\dimexpr0.4\textwidth-2\fboxsep-2\fboxrule}
    	    \textit{Las estadísticas del rendimiento de los alumnos son muy útiles} 	
    \end{minipage} \hspace*{0.3cm}
    \begin{minipage}{\dimexpr0.5930\textwidth-2\fboxsep-2\fboxrule}
      	\begin{tabularx}{\linewidth}{|Y|Y|Y|Y|Y|} \hline
    	     1 & 2 & 3 & 4 & 5  \tabularnewline \hline
    	\end{tabularx}
    \end{minipage}		
    
    \noindent \begin{minipage}{\dimexpr0.4\textwidth-2\fboxsep-2\fboxrule}
    	    \textit{Siento que tengo el control de todo lo que sucede en la aplicación} 	
    \end{minipage} \hspace*{0.3cm}
    \begin{minipage}{\dimexpr0.5930\textwidth-2\fboxsep-2\fboxrule}
      	\begin{tabularx}{\linewidth}{|Y|Y|Y|Y|Y|} \hline
    	     1 & 2 & 3 & 4 & 5  \tabularnewline \hline
    	\end{tabularx}
    \end{minipage}		
    
    \noindent \begin{minipage}{\dimexpr0.4\textwidth-2\fboxsep-2\fboxrule}
    	    \textit{Considero que el rol de portavoz discrimina a los otros integrantes del grupo} 	
    \end{minipage} \hspace*{0.3cm}
    \begin{minipage}{\dimexpr0.5930\textwidth-2\fboxsep-2\fboxrule}
      	\begin{tabularx}{\linewidth}{|Y|Y|Y|Y|Y|} \hline
    	     1 & 2 & 3 & 4 & 5  \tabularnewline \hline
    	\end{tabularx}
    \end{minipage}		
    
    \noindent \begin{minipage}{\dimexpr0.4\textwidth-2\fboxsep-2\fboxrule}
    	    \textit{Usaría esta aplicación para organizar actividades con mis alumnos} 	
    \end{minipage} \hspace*{0.3cm}
    \begin{minipage}{\dimexpr0.5930\textwidth-2\fboxsep-2\fboxrule}
      	\begin{tabularx}{\linewidth}{|Y|Y|Y|Y|Y|} \hline
    	     1 & 2 & 3 & 4 & 5  \tabularnewline \hline
    	\end{tabularx}
    \end{minipage}		
    

    \textit{En caso de que haya que añadir funcionalidades y/o modificar las existentes indícalas y cómo se debería hacer}
    
\noindent    \textbf{Si ha tenido el rol de alumno dentro de la app}
    
    \noindent \begin{minipage}{\dimexpr0.4\textwidth-2\fboxsep-2\fboxrule}
    	    \textit{La aplicación me ha resultado entretenida de usar} 	
    \end{minipage} \hspace*{0.3cm}
    \begin{minipage}{\dimexpr0.5930\textwidth-2\fboxsep-2\fboxrule}
      	\begin{tabularx}{\linewidth}{|Y|Y|Y|Y|Y|} \hline
    	     1 & 2 & 3 & 4 & 5  \tabularnewline \hline
    	\end{tabularx}
    \end{minipage}		
    
    \noindent \begin{minipage}{\dimexpr0.4\textwidth-2\fboxsep-2\fboxrule}
    	    \textit{Las funciones que me permite realizar la aplicación son intuitivas y fáciles de usar} 	
    \end{minipage} \hspace*{0.3cm}
    \begin{minipage}{\dimexpr0.5930\textwidth-2\fboxsep-2\fboxrule}
      	\begin{tabularx}{\linewidth}{|Y|Y|Y|Y|Y|} \hline
    	     1 & 2 & 3 & 4 & 5  \tabularnewline \hline
    	\end{tabularx}
    \end{minipage}		
    
    \noindent \begin{minipage}{\dimexpr0.4\textwidth-2\fboxsep-2\fboxrule}
    	    \textit{He sentido que he tenido responsabilidad de los resultados de mi equipo} 	
    \end{minipage} \hspace*{0.3cm}
    \begin{minipage}{\dimexpr0.5930\textwidth-2\fboxsep-2\fboxrule}
      	\begin{tabularx}{\linewidth}{|Y|Y|Y|Y|Y|} \hline
    	     1 & 2 & 3 & 4 & 5  \tabularnewline \hline
    	\end{tabularx}
    \end{minipage}		
    
    \noindent \begin{minipage}{\dimexpr0.4\textwidth-2\fboxsep-2\fboxrule}
    	    \textit{Me he sentido discriminado por los compañeros de mi equipo} 	
    \end{minipage} \hspace*{0.3cm}
    \begin{minipage}{\dimexpr0.5930\textwidth-2\fboxsep-2\fboxrule}
      	\begin{tabularx}{\linewidth}{|Y|Y|Y|Y|Y|} \hline
    	     1 & 2 & 3 & 4 & 5  \tabularnewline \hline
    	\end{tabularx}
    \end{minipage}		
    
    \noindent \begin{minipage}{\dimexpr0.4\textwidth-2\fboxsep-2\fboxrule}
    	    \textit{El profesor evalúa las respuestas sin saber el nombre del alumno/a que ha contestado ¿Consideras que este es un sistema de evaluación que te beneficia?} 	
    \end{minipage} \hspace*{0.3cm}
    \begin{minipage}{\dimexpr0.5930\textwidth-2\fboxsep-2\fboxrule}
      	\begin{tabularx}{\linewidth}{|Y|Y|Y|Y|Y|} \hline
    	     1 & 2 & 3 & 4 & 5  \tabularnewline \hline
    	\end{tabularx}
    \end{minipage}		
    
    \noindent \begin{minipage}{\dimexpr0.4\textwidth-2\fboxsep-2\fboxrule}
    	    \textit{En caso de ser necesario, me gustaría ser evaluado utilizando esta aplicación} 	
    \end{minipage} \hspace*{0.3cm}
    \begin{minipage}{\dimexpr0.5930\textwidth-2\fboxsep-2\fboxrule}
      	\begin{tabularx}{\linewidth}{|Y|Y|Y|Y|Y|} \hline
    	     1 & 2 & 3 & 4 & 5  \tabularnewline \hline
    	\end{tabularx}
    \end{minipage}		
    
    \noindent \begin{minipage}{\dimexpr0.4\textwidth-2\fboxsep-2\fboxrule}
    	    \textit{Me gustaría tener más control sobre cómo se crean los grupos para elegir directamente con quién quiero estar} 	
    \end{minipage} \hspace*{0.3cm}
    \begin{minipage}{\dimexpr0.5930\textwidth-2\fboxsep-2\fboxrule}
      	\begin{tabularx}{\linewidth}{|Y|Y|Y|Y|Y|} \hline
    	     1 & 2 & 3 & 4 & 5  \tabularnewline \hline
    	\end{tabularx}
    \end{minipage}		
    \vspace{0.2cm}
    
\noindent    \textbf{Si ha tenido el rol de portavoz en algún grupo}
    
    \noindent \begin{minipage}{\dimexpr0.4\textwidth-2\fboxsep-2\fboxrule}
    	    \textit{Siento que he tenido más responsabilidad de los resultados del grupo que los integrantes del mismo} 	
    \end{minipage} \hspace*{0.3cm}
    \begin{minipage}{\dimexpr0.5930\textwidth-2\fboxsep-2\fboxrule}
      	\begin{tabularx}{\linewidth}{|Y|Y|Y|Y|Y|} \hline
    	     1 & 2 & 3 & 4 & 5  \tabularnewline \hline
    	\end{tabularx}
    \end{minipage}		
    
    \noindent \begin{minipage}{\dimexpr0.4\textwidth-2\fboxsep-2\fboxrule}
    	    \textit{Considero que es necesario este rol} 	
    \end{minipage} \hspace*{0.3cm}
    \begin{minipage}{\dimexpr0.5930\textwidth-2\fboxsep-2\fboxrule}
      	\begin{tabularx}{\linewidth}{|Y|Y|Y|Y|Y|} \hline
    	     1 & 2 & 3 & 4 & 5  \tabularnewline \hline
    	\end{tabularx}
    \end{minipage}		
    
   \textit{En caso de que haya que añadir funcionalidades y/o modificar las existentes indícalas y cómo se debería hacer}
    
\noindent    \textbf{Si no ha tenido el rol de portavoz en algún grupo}
    
    \noindent \begin{minipage}{\dimexpr0.4\textwidth-2\fboxsep-2\fboxrule}
    	    \textit{El portavoz ha sido una persona responsable que se ha preocupado de los resultados del equipo} 	
    \end{minipage} \hspace*{0.3cm}
    \begin{minipage}{\dimexpr0.5930\textwidth-2\fboxsep-2\fboxrule}
      	\begin{tabularx}{\linewidth}{|Y|Y|Y|Y|Y|} \hline
    	     1 & 2 & 3 & 4 & 5  \tabularnewline \hline
    	\end{tabularx}
    \end{minipage}		    
	
    \noindent \begin{minipage}{\dimexpr0.4\textwidth-2\fboxsep-2\fboxrule}
    	    \textit{La responsabilidad que se le otorga al portavoz del equipo podría convertirse en una desventaja para mí} 	
    \end{minipage} \hspace*{0.3cm}
    \begin{minipage}{\dimexpr0.5930\textwidth-2\fboxsep-2\fboxrule}
      	\begin{tabularx}{\linewidth}{|Y|Y|Y|Y|Y|} \hline
    	     1 & 2 & 3 & 4 & 5  \tabularnewline \hline
    	\end{tabularx}
    \end{minipage}		 
    
    \textit{En caso de que haya que añadir funcionalidades y/o modificar las existentes indícalas y cómo se debería hacer}

    \newpage
    
    \addcontentsline{toc}{chapter}{Anexo C: Atribuciones a autores de imágenes vectoriales usadas}
 	\chapter*{\bfseries Anexo C: Atribuciones a autores de imágenes vectoriales usadas}
 	
 	\begin{itemize}
 	   \item \textbf{De} \url{https://www.flaticon.es/autores/roundicons}\par
 	    cancelar\par
         comprobando
 	    
 	    \item \textbf{De} \url{https://www.flaticon.es/autores/freepik}\par
 	    
 	    stopwatch\par
        group-1\par
        group\par
        user\par
        usersFolder\par
        people\par
        black-circle\par
        offer\par
        pdf-file\par
        reading-book\par
        teacher-at-the-blackboard\par
        
        \item \textbf{De} \url{https://www.flaticon.com/authors/pixel-perfect}\par
        remove\par
        tap\par
        robot\par
        image\par
        
        \item \textbf{De} \url{https://www.flaticon.com/authors/alfredo-hernandez}\par
        dots
        
        \item \textbf{De} \url{https://www.flaticon.com/authors/smashicons}\par
        questionnaire

        \item \textbf{De} \url{https://www.flaticon.com/authors/icongeek26}\par
        closeFolder\par
        openFolder
 	\end{itemize}
\end{document}